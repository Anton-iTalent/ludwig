%%%%%%%%%%%%%%%%%%%%%%%%%%%%%%%%%%%%%%%%%%%%%%%%%%%%%%%%%%%%%%%%%%%%%%%%%%%%%
%
%  hybrid.tex
%
%  Description of  LB-FD hybrid considerations.
%
%%%%%%%%%%%%%%%%%%%%%%%%%%%%%%%%%%%%%%%%%%%%%%%%%%%%%%%%%%%%%%%%%%%%%%%%%%%%%

\section{BBL and hybrid dynamics}

[I assume this will be preceded by some description of the
governing equations for the tensor order parameter and their
treatment.]

The well-established procedure of bounce-back on links (BBL)
\cite{Ladd04, nguyen} has been employed for the representation
of the colloids as moving solid objects within the LBM.
In the hybrid LB/FD approach used here, BBL is retained for
the distributions, which are simply reflected at the solid-fluid
surface with a correction which depends on the local surface
velocity (see Figure Xa). The resulting change in momentum is
summed over the links to give the net hydrodynamic
force on the colloid, which is then used to update the
particle velocity, and
hence position, in a molecular dynamics-like step. Boundary
conditions for the finite difference equations for the order
parameter tensor are dealt with in a different way.

First, we note that the assignment of solid and fluid lattice
nodes for the order parameter follows that for the density:
inside and outside are distinguished using the nominal radius
of the colloid $a_0$ and its position. It is useful, in addition,
to think about a series of control volumes surrounding each lattice
node whose faces are aligned with the lattice (Figure Xb). A set of
these
faces constitute the solid-fluid boundary in the hybrid picture.

Boundary conditions for $Q_{\alpha\beta}$ are of two types:
homeotropic, where the director  $n_\alpha^0$ is aligned with
the local unit normal to the surface $\hat{n}_\alpha$, and planar,
where the director lies in the plane of the tangent to the
surface. For either choice of director at the surface, we
may set the corresponding value of $Q_{\alpha\beta}$ at
lattice nodes immediately inside the surface via
\begin{equation}
Q_{\alpha\beta}^0 = S^0 (n_\alpha^0 n_\beta^0
- {\scriptstyle\frac{1}{3}}\delta_{\alpha\beta})
\end{equation}
where the constant $S^0$ controls the degree of surface order.
This allows us
to compute, at all fluid nodes, the derivatives
$\nabla_\gamma Q_{\alpha\beta}$ and $\nabla^2 Q_{\alpha\beta}$
using the same finite difference stencil. This allows the
molecular field and hence the diffusive terms in the Beris
Edwards equation to be computed.

Also appearing in the Beris-Edwards equations is the velocity
gradient tensor, which can be handled in a similar fashion
close to the colloid. The velocity field at solid nodes
immediately inside the colloid surface are set to the solid
body velocity $\bf{u} + \bf{\Omega} \times \bf{r}$. Again,
the velocity gradient tensor $\partial_\alpha u_\beta$ may
be computed using the same stencil at all fluid nodes.

Advective fluxes of order parameter are computed at the faces
of the control volumes, and the boundary condition is zero
normal flux at solid-fluid interfaces. Note that the colloid
is assumed to be stationary in assigning these fluxes.

The force on the fluid originating from the order parameter
is computed via the discrete divergence of the stress
$P_{\alpha\beta}$. In the fluid, this is implemented by
interpolating $P_{\alpha\beta}$ to the control volume faces
and taking differences between faces in each direction. This
method has the advantage that, with the introduction of colloids,
an interpolation/extrapolation of $P_{\alpha\beta}$ to the
solid-fluid boundary is possible. This allows one to compute the
divergence of the stress at fluid nodes adjacent to the colloid
in the normal way. At the same time, the discrete equivalent of
\begin{equation}
F_\alpha^\mathrm{coll} = \int P_{\alpha\beta} \hat{n}_\beta dS
\end{equation}
by summing $P_{\alpha\beta}$ over the relevant solid-fluid control
volume faces. By construction, this ensures that momentum lost by
the fluid is gained by the colloid, i.e., global momentum is
conserved.

Finally, movement of the colloid across the lattice is accompanied
by changes in its discrete shape. The events necessitate the
removal or replacement of fluid information. For the replacement
of fluid at newly exposed lattice nodes, this means an
interpolation of nearby order parameter values in the fluid to
provide the new information. This is analogous to what is done
for the LB distributions.


\begin{figure}[h]
\begin{center}

%\input{/home/kevin/doc/rgrid/xfig/hybrid1.epic}
\input{hybrid2.epic}
\end{center}
\caption{The colloid (represented by the solid circle) moves continuously
across the lattice. Lattice sites inside are designated solid, and those
outside fluid (open and closed points, respectively). In the lattice
Boltzmann picture (left) the surface is defined by a set of links
$f_b$, which involve discrete vectors $\mathbf{c}_b \Delta t$ which
connect fluid and solid sites. For the order parameter (right), the
colloid is represented by the set of faces, e.g., that between sites
$i,j$ and $i+1,j$ with unit normal $\hat{n_x}$. Discretisation effects
are found to be negigible for radii greater than about 5 lattice units.}
\end{figure}
