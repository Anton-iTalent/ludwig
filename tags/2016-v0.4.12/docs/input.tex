%%%%%%%%%%%%%%%%%%%%%%%%%%%%%%%%%%%%%%%%%%%%%%%%%%%%%%%%%%%%%%%%%%%%%%%%%%%%%%
%
%  input.tex
%
%  Instructions for contents of the user input file.
%
%  Edinburgh Soft Matter and Statistical Physics Group and
%  Edinburgh Parallel Computing Centre
%
%  (c) 2016 The University of Edinburgh
%
%%%%%%%%%%%%%%%%%%%%%%%%%%%%%%%%%%%%%%%%%%%%%%%%%%%%%%%%%%%%%%%%%%%%%%%%%%%%%%

\section{The Input File}

\subsection{General}

By default, the run time expects to find user input in a file
\texttt{input} in the current working directory. If a different
file name is required, its name
should be provided as the sole command line argument, e.g.,
\begin{lstlisting}
./Ludwig.exe input_file_name
\end{lstlisting}
If the input file is not located in the current working directory
the code will terminate immediately with an error message.

When an input file is located, its content is read by a single MPI
task, and its contents then broadcast to all MPI relevant tasks.
The format of the file is plain ASCII text, and its contents are
parsed on a line by line basis. Lines may contain the following:
\begin{itemize}
\item comments introduced by \texttt{\#}.
\item key value pairs separated by white space.
\end{itemize}
Blank lines are treated as comments. The behaviour of the code is
determined by a set of key value pairs. Any given key may appear
only once in the input file; unused key value pairs are not reported.
If the key value pairs are not correctly formed, the code will terminate
with an error message and indicate the offending input line.

Key value pairs may be present in the input file, but have no effect for
any given run: they are merely ignored. Relevant control parameters for
given input are reported in the standard output.

\subsubsection{Key value pairs}

Key value pairs are made up of a key --- an alphanumeric string with no
white space --- and corresponding value following white space. Values
may take on the follow forms:
\begin{lstlisting}
key_string           value_string

key_integer_scalar  1
key_integer_vector  1_2_3

key_double_scalar    1.0
key_double_vector    1.0_2.0_3.0
\end{lstlisting}
Values which are strings should contain no white space. Scalar parameters
may be integer values, or floating point values with a decimal point
(scientific notation is also allowed).  Vector parameters are introduced
by a set of three values (to be interpreted as $x,y,z$ components of the
vector in Cartesian coordinates) separated by an underscore. The identity
of the key will specify what type of value is expected. Keys and (string)
values are case sensitive.


Most keys have an associated default value which will be used if
the key is not present. Some keys must be specified: an error will
occur if they are missing. The remainder of this part
of the guide details the various choices for key value pairs,
along with any default values, and any relevant constraints.

\subsection{The Free Energy}
\label{input-free-energy}

The choice of free energy is determined as follows:
\begin{lstlisting}
free_energy   none
\end{lstlisting}
The default value is \texttt{none}, i.e., a simple Newtonian fluid is used.
Possible values of the \texttt{free\_energy} key are:
\begin{lstlisting}
#  none                     Newtonian fluid [DEFAULT]
#  symmetric                Symmetric binary fluid (finite difference)
#  symmetric_lb             Symmetric binary fluid (two distributions)
#  brazovskii               Brazovskii smectics
#  surfactant               Surfactants
#  polar_active             Polar active gels
#  lc_blue_phase            Liquid crystal (nematics, cholesterics, BPs)
#  lc_droplet               Liquid crystal emulsions
#  fe_electro               Single fluid electrokinetics
#  fe_electro_symmetric     Binary fluid electrokinetics
\end{lstlisting}

The choice of free energy will control automatically a number of factors
related to choice of order parameter, the degree of parallel communication
required, and so on. Each free energy has a number of associated parameters
discussed in the following sections.

Details of general (Newtonian) fluid parameters, such as viscosity,
are discussed in Section~\ref{input-fluid-parameters}.

\subsubsection{Symmetric Binary Fluids}

We recall that the free energy density is, as a function of compositional
order $\phi$:
\[
{\textstyle \frac{1}{2}} A\phi^2 +
{\textstyle \frac{1}{4}} B\phi^4 +
{\textstyle \frac{1}{2}} \kappa (\mathbf{\nabla}\phi)^2.
\]

Parameters are introduced by (with default values):
\begin{lstlisting}
free_energy  symmetric
A            -0.0625                         # Default: -0.003125
B            +0.0625                         # Default: +0.003125
K            +0.04                           # Default: +0.002
\end{lstlisting}
Common usage has $A < 0$ and $B = -A$ so that $\phi^\star = \pm 1$.
The parameter $\kappa$
(key \texttt{K}) controls
the interfacial energy penalty and is usually positive.

\subsubsection{Brazovskii smectics}
\label{input-brazovskki-smectics}
The free energy density is:
\[
{\textstyle \frac{1}{2}} A\phi^2 +
{\textstyle \frac{1}{4}} B\phi^4 +
{\textstyle \frac{1}{2}} \kappa (\mathbf{\nabla}\phi)^2 +
{\textstyle \frac{1}{2}} C (\nabla^2 \phi)^2 
\]

Parameters are introduced via the keys:
\begin{lstlisting}
free_energy  brazovskii
A             -0.0005                        # Default: 0.0
B             +0.0005                        # Default: 0.0
K             -0.0006                        # Default: 0.0
C             +0.00076                       # Default: 0.0
\end{lstlisting}
For $A<0$, phase separation occurs with a result depending on $\kappa$:
one gets two symmetric phases for $\kappa >0$ (cf.\ the symmetric case)
or a lamellar phase for $\kappa < 0$. Typically, $B = -A$ and the
parameter in the highest derivative $C > 0$.

\subsubsection{Surfactants}
\label{input-surfactants}

The surfactant free energy should not be used at the present time.

\subsubsection{Polar active gels}
\label{input-polar-active-gels}

The free energy density is a function of vector order parameter $P_\alpha$:
\[
{\textstyle \frac{1}{2}} A P_\alpha P_\alpha +
{\textstyle \frac{1}{4}} B (P_\alpha P_\alpha)^2 +
{\textstyle \frac{1}{2}} \kappa (\partial_\alpha P_\beta)^2
\]

There are no default parameters:
\begin{lstlisting}
free_energy        polar_active
polar_active_a    -0.1                       # Default: 0.0
polar_active_b    +0.1                       # Default: 0.0
polar_active_k     0.01                      # Default: 0.0
\end{lstlisting}
It is usual to choose $B > 0$, in which case $A > 0$ gives
an isotropic phase, whereas $A < 0$ gives a polar nematic phase.
The elastic constant $\kappa$ (key \texttt{polar\_active\_k})
is positive.

\subsubsection{Liquid crystal}
\label{input-liquid-crystal}
The free energy density is a function of tensor order parameter
$Q_{\alpha\beta}$:
\begin{eqnarray}
{\textstyle\frac{1}{2}} A_0 (1 - \gamma/3)Q^2_{\alpha\beta} -
{\textstyle\frac{1}{3}} A_0 \gamma
                        Q_{\alpha\beta}Q_{\beta\delta}Q_{\delta\alpha} +
{\textstyle\frac{1}{4}} A_0 \gamma (Q^2_{\alpha\beta})^2
\nonumber \\
+ {\textstyle\frac{1}{2}} \Big(
\kappa_0 (\epsilon_{\alpha\delta\sigma} \partial_\delta Q_{\sigma\beta} +
2q_0 Q_{\alpha\beta})^2 + \kappa_1(\partial_\alpha Q_{\alpha\beta})^2 \Big)
\nonumber
\end{eqnarray}

The corresponding \texttt{free\_energy} value, despite its name, is
suitable for nematics and cholesterics, and not just blue phases:
\begin{lstlisting}
free_energy      lc_blue_phase
lc_a0            0.01                       # Deafult: 0.0
lc_gamma         3.0                        # Default: 0.0
lc_q0            0.19635                    # Default: 0.0
lc_kappa0        0.00648456                 # Default: 0.0
lc_kappa1        0.00648456                 # Default: 0.0
\end{lstlisting}
The bulk free energy parameter $A_0$ is positive and controls the
energy scale (key \texttt{lc\_a0}); $\gamma$ is positive and
influences the position in the phase diagram relative to the
isotropic/nematic transition (key \texttt{lc\_gamma}).
The two elastic constants must be equal, i.e., we enforce the
single elastic constant approximation (boths keys \texttt{lc\_kappa0} and
\texttt{lc\_kappa1} must be specified).

Other important parameters in the liquid crystal picture are:
\begin{lstlisting}
lc_xi            0.7                         # Default: 0.0
lc_Gamma         0.5                         # Default: 0.0
lc_active_zeta   0.0                         # Default: 0.0
\end{lstlisting}
The first is $\xi$ (key \texttt{lc\_xi}) is the effective molecular
aspect ratio and should be in the range $0< \xi< 1$. The rotational
diffusion constant is $\Gamma$ (key \texttt{lc\_Gamma}; not to be
confused with \texttt{lc\_gamma}). The (optional) apolar activity
parameter is $\zeta$ (key \texttt{lc\_active\_zeta}).



\subsubsection{Liquid crystal emulsion}
\label{input-liquid-crystal-emulsion}

This an interaction free energy which combines the symmetric and liquid
crystal free energies. The liquid crystal free energy constant $\gamma$
becomes a
function of composition via $\gamma(\phi) = \gamma_0 + \delta(1 + \phi)$,
and a coupling term is added to the free energy density:
\[
WQ_{\alpha\beta} \partial_\alpha \phi \partial_\beta \phi.
\]
Typically, we might choose $\gamma_0$ and $\delta$ so that
$\gamma(-\phi^\star) < 2.7$ and the $-\phi^\star$ phase is isotropic,
while $\gamma(+\phi^\star) > 2.7$ and the
$+\phi^\star$ phase is ordered (nematic, cholesteric, or blue phase).
Experience suggests that a suitable choice is $\gamma_0 = 2.5$ and
$\delta = 0.25$.

For anchoring constant $W > 0$, the liquid crystal anchoring at the
interface is planar, while for $W < 0$ the anchoring is normal. This
is set via key \texttt{lc\_droplet\_W}.

Relevant keys (with default values) are:
\begin{lstlisting}
free_energy            lc_droplet

A                      -0.0625
B                      +0.0625
K                      +0.053

lc_a0                   0.1
lc_q0                   0.19635
lc_kappa0               0.007
lc_kappa1               0.007

lc_droplet_gamma        2.586                # Default: 0.0
lc_droplet_delta        0.25                 # Default: 0.0
lc_droplet_W           -0.05                 # Default: 0.0
\end{lstlisting}
Note that key \texttt{lc\_gamma} is not set in this case.

\subsection{System Parameters}
\label{input-system-parameters}

Basic parameters controlling the number of time steps
and the system size are:
\begin{lstlisting}
N_start      0                              # Default: 0
N_cycles     100                            # Default: 0
size         128_128_1                      # Default: 64_64_64
\end{lstlisting}
A typical simulation will start from time zero (key \texttt{N\_start})
and run for a certain number of time steps (key \texttt{N\_cycles}).
The system size (key \texttt{size}) specifies the total number of
lattice sites in each dimension. If a two-dimensional system is
required, the extent in the $z$-direction must be set to unity, as
in the above example.

If a restart from a previous run is required, the choice of parameters
may be as follows:
\begin{lstlisting}
N_start      100
N_cycles     400
\end{lstlisting}
This will restart from data previously saved at time step 100, and
run a further 400 cycles, i.e., to time step 500.
% TODO Cross-reference to I/O.

\subsubsection{Parallel decomposition}

In parallel, the domain decompostion is closely related to the
system size, and is specified as follows:
\begin{lstlisting}
size         64_64_64
grid         4_2_1
\end{lstlisting}
The \texttt{grid} key specifies the number of MPI tasks required in
each coordinate direction. In the above example, the decomposition
is into 4 in the $x$-direction, into 2 in the $y$-direction, while
the $z$-direction is not decomposed. In this example, the local domain
size per MPI
task would then be $16\times32\times64$. The total number of MPI tasks
available must match the total implied by \texttt{grid} (8 in the
example).

The \texttt{grid} specifications must exactly divide the system size;
if no decomposition is possible, the code will terminate with an error
message.
If the requested decomposition is not valid, or \texttt{grid} is
omitted, the code will try to supply a decomposition based on
the number of MPI tasks available and \texttt{MPI\_Dims\_create()};
this may be implementation dependent.


\subsection{Fluid Parameters}
\label{input-fluid-parameters}

Control parameters for a Newtonian fluid include:
\begin{lstlisting}
fluid_rho0                 1.0
viscosity                  0.166666666666666
viscosity_bulk             0.166666666666666
isothermal_fluctuations    off
temperature                0.0
\end{lstlisting}
The mean fluid density is $\rho_0$ (key \texttt{fluid\_rho0}) which
defaults to unity in lattice units; it is not usually necessary to
change this. The shear viscosity is
\texttt{viscosity} and as default value 1/6 to correspond to
unit relaxation time in the lattice Boltzmann picture. Reasonable
values of the shear viscosity are $0.2 > \eta > 0.0001$ in lattice
units. Higher values move further into the over-relaxation region, and can
result in poor behaviour. Lower
values increase the Reynolds number and tend to cause
problems with stability. The bulk
viscosity has a default value which is equal to whatever shear
viscosity has been selected. Higher values of the bulk viscosity
may be set independently and can help to suppress large deviations
from incompressibility and maintain numerical stability
in certain situations.

If fluctuating hydrodynamics is wanted, set the value of
 \texttt{isothermal\_fluctualtions} to \texttt{on}. The associated
temperature is in lattice units: reasonable values (at $\rho_0 = 1$)
are $0 < kT < 0.0001$. If the temperature is too high, local
velocities will rapidly exceed the Mach number constraint and
the simulation will be unstable.

\vfill
\pagebreak
