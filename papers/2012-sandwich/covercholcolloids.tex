%\documentclass[aps,prl,groupedaddress,showpacs]{revtex4}
\documentclass[12pt]{article}

\usepackage{graphicx,subfigure}
\usepackage{color}

\pagestyle{empty}

\begin{document}

Dear Editor,\\

We would like to submit the attached manuscript, `Self-assembly of 
colloid-cholesteric composites: A route to switchable optical materials', 
for consideration for publication in {\it Nature Materials}. \\

We believe that our work merits publication in {\it Nature Materials} for the following reasons:\\

1. We give evidence that colloidal dispersions in chiral liquid 
crystals lead to a much larger range of structures than previously
expected. These new soft materials can be realised by tuning
the anchoring of the liquid crystal at the colloidal surface,
or at boundary planes, as well as the particle concentration. 
The geometries we consider are directly 
relevant to the physics of devices which could be built by
using the new materials we find in simulations. All our
structures can therefore be, in principle, found in the lab.

2.  At least one of our new colloid-liquid crystal composite materials 
is switchable by a field. Remarkably, we have also found conditions which 
should lead to the design of a multistable switchable device, which can
retain memory of several different states after field removal:
this is a prerequisite for energy-saving electronics-free
liquid crystal devices (such as electronic paper and smart glass).

3. In scale and scope the work represents a major breakthrough in what 
can be achieved computationally in studying the phase behaviour of soft 
composite materials. \\

Given the substantial nature of the advancement we provide here, 
and its relevance for (i) our understanding of the fundamental behaviour
of multiscale composite materials and (ii) our capabilities to build new
devices based on them, we believe that publication in 
{\it Nature Materials} is warranted.\\

With best wishes,\\

K. Stratford, O. Henrich, J.~S. Lintuvuori, M.~E. Cates, D. Marenduzzo

\end{document}

J.J. de Pablo

T.C. Lubensky

H. Tanaka

R. D. Kamien

H. Stark

J.C. Loudet / Poulin

