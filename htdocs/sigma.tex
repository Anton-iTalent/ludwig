%%%%%%%%%%%%%%%%%%%%%%%%%%%%%%%%%%%%%%%%%%%%%%%%%%%%%%%%%%%%%%%%%%%%%%%%%%%%%%
%
%  sigma.tex
%
%  A rather long section on the calibration of surface tension
%  for the binary fluid models.
%
%%%%%%%%%%%%%%%%%%%%%%%%%%%%%%%%%%%%%%%%%%%%%%%%%%%%%%%%%%%%%%%%%%%%%%%%%%%%%%

\section{Calibration of the Surface Tension}

The binary fluid model introduces an interfacial tension $\sigma$ which
can be predicted from the choice of free energy \cite{swift,kendon}.
In particular, Kendon et al. 2000 described the process of calibrating
the actual interfacial tension of the model fluid. This was based on
a measurement of the Laplace pressure difference between the fluid
inside and outside a spherical droplet via
\begin{equation}
\Delta p = 2 \sigma / R
\end{equation}
where $R$ is the radius of the spherical drop at equilibrium. This
measurement appears quite difficult in practice owing to tendency of
any initial drop to evaporate \cite{yue2007}.

\subsection{New measurement}

Here, we prefer a measure based on the expected profile of the
order parameter $\phi$ and the free energy density at a flat interface.
In order to sample all orientations of the interface with respect to
the lattice we again take a large droplet of fluid A initialised at
rest in fluid B. We assume that the interface is locally flat and so
the procedure holds.

If a droplet with radius $r_0$ is constructed with an initial interfacial
profile of $\phi(r) = \tanh{(r - r_0) / \xi_0}$
with $r$ in the radial direction,  then the free energy density will be
(with $-A = B$)
\begin{equation}
 e(r) = -{\scriptstyle \frac{1}{4}} |A| [1 - \mathrm{sech}^4 (r-r_0)/\xi_0]
+ {\scriptstyle \frac{1}{2}}
 (\kappa/\xi_0^2)\, \mathrm{sech}^4{(r-r_0)/\xi_0}.
\end{equation}
Obtaining a fit to $\xi_0$ from the measured interfacial profile of
$\phi (r)$ and to $ e_0 = \kappa / 2 \xi_0^2$ from the peak in the measured
profile of free energy
density allows an estimate of the interfacial tension to be made:
\begin{equation}
\sigma = 2(8/9)^{1/2} \xi_0 |e_0|.
\end{equation}
This measurement is taken after the initial profile has been allowed
to relax for at least 10 times the larger of the diffusion time
$\xi_0^2 / M$ or the momentum diffusion time $\xi_0^2 / \eta$.


We have measured the surface tension using the following approaches.

\textit{Method 1}.

LB for two distributions with relaxation of order parameter
following Kendon etal. The force on the fluid is applied
via the equilibrium stress following Swift et al.

\textit{Method 2}.

As method 1 but using the realxation of order parameter
following Stratford etal.

\textit{Method 3}.

As Method 1 but with force via divergence of the chemical stress.

\textit{Method 4}.

As Method 3 but with force via divergence of the chemical stress.

\textit{Method 5}.

Using finite difference (1st order upwind) for the order parameter.

\textit{Method 6}.

Using finite difference (3rd order upwind).

\subsection{Results}

We follow the parameter sets in Kendon et al Table~2. These
are reproduced here in Table~\ref{tab:r1}. Measured values for
Run028 parameters are show in Table~\ref{tab:newr028}.

\begin{table}
\begin{center}
\begin{tabular}{llllllll}
\hline
Run & $-A, B$ & $\kappa$ & $\eta$ & $M$ & $\sigma$  & $L_0$ & $T_0$ \\
\hline
Run028 & 0.083 & 0.053 & 1.41 & 0.05  & 0.0625 & 31.8 & 718 \\
Run022 & 0.0625 & 0.04 & 0.5 & 0.25   & 0.0471 & 5.31 & 56.3 \\
Run029 & 0.0625 & 0.04 & 0.2 & 0.15   & 0.0471 & 0.849 & 3.61 \\
Run020 & 0.00625 & 0.004 & 0.025 & 2.00 & 0.00471 & 0.133 & 0.704 \\
Run030 & 0.00625 & 0.004 & 0.0065 & 1.25 & 0.00471 & 0.00829 & 0.0124 \\
Run019 & 0.003125 & 0.002 & 0.0014 & 4.00 & 0.00236 & 0.000831 & 0.000493 \\
Run032 & 0.00125  & 0.0008 & 0.0005 & 5.0 & 0.000943 & 0.000265 & 0.000141\\
\hline  
\end{tabular}
\end{center}
\caption{Table showing common parameter values following Kendon et al.
with the theoretical interfacial tension, $L_0$ and $T_0$. Note that
the order parameter mobility is used here (cf $\tilde{M} = 2M$) in
Kendon et al.}
\label{tab:r1}
\end{table} 

\begin{table}
\begin{tabular}{lllllll}
\hline
Method & $\xi_0$ & $e_{min}$ & $e_{max}$ & $\sigma$  & $\phi_{min}$
& $\phi_{max}$\\
\hline
\multicolumn{7}{c}{Run028}\\
\hline
Theory   & 1.13 & -0.0207 &  0.0207 & 0.0625 & -1.0 & 1.0 \\
Method 1 & 1.00 & -0.0207 & 0.0128 & 0.063(3) & -1.0091 & 1.0028 \\
Method 2 & 1.00 & -0.0207 & 0.0126 & 0.063(1) & -1.0113 & 0.9993 \\
Method 3 & && & & & \\
Method 4 & && & & & \\
Method 5 & && & & & \\
Method 6 & --& --& --& --& --& --\\
\hline
\multicolumn{7}{c}{Run029}\\
\hline
Theory & 1.13 & -0.0156 & 0.0156 & 0.0471  & -1.0 & 1.0\\
Method 2 & 0.99 & -0.0156 & 0.010(?) & 0.048(4) & -1.0110 & 1.0003\\ 
Method 6 & 1.02 & -0.0156 & 0.0099 & 0.049(0) & -1.0048 & 1.0002\\
\hline
\multicolumn{7}{c}{Run030}\\
\hline
Theory & 1.13 & -0.00156 & 0.00156 & 0.00471 & -1.0 & 1.0\\
Method 2 & 1.04 & -0.00156 & 0.0008(?) & 0.0047(1) & -1.0167 & 1.0063\\
Method 6 & 1.08 & -0.00156 & 0.0007(?) & 0.0048(0) & -1.0028 & 1.0031\\
\hline
\multicolumn{7}{c}{Run019}\\
\hline
Theory & 1.13 & -0.000781 & 0.000781 & 0.00236 & -1.0 & 1.0\\
Method 2 & 1.29(?) & -0.000781 & 0.00015(?) & 0.0022(8) & -1.0312 & 1.0242\\
Method 6 & 1.23(?) & 0.0010(?) &&  0.0024(?) & -1.0085 & 1.0019\\
\hline 
\multicolumn{7}{c}{Run032}\\
\hline
Theory   & 1.13 &  -0.000312 & 0.000312 & 0.000943  & -1.0 &1.0\\
Method 2 & 1.07  & -0.000312& 0.000157 & 0.00094(7) & -1.0511 & 1.0311\\
Method 6 & 1.6(5)& -0.000312 & not st. & 0.0010(7) &-1.0056 & 1.0010\\
\hline
\end{tabular}
\caption{Results for the different methods for Run208 parameters.}
\end{table}

\section{Upwind Advection Schemes}

The solution to the Cahn-Hilliard equation for the order parameter
\begin{equation}
\partial_t \phi + \partial_\alpha (u_\alpha \phi + M\partial_\alpha \mu) = 0
\label{eq:ch}
\end{equation}
assumes that the valocity field $u_\alpha$ is known, along with the
order parameter mobility $M$. Adopting a divergence form ensures that
any finite difference scheme conserves the total order parameter in
the system.

Considering just the advective part of \ref{eq:ch}, a finite difference
approach boils down to finding an interpolation of $\phi(\mathbf{r})$ to
the faces of a control volume surrounding the lattice site at $\mathbf{r}$.
So, in one dimension we have
\begin{equation}
\phi_i^{n+1} = \phi_i^n + \frac{\Delta t}{\Delta x} (u_w \phi_w - u_e \phi_e)
\label{eq:phifd}
\end{equation}
where subscripts $w$ and $e$ refer to compass directions with index $i$
incresing eastward. A first order upwind scheme approximates the
interfacial value $\phi_w$ depending on the direction of the velocity
at the face, viz:
\begin{equation}
\phi_w = \left\{
\begin{array}{ll} \phi_i & u_w < 0, \\ \phi_{i-1} & u_w >= 0.
\end{array} \right.
\end{equation}
This choice is conditionally stable for the Euler forward time step
of Eq.~(\ref{eq:phifd}), but highly dissipative.



Better accuracy requires
a higher-order approximation to the interfacial value of $\phi$.

\subsection{Uniformly third order scheme}

We follow Leonard et al. \cite{utopia} in adopting a uniformly third
order approximation dependent upon both the normal and tangential
interfacial Courant numbers. It is extended here to three dimensions.

Consider the two dimensional problem of Figure \ref{fig:utopia}.

The flux at face $W$ is then
\[\begin{array}{ll}
c_x \Big\{ &
1/2(\phi_c + \phi_w) - 1/2c_x (\phi_c - \phi_w)
 - 1/6 (1 -c_x^2)(\phi_c - 2\phi_w + \phi_{ww})\\
& - 1/2c_y (\phi_w - \phi_{sw})\\
& - c_y (1/4 - 1/3c_x)(\phi_c - \phi_w - \phi_s + \phi_{sw})
 - c_y (1/4 - 1/6c_y)(\phi_{nw} - 2\phi_w + \phi_{sw})\\
& - 1/2c_z (\phi_w - \phi_{wd}) \\
& - c_z (1/4 - 1/3c_x) (\phi_c - \phi_w -\phi_{cd} + \phi_{wd})
 - c_z (1/4 - 1/6 c_y) (\phi_{wu} - 2 \phi_w + \phi_{wd}) \\
& + c_y c_z \big[
1/3(\phi_w - \phi_{wd}) - 1/3 (\phi_{sw} - \phi_{swd}) \\
& + (1/6 - 1/4c_x) (\phi_c - \phi_w -\phi_s + \phi_{sw}
    -(\phi_{cd} - \phi_{wd} - \phi_{sd} + \phi_{swd}))\\
& + (1/6 - 1/8c_y)(\phi_{nw} - 2 \phi_w + \phi_{sw} 
                - (\phi_{nwd} - 2\phi_{wd} + \phi_{swd}))\\
& + (1/6 - 1/8c_z)(\phi_{wu} - 2\phi_w + \phi_{wd}
                 - (\phi_{swu} - 2\phi_{sw} + \phi_{swd}))\big] \Big\}
\end{array}\]

\subsection{Gory details}

The flux-integral method of Leonard et al. assumes that within a
given cell, the order parameter is piecewise quadratic. So, for
cell $W$ we have
\begin{eqnarray}
\phi (\zeta, \eta, \theta) = \phi_W - {\scriptstyle \frac{1}{24}} 
(\phi_C + \phi_{SW} + \phi_{WW} + \phi_{NW} + \phi_{WD} + \phi_{WU}
- 6\phi_W)\\
+ {\scriptstyle \frac{1}{2}} (\phi_C - \phi_{WW}) \zeta
+ {\scriptstyle \frac{1}{2}} (\phi_C - 2\phi_{W} + \phi_{WW}) \zeta^2\\
+ {\scriptstyle \frac{1}{2}} (\phi_{NW} - \phi_{SW}) \eta
+ {\scriptstyle \frac{1}{2}} (\phi_{NW} - 2\phi_{W} + \phi_{SW}) \eta^2\\
+ {\scriptstyle \frac{1}{2}} (\phi_{WU} - \phi_{WD}) \theta
+ {\scriptstyle \frac{1}{2}} (\phi_{WU} - 2\phi_{W} + \phi_{WD}) \theta^2
\end{eqnarray}


\section{Lattice kinetic equation viewed as finite difference}

For binary fluid problems the second distribution $g_i (\mathbf{x};t)$,
representing the composition, obeys the evolution equation
\begin{equation}
g_i (\mathbf{x}; t + \Delta t )
= g_i^\star (\mathbf{x} - \mathbf{c}_i \Delta t; t)
\label{lke}
\end{equation}
where the star indicates the post-collision distribution. Here we consider
the distribution to be set by requiring the first moment
\begin{equation}
j_\alpha = \sum_i g_i^\star (\mathbf{x};t) c_{i\alpha} = \phi u_\alpha
\label{eq:moment1}
\end{equation}
and the second moment as
\begin{equation}
\Phi_{\alpha\beta} = \sum_i g_i^\star (\mathbf{x}; t) c_{i\alpha} c_{i\beta}
= \phi u_\alpha u_\beta + 2M\mu (\mathbf{x};t) \delta_{\alpha\beta}.
\end{equation}
Here the mobility $M$ enters through the terms related to the chemical
potential $\mu (\mathbf{x};t)$. The distribution is then set using the
reprojection
\begin{equation}
g_i^\star (\mathbf{x}; t) = \delta_{i0}\phi
+ w_i (j_\alpha u_{i\alpha} / c_s^2
+      \Phi_{\alpha\beta} Q_{i\alpha\beta} / 2c_s^4),
\end{equation}
where the $\delta_{i0}$ moves $\phi$ mostly into the non-propagating
distribution.

For a uniform velocity field $u_\alpha = (u_x, u_y, u_z)$ it is possible
to expand \ref{eq:lke} in a finite difference form which includes three
parts: the advective part in $\phi$ related to the first moment, a
dissipative contribution in $\phi$ related to the $\phi u_\alpha u_\beta$
term in the second moment, and a part related to the
 diffusion of the chemical potential.

\subsection{Advective terms}

Combining \ref{eq:lke} and \ref{eq:moment1} and taking the sum over
the distributions we have
\begin{equation}
\sum_i g_i (\mathbf{x};t) = \sum_i \Big( g_0^\star (\mathbf{x};t)  +
(1/c_s^2)  w_i \phi (\mathbf{x} - \mathbf{c}_i \Delta t;t)
u_\alpha c_{i\alpha} \Big).
\end{equation}
Introducing a finite finiterence notation $\phi_{ijk}^n
= \sum_{i'} g_{i'}^\star (\mathbf{x}; t)$ where the indices $i,j,k$
represent the spatial discretistation, and the superscript $n$ represents
the discrete time level we have
\begin{eqnarray}
\phi_{ijk}^{n+1}  = \phi_{ijk}^n - u_x w_1/c_s^2 \Big\{
\phi_{i+1 j k}^n - \phi_{i-1 j k}^n \\
+ 1/2(
\phi_{i+1 j+1 k}^n - \phi_{i-1 j-1 k}^n +
\phi_{i+1 j-1 k}^n - \phi_{i-1 j+1 k}^n +
\phi_{i+1 j k+1}^n - \phi_{i-1 j k-1}^n +
\phi_{i+1 j k-1}^n - \phi_{i-1 j k-+}^n)
\Big\} \\ +
u_y w_1 / c_s^2 \Big\{
\phi_{i j+1 k}^n - \phi_{i j-1 k}^n \\
+ 1/2 (
\phi_{i+1 j+1 k}^n - \phi_{i-1 j-1 k}^n +
\phi_{i-1 j+1 k}^n - \phi_{i+1 j-1 k}^n +
\phi_{i j+1 k+1}^n - \phi_{i j-1 k-1}^n +
\phi_{i j+1 k-1}^n - \phi_{i j-1 k+1}^n)
\Big\} \\ +
u_z w_1 / c_s^2 \Big\{
\phi_{i j k+1}^n - \phi_{i j k-1}^n \\
+ 1/2 (
\phi_{i+1 j k+1}^n - \phi_{i-1 j k-1}^n +
\phi_{i-1 j k+1}^n - \phi_{i+1 j k-1}^n +
\phi_{i j+1 k+1}^n - \phi_{i j-1 k-1}^n +
\phi_{i j-1 k+1}^n - \phi_{i j+1 k-1}^n) \Big\}.
\end{eqnarray}
We can regcognise here second order centred difference approximations to
the first derivative in $\phi$ in the coordinate direction in the
diagonal directions (cf \cite{sescu2008} with 'isotropy correction factor'
$\beta = 1/2$ in three dimensions).

There are a number of points to note about this. First, there are
terms in $\sum_i g_0^\star$ which are not included above, but are dealt
with below. Second, while the spatial discretisation looks ok, this is
forward in time, which would classically be undesirable. Dissipation
is required to prevent dispersive errors dominating. (Although with
general flow field, the coefficients of the finite difference form are
altered in ways which could provide some upwind bias.)

\subsection{Dissipative terms} 

The dissipative terms from the $\phi u_\alpha u_\beta$ term in the
second moment give
\begin{equation}
\sum_{i'} g_{i'} (\mathbf{x}; t + \Delta t) =
1/2c_s^4 \sum_{i'} w_i \phi (\mathbf{x} - \mathbf{c}_i \Delta t; t)
u_\alpha u_\beta Q_{i'\alpha\beta}.
\end{equation}
With a little effort this may be expanded to give and evolution
equation with the following difusive tendency terms:
\begin{eqnarray}
w_1/2c_s^4  \Big\{
(u_x^2 - c_s^2u^2)(\phi_{i+1 j k} - 2\phi_{ijk} + \phi_{i-1 j k}) \\+
(u_y^2 - c_s^2u^2)(\phi_{i j+1 k} - 2\phi_{ijk} + \phi_{i j-1 k}) \\+
(u_z^2 - c_s^2u^2)(\phi_{i j k+1} - 2\phi_{ijk} + \phi_{i j k-1}) \Big\}\\
+ w_2/2c_s^4 \Big\{
\big[ (u_x^2 + u_y^2) - c_s^2u^2\big]
(\phi_{i+1 j+1 k} - 2\phi_{ijk} + \phi_{i-1 j-1 k}) \\+
\big[ (u_x^2 - u_y^2) - c_s^2u^2 \big]
(\phi_{i+1 j-1 k} - 2\phi_{ijk} + \phi_{i-1 j+1 k}) \\+
\big[ (u_x^2 + u_z^2) - c_s^2u^2 \big]
(\phi_{i-1 j k-1} - 2\phi_{ijk} + \phi_{i+1 j k+1}) \\+
\big[ (u_x^2 - u_z^2) - c_s^2u^2 \big]
(\phi_{i+1 j k-1} - 2\phi_{ijk} + \phi_{i-1 j k+1}) \\+
\big[ (u_y^2 + u_z^2) - c_s^2u^2 \big]
(\phi_{i j-1 k-1} -2\phi_{ijk} + \phi_{i j+1 k+1}) \\+
\big[ (u_y^2 - u_z^2) - c_s^2u^2 \big]
(\phi_{i j+1 k-1} - 2\phi_{ijk} + \phi_{i j-1 k+1}) \Big\}.
\end{eqnarray}
It can be see that the coefficients of these diffusive terms is
related to the aspect ratio of the flow, and that the coefficents
of given terms could become negative.

\subsection{Chemical potential term}

Finally, the terms in the chemical potential give
\begin{equation}
\sum_{i'} g(\mathbf{x};t + \Delta t) =
\sum_{i'} w_i 2M\mu(\mathbf{x} - \mathbf{c}_ \Delta t)
\delta_{\alpha\beta} Q_{i\alpha\beta}.
\end{equation}
Noting that the contraction $\delta_{\alpha\beta} Q_{i\alpha\beta} =
-1, 0, 1$ for velocities with weights $w_0, w_1, w_2$ respectively,
we get
\begin{eqnarray}
\phi_{ijk}^{n+1} = \phi_{ijk}^n -  w_0 2M u_{ijk}/2c_s^4
+ 2M w_2/2c_s^4 \Big\{
\mu_{i+1 j+1 k} + \mu_{i-1 j-1 k} + \mu_{i+1 j-1 k} + \mu_{i-1 j+1 k} + \\
\mu_{i+1 j k+1} + \mu_{i-1 j k-1} + \mu_{i+1 j k-1} + \mu_{i-1 j k+1} +
\mu_{i j+1 k+1} + \mu_{i j-1 k-1} + \mu_{i j+1 k-1} + \mu_{i j-1 k+1} \Big\}.
\end{eqnarray}
This can be combined to give an easily recognisable finite difference
solution for the diffusion equation
\begin{eqnarray}
\phi_{ijk}^{n+1} = \phi_{ijk}^n + 2M w_2 / 2 c_s^4
\Big\{
\mu_{i+1 j+1 k} - 2\mu_{ijk} + \mu_{i-1 j-1 k} + 
\mu_{i+1 j-1 k} - 2\mu_{ijk} + \mu_{i-1 j+1 k} + \\
\mu_{i+1 j k+1} - 2\mu_{ijk} + \mu_{i-1 j k-1} +
\mu_{i+1 j k-1} - 2\mu_{ijk} + \mu_{i-1 j k+1} + \\
\mu_{i j+1 k+1} - 2\mu_{ijk} + \mu_{i j-1 k-1} + 
\mu_{i j+1 k-1} - 2\mu_{ijk} + \mu_{i j-1 k+1} \Big\}.
\end{eqnarray}
Note that this uses the twelve stencil points in $\mu$ related to
the velocities with weight $w_2$ and so should have very good isotropy
properties. The final prefactor is $M$ (check).