%%%%%%%%%%%%%%%%%%%%%%%%%%%%%%%%%%%%%%%%%%%%%%%%%%%%%%%%%%%%%%%%%%%%%%%%%%%%%
%
%  plan.tex
%
%  Software Configuration Management Plan
%
%  See, for example, Annex D of IEEE 828-2012
%  IEEE Standard for Configuration Management in Systems and Software
%  Engineering
%
%  Edinburgh Soft Matter and Statistical Physics Group and
%  Edinburgh Parallel Computing Centre
%
%  (c) 2011-2015 The University of Edinburgh
%  Contributing authors:
%  Kevin Stratford (kevin@epcc.ed.ac.uk)
%
%%%%%%%%%%%%%%%%%%%%%%%%%%%%%%%%%%%%%%%%%%%%%%%%%%%%%%%%%%%%%%%%%%%%%%%%%%%%%


\section{Software Configuration Management Plan}

\subsection{Introduction}

\subsubsection{Purpose}

This section contains information on the \textit{Software Configuration
Management Plan} for the \textit{Ludwig} code.
It is to provide users of the code a background on the way in which
the code is developed, how bugs are dealt with, the way in which
new features may be added, how correctness is ensured and tested, and so on.
It is therefore a part
of the process by which the code is maintained. The Software Configuration
Management Plan will be referred to as `The Plan' throughout the remainder
of the section.

\subsubsection{Scope}

The \textit{Ludwig} code is designed to study simple and complex fluids
based around the numerical solution of the Navier-Stokes equations.
Components of \textit{Ludwig} include the main program, unit and
regression tests, and a small number support libraries, and a small
number of utility programs used to prepare
input and post-process output. The Plan is limited to these
software components.

% IEEE section to be completed
\subsubsection{Relationship to organisation and other projects}

\textit{Ludwig} is an on-going research project, and relies mainly
on funding for the United Kingdom Engineering and Physical Sciences Research
Council to provide support for staff time to work on maintenance and
development.

% IEEE section to be completed
%\subsubsection{Definition of key terms}

\subsubsection{References}

This section is based on the IEEE standard for Configuration
Management IEEE 828-2012 \cite{ieee-828-2012}, and follows the format
set out therein.
Relevant references are included and can be found in the main References
section at the end of the document.
%


% Criteria for identication

\subsection{Identification}

The material under control of the project includes a number of sets
of files: this documentation, the source code, the build and test
suite which is used to monitor the status of the code, and so on.
Control of project files is via a single on-line repository which
uses the subversion (SVN) revision control system. The SVN repository
is currently located at

\texttt{http://ccpforge.cse.rl.ac.uk/}

which is maintained at the Rutherford Appleton Laboratory on behalf
of Collaborative Computational Physics 5 (The Computer Simulation of
Condensed Phases). Subversion identifies different revisions by a
unique revision number. A given version of a file is then uniquely
identified by its path in the repository as it exists at a given
revision number.

Specifically, configuration items are taken to include files in the
trunk of the repository

\texttt{http://ccpforge.cse.rl.ac.uk/svn/ludwig/trunk}

Other branches in the repository are not considered to be 
under configuration management.


% IEEE section
% Limitations and assumptions
% Deliberately empty at the moment.

\subsection{Responsibility and authority}

While the Soft Matter Physics Group is responsible for the overall
scientific direction and development of the code and concomitant
priorities, all Configuration Management activities will
be the responsibility of EPCC at The University of Edinburgh.

\subsection{Project organisation}


\subsubsection{Organisations}

\textit{Ludwig} is developed by a team based in the School of Physics at
The University of Edinburgh. This involves two groups: the Soft Matter
Physics Group, and Edinburgh Parallel Computing Centre (EPCC). These
groups collaborate with a number of workers at different Universities
around the world on different aspects of the code.

\subsubsection{Process Management}

EPCC is responsible for the Software Configuration Management process.
It is anticipated that the cost of this process will be small as the
project team can communicate informally on a regular basis. There is
currently no independent
surveillance of activities to ensure compliance with The Plan.



% Add pointer to standard manifest (README)

%All access to the code is via the CCPForge repository. CCPForge identifies
%three different levels of access:

%\textit{Administrators:} allows administration of the site itself and
%registered members;

%\textit{Developers:} members who have write access to the source code repository;

%\textit{Users:} have read only access to the source code via SVN (either
%as a member or via anonymous download). Member users also have access to
%the tracker.

%As CCPForge is not physically under the control of EPCC, a
%complete dump of the repository to a local resource is undertaken on a
%regular basis for security.



\subsubsection{Control}

Requests for changes or additions to the code should be made via
the issue tracker at CCPForge. The project team will then decide
whether the change is possible and/or desirable. If so, the
implementation and testing of the change will take place, and the
new code committed back to the repository.

\textit{Submitting a change request:} A description of the proposed change
will be submitted to the CCPForge issue tracker as a change request to
provide a record of the change process.

\textit{Evaluating a change request:} The request, together with any
additional information, will be evaluated by the SCM team. The request
will be accepted, refined, or rejected as appropriate. The change owner
will make necessary updates to the change record.

\textit{Implementation of change:} Important changes to behaviour of code
related to a change will be
accompanied by a relevant descriptive record in the change request.

\textit{Version control:} A \texttt{MAJOR.MINOR.PATCH} numerical version
name scheme is used \cite{apacheAPR}.
It is expected that bug fixes and relatively
small changes will be accompanied by a unit increment in the patch level;
more significant changes will be reflected at the minor version number;
major reconstructions will occur rarely and incur a change of major version
number.

\subsubsection{Status}

% Privides tracability of changes, pending completed, etc
% Configuration reports, release notes

The status of changes will be tracked at the CCPForge issue tracker
with a named SCM team member owning the change. The relevant tracker
item will be closed when successfully implemented and tested.

The tracker is available to CPPForge member users to provide a record
of changes to the code. Anonymous users will be alerted to changes via
release notes.

%\subsubsection{Evaluation and review}
%There are currently no formal mechna audit the code?

% \subsubsection{Interface control}
% This section discusses other organisations the developers must
% ``interface'' with.

% IEEE section not relevant
%\subsubsection{Subcontractor / vendor control}


\subsubsection{Release management and delivery}

There are currently no formal releases of the \textit{Ludwig} code.
Any release management and  delivery will be via the SVN repository.


% IEEE section. We don't really have any schedules.
%\subsection{Schedules}

%\subsection{Resources}

% Software resources (dependencies, etc)
% Hardware resources
% Human resources

\subsection{Plan Maintenance}

The Plan is currently under review to consider organisational changes
affecting the responsibilities different stakeholders.



%\begin{thebibliography}{99}

%Software Configuration Management Plans''
%\bibitem{mpi-standard} Message Passing Interface Forum. MPI: A Message
%Passing Interface Standard Version 1.3 (1998).
%\bibitem{paraview} Paraview is a standard visualisation package. 
