%%%%%%%%%%%%%%%%%%%%%%%%%%%%%%%%%%%%%%%%%%%%%%%%%%%%%%%%%%%%%%%%%%%%%%%%%%%%%%
%
%  intro.tex
%
%  Contains introduction and quick start for users.
%
%  Edinburgh Soft Matter and Statistical Physics Group and
%  Edinburgh Parallel Computing Centre
%
%  (c) 2016-2023 The University of Edinburgh
%
%  This material is being migrated to https://ludwig.epcc.ed.ac.uk/index.html
%
%%%%%%%%%%%%%%%%%%%%%%%%%%%%%%%%%%%%%%%%%%%%%%%%%%%%%%%%%%%%%%%%%%%%%%%%%%%%%%

\section{Introduction}

\subsection{Overview}

This introduction provides a brief overview of what the \textit{Ludwig} code
does, how to obtain and build it, and how to run a sample problem.
It is assumed that the reader has at least a general knowledge
of Navier-Stokes hydrodynamics, complex fluids, and to some
extent statistical physics. This knowledge will be required to make sense
of the input and output involved in using the code. Those wanting
to work on or develop the code itself will need knowledge of ANSI C,
and perhaps message passing and CUDA. We assume the reader is using
a Unix-based system.

\subsubsection{Purpose of the code}

The \textit{Ludwig} code has been developed over a number of
years to address specific problems in complex fluids. The underlying
hydrodynamic model is based on the lattice Boltzmann equation
(LBE, or just `LB'). This itself may be used to study simple
(Newtonian) fluids in a number of different scenarios, including porous
media and particle suspensions. However, the
code is more generally suited to complex fluids, where a number of
options are available, among others: symmetric binary fluids and
Brazovskii smectics,
polar gels, liquid crystals, or charged fluid via a Poisson-Boltzmann
equation approach. These features are added in the framework of a
free energy approach, where specific compositional or orientational
order parameters are evolved according to the appropriate
coarse-grained dynamics, but also interact with the fluid in a
fully coupled fashion.

A number of other features are catered for is some or all situations.
These include fluctuating hydrodynamics, and appropriate fluctuating
versions of order parameter dynamics for binary solvents and liquid
crystals. Colloidal particles are available for simulation of
suspensions and dispersions, and with appropriate boundary
conditions binary mixtures, liquid crystals, and charged fluid. Shear
flow is available for fluid-only systems implemented via Lees-Edwards
sliding periodic boundaries. Not all these features play with all the
others: check the specific section of the document for details.

Broadly, the code is intended for complex fluid problems at low
Reynolds numbers, so there is no consideration of turbulence,
high Mach number flows, high density ratio flows, and so on.

\subsubsection{How the code works}

We aim to provide a robust and portable code, written in ANSI C, which
can be used to perform serial and scalable parallel simulations of
complex fluid systems based around hydrodynamics via the lattice
Boltzmann method. Time evolution of modelled quantities takes
place on a fixed regular discrete lattice. The preferred method of
dealing with the corresponding order parameter equations is by
using finite difference. However, for the case of a binary fluid,
a two-distribution lattice Boltzmann approach is also maintained
for historical reference.

Users control the operation of the code via a plain text input file;
output for various data are available. These data may be visualised
using appropriate third-party software (e.g., Paraview). Specific
diagnostic output may require alterations to the code.

Potential users should also note that the complex fluid simulations
enabled by \textit{Ludwig} can be time consuming, prone to instability,
and provide results which are difficult to
interpret. We make no apology for this: that's the way it is.

\subsubsection{Who should read what?}

This documentation is aimed largely at users, but includes information
that will also be relevant for developers. The documentation discusses
the underlying features of the coordinate system, hydrodynamics (LB), and
other generic features, and has a separate section for each free energy.
You will need to consult the section relevant to your problem.


\subsection{Obtaining the code}

The code is publicly available from
\texttt{http://github/ludwig-cf/ludwig/}
which provides revision control via git, issue tracking, and so on.

The current stable release is available as the master branch.

\subsection{Note on Units}

All computation in \textit{Ludwig} is undertaken in ``lattice
units,'' fundamentally related to the underlying LB fluid model
which expects discrete model space and time steps
$\Delta x = \Delta t = 1$. The natural way to approach problems
is then to ensure that appropriate dimensionless quantities are
reasonable. However, ``reasonable'' may not equate to ``matching
experimental values''. For example, typical flows in colloidal
suspensions my exhibit Reynolds numbers as low as $O(10^{-6})$
to $O(10^{-8})$.
Matching such a value in a computation may imply an impractically
small time step; a solution is to let the Reynolds number rise
artificially with the constraint that it remains small compared to $O(1)$.
Further discussion of the issue of units is provided in, e.g.,
\cite{cates_scaling}. Again, consult the relevant section of the documentation
for comments on specific problems.


\vfill
\pagebreak
