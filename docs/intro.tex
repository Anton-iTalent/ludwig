%%%%%%%%%%%%%%%%%%%%%%%%%%%%%%%%%%%%%%%%%%%%%%%%%%%%%%%%%%%%%%%%%%%%%%%%%%%%%%
%
%  intro.tex
%
%  Contains introduction and quick start for users.
%
%  Edinburgh Soft Matter and Statistical Physics Group and
%  Edinburgh Parallel Computing Centre
%
%  (c) 2016 The University of Edinburgh
%
%%%%%%%%%%%%%%%%%%%%%%%%%%%%%%%%%%%%%%%%%%%%%%%%%%%%%%%%%%%%%%%%%%%%%%%%%%%%%%

\section{Introduction}

\subsection{Overview}

This introduction provides a brief overview of what the \textit{Ludwig} code
does, how to obtain and build it, and how to run a sample problem.
It is assumed that the reader has at least a general knowledge
of Navier-Stokes hydrodynamics, complex fluids, and to some
extent statistical physics. This knowledge will be required to make sense
of the input and output involved in using the code. Those wanting
to work on or develop the code itself will need knowledge of ANSI C,
and perhaps message passing and CUDA. We assume the reader is using
a Unix-based system.

\subsubsection{Purpose of the code}

The \textit{Ludwig} code has been developed over a number of
years to address specific problems in complex fluids. The underlying
hydrodynamic model is based on the lattice Boltzmann equation
(LBE, or just `LB'). This itself may be used to study simple
(Newtonian) fluids in a number of different scenarios, including porous
media and particle suspensions. However, the
code is more generally suited to complex fluids, where a number of
options are available, among others: symmetric binary fluids and
Brazovskii smectics,
polar gels, liquid crystals, or charged fluid via a Poisson-Boltzmann
equation approach. These features are added in the framework of a
free energy approach, where specific compositional or orientational
order parameters are evolved according to the appropriate
coarse-grained dynamics, but also interact with the fluid in a
fully coupled fashion.

A number of other features are catered for is some or all situations.
These include fluctuating hydrodynamics, and appropriate fluctuating
versions of order parameter dynamics for binary solvents and liquid
crystals. Colloidal particles are available for simulation of
suspensions and dispersions, and with appropriate boundary
conditions binary mixtures, liquid crystals, and charged fluid. Shear
flow is available for fluid-only systems implemented via Lees-Edwards
sliding periodic boundaries. Not all these features play with all the
others: check the specific section of the document for details.

Broadly, the code is intended for complex fluid problems at low
Reynolds numbers, so there is no consideration of turbulence,
high Mach number flows, high density ratio flows, and so on.

\subsubsection{How the code works}

We aim to provide a robust and portable code, written in ANSI C, which
can be used to perform serial and scalable parallel simulations of
complex fluid systems based around hydrodynamics via the lattice
Boltzmann method. Time evolution of modelled quantities takes
place on a fixed regular discrete lattice. The preferred method of
dealing with the corresponding order parameter equations is by
using finite difference. However, for the case of a binary fluid,
a two-distribution lattice Boltzmann approach is also maintained
for historical reference.

Users control the operation of the code via a plain text input file;
output for various data are available. These data may be visualised
using appropriate third-party software (e.g., Paraview). Specific
diagnostic output may require alterations to the code.

Potential users should also note that the complex fluid simulations
enabled by \textit{Ludwig} can be time consuming, prone to instability,
and provide results which are difficult to
interpret. We make no apology for this: that's the way it is.

\subsubsection{Who should read what?}

This documentation is aimed largely at users, but includes information
that will also be relevant for developers. The documentation discusses
the underlying features of the coordinate system, hydrodynamics (LB), and
other generic features, and has a separate section for each free energy.
You will need to consult the section relevant to your problem.


\subsection{Obtaining the code}

Under the auspices of the CCPs (Collaborative Computational Projects),
and in particular CCP5, we have a CCPForge project account at
\texttt{http://ccpforge.cse.rl.ac.uk/}
which provides revision control via SVN, issue tracking, and so on.

\paragraph{Users:}

Code releases may be downloaded via anonymous download
\begin{lstlisting}[style=terminalverbatim]
http://ccpforge.cse.rl.ac.uk/gf/project/ludwig/frs/
\end{lstlisting}
The current release is available as \texttt{ludwig-0.4.0.tar.gz}.
Alternatively, registered users can obtain the code via svn checkout
in the same way as developers.


\paragraph{Developers:}

Developers need to create an account at CCPForge, and let one of the
project owners\footnote{Currently: \texttt{kevin@epcc.ed.ac.uk},
\texttt{ohenrich@epcc.ed.ac.uk}}
know the user name so you can be added to the access list.

To download the entire code, developers need
\begin{lstlisting}
svn checkout --username <user> http://ccpforge.cse.rl.ac.uk/svn/ludwig
\end{lstlisting}
where \texttt{<user>} is replaced by your CCPForge user name. You will
see
\begin{lstlisting}
$ ls ludwig
branches  tags      trunk
\end{lstlisting}
which is the usual svn convention for trunk, branches and tags.
The process for committing code is described in section \ref{xref:developers}.

If you look in  the \texttt{trunk} directory, you should see
\begin{lstlisting}
$ ls trunk
config  LICENSE      mpi_s   src       tests  version.h
docs    Makefile.mk  README  targetDP  util
\end{lstlisting}

\vfill
\pagebreak

\section{Quick Start for Users}

This section provides a brief overview of compiling and running
the code, which should allow users to get off the ground.
The section assumes the code has been downloaded and you are in the
top level directory.

\subsection{Configuration}

You will need to copy one of the existing configuration files from
the \texttt{config} directory to the top level directory. This file
sets all the relevant local details for compilation and so on. For
example:
\begin{lstlisting}
$ cp config/lunix-gcc-default.mk config.mk
\end{lstlisting}
Edit the new \texttt{config.mk} file to check, or adjust, the options
as required. You will need to identify your local C compiler. The
example uses \texttt{gcc}
in serial, and \texttt{mpicc} in parallel. For MPI in particular,
local details may vary. A quick way to compile the code is via
\begin{lstlisting}
$ cd tests
$ make compile-mpi-d3q19
\end{lstlisting}
This integrates a number of separate stages which are described
indivually in the following sections. The resulting executable
will be \texttt{./src/Ludwig.exe} at the top level.

\subsubsection{targetDP}

A \texttt{targetDP} layer is required by the code in all cases. It
allows compilation for both simple CPU systems and
for a number of accelerator systems using a single source code. This
will make the appropriate choices depending what has been specified
in the top-level \texttt{config.mk} file. From the top-level directory
\begin{lstlisting}
$ cd targetDP
$ make
\end{lstlisting}
should produce the appropriate \texttt{libtarget.a} library.

\subsubsection{Serial}
\label{section:quick-serial}

First, compile the MPI stub library in the \texttt{mpi\_s}
directory. You should be able to build the library and run the
tests using, e.g.:

\begin{lstlisting}
$ make libc
gcc -Wall -I.  -c mpi_serial.c
ar -cru libmpi.a mpi_serial.o
$ make testc
gcc -Wall   mpi_tests.c -L. -lmpi
./a.out
Running mpi_s tests...
Finished mpi_s tests ok.
\end{lstlisting}

Now compile the main code in the \texttt{src} directory.
Compilation should look like:

\begin{lstlisting}
$ make serial
make serial-d3q19
make serial-model "LB=-D_D3Q19_" "LBOBJ=d3q19.o"
make lib
make libar "INC=-I. -I../mpi_s" "LIBS= -L../mpi_s -lmpi -lm"
gcc -D_D3Q19_ -Wall -O2 -I. -I../mpi_s -c model.c
gcc -D_D3Q19_ -Wall -O2 -I. -I../mpi_s -c propagation.c
...
\end{lstlisting}
which shows that the default lattice Boltzmann model is D3Q19.
Successful compilation will provide an executable \texttt{Ludwig.exe}
which is linked against the MPI stub library.

\subsubsection{Parallel}

Here, you do not need to compile the stub library. Simply compile
the main code with, e.g.,:
\begin{lstlisting}
$ make mpi
make mpi-d3q19
make mpi-model "LB=-D_D3Q19_" "LBOBJ=d3q19.o"
make libmpi
make libar "CC=mpicc"
mpicc -D_D3Q19_ -Wall -O2 -I. -c model.c
mpicc -D_D3Q19_ -Wall -O2 -I. -c propagation.c
...
\end{lstlisting}

Again, this should produce an executable \texttt{./Ludwig.exe}
in the current directory, this time linked against the true MPI
library.

\subsection{Running an Example}

\subsubsection{Input file}

The behaviour of \textit{Ludwig} at run time is controlled by
a plain text input file which consists of a series for key value
pairs. Each key value pair controls a parameter or parameters
within the code, for example the system size and number of time steps,
the fluid properties density and viscosity, the free energy type
and associated parameters (if required), frequency and type of
output, parallel decomposition, and so on.

For most keys, the associated property has some default value which
will be used by the code if the relevant key is not present (or
commented out) in the input file. While such default values are
chosen to be at least sensible, users need to be aware that all
necessary keys need to be considered for a given problem. However,
many keys are irrelevant for any given problem.

A significant number of example input files are available as
part of the test suite, and these can form a useful starting
point for a given type of problem. We will consider one which
uses some of the common keys.

\subsubsection{Example input}

We will consider a simple example which computes the motion of
a single spherical colloidal particle in an initially stationary
fluid of given viscosity. The velocity may be measured as a
function of time. Assume we have an executable in the \texttt{src}
directory.

Make a copy of the input file
\begin{lstlisting}
$ cp ../tests/regression/d3q19/serial-auto-c01.inp .
\end{lstlisting}

Inspection of this file will reveal the following: blank lines
and comments --- lines beginning with \texttt{\#} --- may be included,
but are ignored at run time; other lines are parsed as key value
pairs, each pair on a separate line, and with key and value separated
by one or more spaces. Values may be characters or strings, a
scalar integer or 3-vector of integers, or a scalar or 3-vector
of floating point numbers. Values which are 3-vectors are delineated
by an underscore character (not a space), e.g., \texttt{1\_2\_3},
and always correspond to a Cartesian triple $(x,y,z)$.
A given value is parsed appropriately in the context of the associated key.

So, the first few lines of the above file are:
\lstinputlisting[firstline=1, lastline= 16]
{../tests/regression/d3q19/serial-auto-c01.inp}
Here, the first key value pair \texttt{N\_cycles 40} sets the number
of time steps to 40, while the second, \texttt{size 64\_64\_64}, sets
the system size to be 64 points in each of the three coordinate
directions. The \texttt{grid} key relates to the parallel
domain decomposition as is discussed in the following
section.

\subsubsection{Run time}

The executable takes a single argument on the command line which
is the name of the input file, which should be in the same
directory as the executable. If no input file name is supplied,
the executable will look for a default \texttt{input}. If no
input file is located at all, an error to that effect will be
reported.

\textbf{Serial: }
If a serial executable is run in the normal way with the copy of the
input file as the argument the code should take around 10--20 seconds
to execute 40 steps. The fist few lines of output should look like:

\begin{lstlisting}[belowskip=0pt]
$ ./Ludwig.exe ./serial-auto-c01.inp
\end{lstlisting}
\lstinputlisting[aboveskip=0pt, firstline=1, lastline= 16]
{../tests/regression/d3q19/serial-auto-c01.log}
This output shows that the appropriate input file has been read, and
the system size set correspondingly with a number of default settings
including periodic boundary conditions. ``No free energy'' tells us we
are using a single Newtonian fluid.

Normal termination of execution is accompanied by a report
of the time take by various parts of the code, and finally by
\begin{lstlisting}
...
Ludwig finished normally.
\end{lstlisting}

%\paragraph{Parallel: }
\textbf{Parallel: }
The executable compiled and linked against the MPI library can
be run with the MPI launcher for the local system, often
\texttt{mpirun}. For example, a run on 8 MPI tasks produces a
similar output to that seen in the serial case, but reports
details of the local domain decomposition:
\begin{lstlisting}
$ mpirun -np 8 ./Ludwig.exe ./serial-auto-c01.inp
Welcome to Ludwig v0.1.26 (MPI version running on 8 processes)
...
System details
--------------
System size:    64 64 64
Decomposition:  2 2 2
Local domain:   32 32 32
...
\end{lstlisting}
The decomposition is controlled by the \texttt{grid} key in
the input. Here, \texttt{grid 2\_2\_2} is consistent with the
8 MPI tasks specified on the command line, and the resulting
local decomposition is 32 lattice points in each coordinate direction.
If no grid is specified in the input, or a request is
make for a grid which cannot be met (e.g., the product
of the grid dimensions does not agree with the total number
of MPI tasks available)
the code will try to determine its own decomposition as best
it can. If no valid parallel decomposition is available at all,
the code will exit with a message to that effect.

Further details of various input key value pairs are given in
relevant sections of the documentation.

\subsubsection{Output}

The standard output of the running code produces a number of
aggregate quantities which allow a broad overview of the progress
of the computation to be seen by the user. These include, where
relevant, global statistics related to fluid density and momentum,
the integrated free energy, particle-related quantities, and so on.
The frequency of this information can be adjusted from the input
file (see, e.g., \texttt{freq\_statistics}).

Output for lattice-based quantities (for example, the velocity field
$u(\mathbf{r}; t)$) is via external file. This output is in serial
or in parallel according to how the model is run, and may be in
either ASCII or raw binary format as required. Output files are produced
with time step encoded in the file, and an extension which describes the
parallel output decomposition. Further, each quantity output in this way
is accompanied by a metadata description with \texttt{meta} extension.
For example, the two output files
\begin{lstlisting}
bash-3.2$ ls dist*
dist-00000020.001-001   dist.001-001.meta
\end{lstlisting}
contain the LB distributions for time step 20 (the file is 1 of a total
number of 1 in parallel), and the plain text metadata description,
respectively.

To ensure output for based-based quantities is in the correct order for
analysis, post-processing
may be required (always if the code is run in parallel). This uses a utility
provided for the purpose which required both the data and the metadata
description to recombine the parallel output. This utility is described
ELSEWHERE.

Output for colloidal particle data is again to file with a name encoding
the time step and parallel docomposition of the output. For example,
\begin{lstlisting}
bash-3.2$ ls config*
config.cds00000020.001-001
\end{lstlisting}
contains particle data for time step 20 in a format described ELSEWHERE.
These data may be requested in ASCII or raw binary format.

\subsubsection{Errors and run time failures}

The code attempts to provide meaningful diagnostic error messages
for common problems. Such errors include missing or incorrectly
formatted input files, inconsistent input values, and unacceptable
feature combinations. The code should also detect other run time
problems such as insufficient memory and errors writing output
files. These errors will result in termination.

Instability in the computation will often be manifested by numerically
absurd values in the statistical output (ultimately \texttt{NaN} in
many cases). Instability may or may not result in termination, depending
on the problem. Such instability is very often related to poor parameter
choices, of which there can be many combinations. Check the
relevant section of the documentation for advice on reasonable starting
parameters for different problems.


\subsection{Note on Units}

All computation in \textit{Ludwig} is undertaken in ``lattice
units,'' fundamentally related to the underlying LB fluid model
which expects discrete model space and time steps
$\Delta x = \Delta t = 1$. The natural way to approach problems
is then to ensure that appropriate dimensionless quantities are
reasonable. However, ``reasonable'' may not equate to ``matching
experimental values''. For example, typical flows in colloidal
suspensions my exhibit Reynolds numbers as low as $O(10^{-6})$
to $O(10^{-8})$.
Matching such a value in a computation may imply an impractically
small time step; a solution is to let the Reynolds number rise
artificially with the constraint that it remains small compared to $O(1)$.
Further discussion of the issue of units is provided in, e.g.,
\cite{cates_scaling}. Again, consult the relevant section of the documentation
for comments on specific problems.


\vfill
\pagebreak
