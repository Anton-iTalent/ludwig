%%%%%%%%%%%%%%%%%%%%%%%%%%%%%%%%%%%%%%%%%%%%%%%%%%%%%%%%%%%%%%%%%%%%%%%%%%%%%%
%
%  lc-emulsion.tex
%
%  Liquid Crystal Emulsions.
%
%  Edinburgh Soft Matter and Statistical Physics Group and
%  Edinburgh Parallel Computing Centre
%
%  Contributing Authors:
%
%  Adriano Tiribocchi made the original draft.
%  Juho lintuvuori (juho.lintuvuori@u-bordeaux.fr)
%  Kevin Stratforrd (kevin@epcc.ed.ac.uk)
%
%%%%%%%%%%%%%%%%%%%%%%%%%%%%%%%%%%%%%%%%%%%%%%%%%%%%%%%%%%%%%%%%%%%%%%%%%%%%%%


\section{Liquid Crystal Emulsions}

This section describes the approach used to model liquid crystal
emulsions which might include, for example, droplets of liquid
crystal in water. This combines elements of the binary fluid model
discussed in Section~\ref{section-binary} with the liquid crystal
approach of Section~\ref{section-lc}. 

\subsection{Combined Approach}

\subsubsection{Order parameters}

To describe a liquid crystal emulsion, two order parameters are used:
the composition $\phi({\mathbf r}; t)$ and the orientational order
$Q_{\alpha\beta}({\mathbf r}; t)$.


\subsubsection{Free energy densities}

The free energy density is here the sum of three contributions:
\begin{equation}
f(\phi, Q_{\alpha\beta}) = f_{\rm c}(\phi) + f_{\rm lc}(Q_{\alpha\beta})
                         + f_{\rm anchoring}(\phi, Q_{\alpha\beta}).
\end{equation}

The first contribution relates to the composition using the standard
symmetric approach
\begin{equation}
f_{\rm c}(\phi) =
 {\textstyle\frac{1}{2}}A\phi^2
+ {\textstyle\frac{1}{4}}B\phi^4
+ {\textstyle\frac{1}{2}}\kappa (\partial_\alpha \phi)^2
\end{equation}
(cf.\ Equation~\ref{eq-binary-fed}),
where we recall that in the separated regime $A$ is a negative constant,
$B$ is a positive constant, and $\kappa$ is a positive constant which
determines the interfacial tension. The extremal values of the
composition are $\phi^\star = (-A/B)^{1/2}$ while
uniformly mixed components in equal proportion have $\phi = 0$.

The second contribution is the liquid crystal bulk free energy density
(Equation~\ref{eq-lc-fed-bulk}):
\begin{eqnarray}
f_{\rm lc} &=&
  {\textstyle\frac{1}{2}}A_0 \left(1-\gamma(\phi)/3\right) Q_{\alpha\beta}^2
- {\textstyle\frac{1}{3}}A_0 \gamma(\phi)
  Q_{\alpha\beta} Q_{\beta\pi} Q_{\pi\alpha}
+ {\textstyle\frac{1}{4}} A_0 \gamma(\phi) (Q_{\alpha\beta})^2\nonumber\\
&+&{\textstyle\frac{1}{2}} \kappa_0 (\partial_\alpha Q_{\alpha\beta})^2
+ {\textstyle\frac{1}{2}} \kappa_1
(\epsilon_{\alpha\mu\nu} \partial_\mu Q_{\nu\beta} + 2q_0 Q_{\alpha\beta})^2,
\label{eq-lc-fed}
\end{eqnarray}
where in the first line of Equation~\ref{eq-lc-fed} the composition enters parametrically through the control
parameter $\gamma = \gamma(\phi)$. This will be discussion further
below. The distortion terms in the
second line of Equation~\ref{eq-lc-fed} related to the elastic constants $\kappa_0$ and $\kappa_1$
are unchanged (and are as appear in Equation~\ref{equation-lc-gradient-fe}).

Finally, there is a coupling between the liquid crystal and the
composition which determines the anchoring (normal or planar) at
the boundary of the droplet:
\begin{equation}
f_{\rm anchoring}(\phi,Q_{\alpha\beta}) =
W Q_{\alpha\beta}\partial_{\alpha}\phi\partial_{\beta}\phi.
\end{equation}
For $W>0$ the anchoring is planar, and for $W<0$ the anchoring is normal.

\subsubsection{Control of the isotropic/nematic transition}
The function $\gamma(\phi)$ allows control of the liquid crystal
ordering as the composition changes. An appropriate choice of function
for the case of $\phi^\star = 1$ is
\begin{equation}
\gamma(\phi)=\gamma_0+\Delta\left(1+\phi\right)
\end{equation}
so that $\gamma(\phi=-1)=\gamma_0$, and $\gamma(\phi=1)=\gamma_0+2\Delta$.
We should choose $\gamma_0$ and $\Delta$ such that $\gamma(\phi=-1)$ 
is smaller than 2.7 (so that for $\phi=-1$ the fluid is isotropic)
and $\gamma(\phi=+1)$ is larger than 2.7 (so that when $\phi=1$ the
liquid crystal is in the ordered phase, nematic or cholesteric etc.).
Experience suggests that for example 
$\gamma_0=2.5$ and $\Delta=0.25$ is a suitable choice.

\subsubsection{Chemical potential}

The chemical potential $\mu(\phi,Q_{\alpha\beta})$ resulting from the
combined free energy density is the functional derivative of the
total energy with respect to the composition~$\phi$, 

\begin{equation}
\mu = \mu_{\rm c} + \mu_{\rm lc} + \mu_{\rm anchoring} = \frac{\delta F}{\delta \phi} 
= \frac{\partial F}{\partial \phi} - \partial_\alpha \frac{\partial F}{\partial (\partial_\alpha \phi)}  
\end{equation}

In the
case that $\gamma(\phi) \sim \Delta\cdot\phi$ we generate terms
\begin{equation}
\mu_{\rm c} + \mu_{\rm lc} = A\phi + B\phi^3 -\kappa \partial_{\gamma}\partial_{\gamma}\phi
- \textstyle{\frac{1}{6}} A_0 \Delta Q_{\alpha\beta}^2
- \textstyle{\frac{1}{3}} A_0 \Delta
                          Q_{\alpha\beta} Q_{\beta\pi} Q_{\pi\alpha}
+ \textstyle{\frac{1}{4}} A_0 \Delta (Q_{\alpha\beta}^2)^2
\end{equation}
from $f_{\rm c}(\phi)$ and $f_{\rm lc}(Q_{\alpha\beta})$.
The interfacial or anchoring contribution to the chemical potential is
\begin{equation} 
\mu_{\rm anchoring}=-W \left(
 \partial_{\alpha}\phi\partial_{\beta}Q_{\alpha\beta}
+\partial_{\beta}\phi\partial_{\alpha}Q_{\alpha\beta}\right)
- 2W Q_{\alpha\beta} \partial_\alpha\partial_\beta\phi.
\end{equation}

\subsubsection{Molecular field}

The molecular field resulting from the combined free energy is the functional
derivative of the total free energy with respect to the tensor order parameter $Q_{\alpha\beta}$,

\begin{equation}
H_{\alpha\beta} = H_{\alpha\beta,{\rm lc}} + H_{\alpha\beta, {\rm anchoring}}  = \frac{\delta F}{\delta Q_{\alpha\beta}} 
= \frac{\partial F}{\partial Q_{\alpha\beta}} - \partial_\gamma \frac{\partial F}{\partial (\partial_\gamma Q_{\alpha\beta})}  
\end{equation}

The terms in the molecular field resulting from the bulk free energy
density of the liquid crystalline part are:
\begin{equation}
H_{\alpha\beta,{\rm lc}} = -A_0 (1 - \gamma(\phi)/3) Q_{\alpha\beta}
+A_0 \gamma(\phi)\left[ Q_{\alpha\pi} Q_{\pi\beta}
     - {\textstyle\frac{1}{3}} \delta_{\alpha\beta} Q_{\pi\sigma}^2\right]
-A_0 \gamma (\phi)Q_{\pi\nu}^2 Q_{\alpha\beta} \\ \nonumber
\end{equation}
where the terms related to the elastic contants $\kappa_0$ and $\kappa_1$
have been omitted (cf. Equation~\ref{eq-lc-h-full}). The interfacial
contribution to the molecular field is
\begin{equation}
H_{\alpha\beta, {\rm anchoring}}  = -W \left( \partial_{\alpha}\phi\partial_{\beta}\phi
- {\textstyle\frac{1}{3}}
  \partial_{\pi}\phi \partial_{\pi}\phi \delta_{\alpha\beta} \right).
\end{equation}
As before, relevant contributions to the molecular field have
been rendered traceless.

\subsubsection{Stress and the force on the fluid}
The thermodynamic contribution to the pressure tensor is
\begin{equation}
\label{eq-lc-emulsion-stress}
\Pi_{\alpha\beta} = \sigma_{\alpha\beta} + \tau_{\alpha\beta} + P_{\alpha\beta}
\end{equation}
where $\sigma_{\alpha\beta}$ and $\tau_{\alpha\beta}$ are the symmetric and
antisymmetric contributions
arising from the liquid crystalline order: 
\begin{equation}
\sigma_{\alpha\beta} =
-p_0\delta_{\alpha\beta}
- \xi H_{\alpha\pi}
  \left(Q_{\pi\beta}+{\textstyle \frac{1}{3}} \delta_{\alpha\beta}\right)
- \xi
  \left(Q_{\alpha\pi} + {\textstyle\frac{1}{3}}\delta_{\alpha\pi}\right)
  H_{\pi\beta}
+ 2\xi \left(Q_{\alpha\beta} + {\textstyle \frac{1}{3}}\delta_{\alpha\beta}
\right) Q_{\pi\nu} H_{\pi\nu}
\nonumber
\end{equation}
and
\begin{equation}
\tau_{\alpha\beta} = Q_{\alpha\pi} H_{\pi\beta} - H_{\alpha\pi}Q_{\pi\beta}.
\nonumber
\end{equation}

The term $P_{\alpha\beta}$ includes the binary fluid and coupling
(between LC and binary fluid) contributions and is somewhat more
complicated than in the bare liquid crystal case:
\begin{eqnarray}
\label{eq-lc-emulsion-pab}
P_{\alpha\beta} =
-\left( \phi\frac{\delta  F}{\delta\phi} - F \right) \delta_{\alpha\beta}
-\frac{\delta F}{\delta \left(\partial_{\beta}\phi\right)}\partial_{\alpha}\phi
-\frac{\delta F}{\delta\left(\partial_{\beta}Q_{\gamma\nu}\right)}\partial_{\alpha}Q_{\gamma\nu},
\end{eqnarray}
where we have re-introduced the full free energy functional $F$.
This term is discussed in the following section.

\subsection{Implementation}

It is convenient \cite{sulaiman2006} to rewrite a divergence of the
stress  $-\partial_\beta P_{\alpha\beta}$ with the stress defined in
Equation~\ref{eq-lc-emulsion-pab}. This force can be expanded in full as
\begin{equation}
\partial_{\alpha}\left(\phi\mu - F\right)
+\partial_{\beta} \left(\frac{\delta F}{\delta \left(\partial_{\beta}\phi\right)}\right)\partial_{\alpha}\phi
+\frac{\delta F}{\delta \left(\partial_{\beta}\phi\right)}\partial_{\beta}\left(\partial_{\alpha}\phi\right)
+\partial_{\beta}\left(\frac{\delta F}{\delta Q_{\pi\nu,\beta}}\right)
 \partial_{\alpha} Q_{\pi\nu}
+\frac{\delta F}{\delta Q_{\pi\nu,\beta}} \partial_{\beta}\left(\partial_{\alpha} Q_{\pi\nu}\right).
\label{eq-lc-emulsion-force}
\end{equation}
Applying the chain rule on the first bracket in Equation \ref{eq-lc-emulsion-force} leads to
\begin{equation}
\partial_{\alpha}(\phi\mu) 
- \frac{\partial F}{\partial \phi} \partial_\alpha \phi
- \frac{\partial F}{\partial(\partial_\beta \phi)} \partial_\alpha (\partial_\beta\phi)
- \frac{\partial F}{\partial Q_{\pi\nu}} \partial_\alpha Q_{\pi\nu}  
- \frac{\partial F}{\partial Q_{\pi\nu,\beta}} \partial_\alpha Q_{\pi\nu,\beta}\nonumber.
\end{equation}
The expansion of the functional derivatives in Equation \ref{eq-lc-emulsion-force} gives
\begin{eqnarray}
&&\partial_{\beta} \left(\frac{\partial F}{\partial (\partial_{\beta}\phi)} - \partial_\gamma \frac{\partial F}{\partial (\partial_\gamma\partial_\beta\phi)}\right) \partial_{\alpha}\phi
+\left(\frac{\partial F}{\partial (\partial_{\beta}\phi)} - \partial_\gamma \frac{\partial F}{\partial (\partial_\gamma\partial_\beta\phi)}\right) \partial_\beta\left(\partial_{\alpha}\phi\right)\nonumber\\
&&\partial_{\beta} \left(\frac{\partial F}{\partial Q_{\pi\nu,\beta}} - \partial_\gamma \frac{\partial F}{\partial (\partial_\gamma Q_{\pi\nu,\beta)}}\right) \partial_{\alpha}Q_{\pi\nu}
+ \left(\frac{\partial F}{\partial Q_{\pi\nu,\beta}} - \partial_\gamma \frac{\partial F}{\partial (\partial_\gamma Q_{\pi\nu,\beta)}}\right) \partial_\beta Q_{\pi\nu,\alpha}\nonumber.
\end{eqnarray}
Using the symmetry of partial derivatives and $$\frac{\partial F}{\partial (\partial_\gamma\partial_\beta\phi)} = \frac{\partial F}{\partial (\partial_\gamma Q_{\mu\nu,\beta})}=0$$
this can be recast with some manipulation as 
\begin{eqnarray}
-\partial_\beta P_{\alpha\beta} &=& -\left( \frac{\partial F}{\partial Q_{\pi\nu}} 
       - \partial_{\beta}\frac{\partial F}{\partial Q_{\pi\nu,\beta}}
 \right) \partial_{\alpha}Q_{\pi\nu}
- \left(\frac{\partial F}{\partial\phi} 
- \partial_{\beta} \frac{\partial F}{\partial \left(\partial_{\beta}\phi\right)}\right)\partial_{\alpha}\phi + \partial_{\alpha}\left(\phi\mu\right)\nonumber\\
&=& -\left(\frac{\delta F}{\delta Q_{\pi\nu}}\right) \partial_{\alpha}Q_{\pi\nu} - \left(\frac{\delta F}{\delta \phi}\right) \partial_{\alpha} \phi + \partial_{\alpha}\left(\phi\mu\right)\nonumber\\ 
&=& H_{\pi\nu}\; \partial_{\alpha} Q_{\pi\nu} + \phi \;\partial_{\alpha}\mu,
\label{eq-lc-emulsion-hgradq}
\end{eqnarray}
which is a local force on the fluid.
The total force on the fluid locally is the divergence of the stress,
where the total stress is $\Pi_{\alpha\beta}$
given by equation~(\ref{eq-lc-emulsion-stress}).
It is sometimes useful to compute the various contributions to the
force in different ways. The contribution from the symmetric and
antisymmetric stress $\sigma_{\alpha\beta}$ and  $\tau_{\alpha\beta}$
can be calculated numerically as a divergence. The contribution from
$P_{\alpha\beta}$ can be computed numerically as a divergence, or via the
analytical form of the divergence Equation~\ref{eq-lc-emulsion-hgradq}.
The  last option is
at the expense of global conservation of momentum.


\vfill
\pagebreak
