%%%%%%%%%%%%%%%%%%%%%%%%%%%%%%%%%%%%%%%%%%%%%%%%%%%%%%%%%%%%%%%%%%%%%%%%%%%%%%
%
%  examples.tex
%
%  Some worked examples:
%  1. Conductance in simple porous media
%
%  Edinburgh Soft Matter and Statistical Physics Group and
%  Edinburgh Parallel Computing Centre
%
%  (c) 2010-2017 The University of Edinburgh
%
%  Contributing authors:
%  Kevin Stratford (kevin@epcc.ed.ac.uk)
%
%%%%%%%%%%%%%%%%%%%%%%%%%%%%%%%%%%%%%%%%%%%%%%%%%%%%%%%%%%%%%%%%%%%%%%%%%%%%%%

\section{Examples}
\label{section-examples}

\subsection{Permeability calculations in simple porous media}

Single fluid permeability calculations for a given porous structure
can be undertaken by driving a flow via the fluid body force. Note
that the structure must be periodic in the direction of the force
(this may mean duplicating a 'reflected' version of a given sample
to create the correct input). The body force can be specified so
that, e.g., to drive a flow in the positive $x-$direction
\begin{lstlisting}
force 0.001_0.000_0.000
\end{lstlisting}
The force should not be so large that the maximum velocity generated
threatens the Mach number constraint. To get a measurement of the
flow at equilibrium, the calculation should be run at least the
momentum diffusion time for the system ($L^2/\eta$ in LB time steps).

The net flow can be measured by combining the statistics for the
total momemnum and the total density (which is equal to the volume
with $\rho_0 = 1$). We will see an example of this later in the
section.

Note that in the case of porous media with narrow channels at the grid
scale, the wall velocity can be dependent on the viscosity of the fluid.
This is an artefact of the bounce-back on links and will result in a
viscosity-dependent permeability (see, e.g., \cite{lipanmiller}).
To minimise this effect, e.g.,  Ginzburg and d'Humi\`eres \cite{ginzburg}
corrects the viscosity-dependence of the apparent boundary position
using a three relaxation time scheme.

Before considering a concrete example, it's worth reviewing the basic
calculation of the conductance in ideal geometries.


\subsubsection{Circular and rectangular capillaries}

\label{section-examples-exact-conductance}

For an infinite  capillary of circular cross section radius $a$,
the conductance is known analytically\cite{papanastasiou}. The
flow per unit area $J$ is given by
\begin{equation}
J = - \frac{1}{8\eta} \frac{\partial p}{ \partial x} a^2. 
\end{equation}
If the pressure gradient is replaced by a uniform body force
$-\partial p/ \partial x = \rho g$, then one can define a
viscosity-independent conductance $C$ via $ J = C \rho g / \eta$,
i.e., $C = a^2/8$. So, by measuring the volume flux at steady state
for a fixed applied body force, one can compute an estimate of the
conductance to compare with this result.

An analytical expression is also available for the conductance
of a square or rectangular capillary. For an infinite capillary
of square cross section width $w \times h$ (where $h$ is the
longer),
the volume flux per unit area $J$ is expected to be
\cite{papanastasiou,edo1}
\begin{equation}
J = - \frac{1}{3\eta} \frac{\partial p}{\partial x} (h/2)^2
\left[
1 - 6(h/w) \sum_{k=1}^{\infty} \frac{\tanh(\alpha_k w/h)}{\alpha_k^5} 
\right]
\end{equation}
where $\alpha_k = (2k - 1)\pi/2$. The pressure gradient
$-\partial p / \partial x$ is replaced by the body force $\rho g$ and
one can difine the viscosity-independent conductance $C$ via
$J = C\rho g / \eta$. The calculation proceeds as above.

\subsubsection{Setting up a structure}

We will work out the conductance of a simple one-dimensional capillary
of circular cross section. We can set up the structure using the
utility program found in \texttt{util/capillary.c}.

We set the system size to be $(10,10,32)$ in the $x-$, $y-$, and
$z-$directions, respectively. The cross section in $x-y$ will be
a discrete  approximation to a circle. The nominal radius is
$a = 4$ lattice units (allowing for solid sites at each edge).
In \texttt{capillary.c} we
set
\begin{lstlisting}
const int xmax = 10;
const int ymax = 10;
const int zmax = 32;
\end{lstlisting}
Note that the conductance should be independent of the length of the
capillary in the $z-$direction. We will check this later.

We choose a circular cross section via:
\begin{lstlisting}
enum {CIRCLE, SQUARE, XWALL, YWALL, ZWALL, XWALL_OBSTACLES};
const int xsection = CIRCLE;
\end{lstlisting}
For this problem, the wetting parameters are irrelevant, and can be
ignored. In addition we set
\begin{lstlisting}
enum {STATUS_ONLY, STATUS_WITH_C_H, STATUS_WITH_SIGMA};
const int output_type = STATUS_ONLY;
\end{lstlisting}
to specify that the output will contain only the structural information.
The output filename containing the structure is set via
\begin{lstlisting}
const char * filename = "capillary.001-001";
\end{lstlisting}

The program may be compiled and run via
\begin{lstlisting}
$ make capillary.c -lm
$ ./capillary
\end{lstlisting}
The output provides a simple representation of the cross section,
and a count of the number of solid sites, the number of fluid sites,
and the total. Again, the wetting parameters are irrelevant. For this
case we have
\begin{lstlisting}
Cross section (0 = fluid, 1 = solid)
 1 1 1 1 1 1 1 1 1 1
 1 1 1 0 0 0 0 1 1 1
 1 1 0 0 0 0 0 0 1 1
 1 0 0 0 0 0 0 0 0 1
 1 0 0 0 0 0 0 0 0 1
 1 0 0 0 0 0 0 0 0 1
 1 0 0 0 0 0 0 0 0 1
 1 1 0 0 0 0 0 0 1 1
 1 1 1 0 0 0 0 1 1 1
 1 1 1 1 1 1 1 1 1 1
n = 3200 nsolid = 1536 nfluid = 1664
\end{lstlisting}
Note that the outermost sites in each direction here are solid. There
should be a file \texttt{capillary.001-001} in the current directory.
This should be moved to the executable directory.


\subsubsection{Setting up the input}

Assuming we have compiled the serial code, we must now set up the
input file to be consistent with the capillary structure we want
to use. The \texttt{src/input.ref} file can be used as a template.
We should set the system size
\begin{lstlisting}
size 10_10_32
\end{lstlisting}
The location of the porous media file is specified using
\begin{lstlisting}
porous_media_format BINARY
porous_media_file   capillary
porous_media_type   status_only
\end{lstlisting}
Note that the extension \texttt{.001-001} of the filename has been
removed in the input file (it gets added back automatically by the
main code at run time).

As this is a single fluid calculation with no free energy involved
we must set
\begin{lstlisting}
free_energy none
\end{lstlisting}
What are the fluid parameters? We will set the viscosity to
$1/6$ in lattice units (and the bulk viscosity to the default
value by commenting it out; the default value is the same as
the shear viscosity):
\begin{lstlisting}
viscosity 0.16666666666666666
#bulk_viscosity 0.1
\end{lstlisting}
We expect that the number of time steps to reach a steady state
will be of order $t \approx a^2/\eta$, which we estimate using
$a=4$ so $t \approx 100$ LB time steps. We will try
\begin{lstlisting}
N_cycles 200
\end{lstlisting}
Note that we will need output on the total momentum and the volume
flux of the system
as a function of time, so we set
\begin{lstlisting}
freq_statistics 50
\end{lstlisting}

Finally, we need to set a force in the $z$-direction to drive the
flow. Again, the final conductance should be independent of this
force providing the force is small enough that both the Mach number
and the Reynolds number are small compared to unity in steady stead.
We will set
\begin{lstlisting}
force 0.00_0.00_0.0000001
\end{lstlisting}

\subsubsection{Extracting the conductance}

With these parameters specified, we can run the code. The output
should reflect the parameters in the input file, and the time step
loop should start if the parameters, and the \texttt{capillary.001-001}
file has been read successfully. The code reports various fluid
properties. The density is \texttt{[rho]}, and we see that the
total is the same as the number of fluid sites reported by the
\texttt{capillary.c} program:
\begin{lstlisting}
Scalars - total mean variance min max
[rho]    1664.00  1.00000000000-2.2204460e-16  1.00000000000 1.00000000000
\end{lstlisting}
Note that the total density (ie., mass) should remain exactly unchanged
for the duration of the run; if not, there is something wrong.

The integrated volume flux of the fluid in each coordinate direction is
reported, and we should see, as a function of time
\begin{lstlisting}
Velocity - x y z
...
[vol flux]  3.1086245e-15  2.4868996e-14  1.9303426e-03
[vol flux] -6.8278716e-14  4.8849813e-15  2.0251488e-03
[vol flux]  8.4265928e-14  5.5178084e-14  2.0299067e-03
[vol flux]  1.6209256e-14  1.2212453e-14  2.0301454e-03
\end{lstlisting}
Note as we have applied an external force in the $z-$direction,
the $z$-flux increases with time while the volume flux in the
other two directions is constant (zero to machine accuracy, which is
around $10^{-16}$). Further, the $z$-volume flux
is still changing slightly after 200 time steps, so we have
not run long enough to reach a steady state. If we run for
400 steps, we should see that the $z$-flux is unchanged
over the last 100 time steps.
\begin{lstlisting}
Velocity - x y z
...
[vol flux]  6.6391337e-14 -2.2204460e-15  2.0301580e-03
[vol flux]  3.4638958e-14  3.6415315e-14  2.0301580e-03
\end{lstlisting}
Finally, note the maximum velocity in the flow direction
is small compared with unity, and so both Mach number and
Reynolds number are also small in this case.


The figure we are interested in is the volume flux in the
$z$-direction which is $2.030\times 10^{-3}$. As the mean
density is unity, this is also the integrated volume flux $J_v$.
The volume flux per unit area $J = J_v / a^2 L_z$, where $a^2 L_z$
is the volume of the discrete system (1664 in lattice units).
Following section \ref{section-examples-exact-conductance},
we can write a viscosity independent conductance
$C = J\eta / \rho g$, with $g$ the force in the $z-$direction.
Putting all this together, we have $C = (2.030\times 10^{-3} / 1664)
\times (1/6) / (1.0 \times 1\times 10^{-7}) = 2.033$. The theoretical
figure is $C = 2$, so the simulation is correct to within a discretisation
error of about 2\%.

\subsubsection{In parallel}

We can run the same calculation using the parallel code (recompile
with \texttt{make mpi}). In the input file we set
\begin{lstlisting}
size 10_10_32
grid 1_1_2
\end{lstlisting}
to set the parallel decomposition explicitly to two MPI tasks in the
$z$-direction. We also set
\begin{lstlisting}
porous_media_format BINARY_SERIAL
porous_media_file   capillary
porous_media_type   status_only
\end{lstlisting}
to ensure the structure file is read correctly in parallel.
We should run this on 2 MPI tasks, e.g.,
\begin{lstlisting}
$ mpirun -np 2 ./Ludwig.exe input
\end{lstlisting}
The final result should be exactly the same as the serial version
to machine accuracy. You may also be able to spot that the execution
time is approximately half that for the serial version (although this
problem is a little small for efficient parallelisation).

\subsubsection{General porous media}

For a general porous media, we can make use of Darcy's Law which
defines a permeability $k$ (with dimensions of area) via
\begin{equation}
J_v = - \frac{k A}{\eta} \frac{\partial P}{\partial z}
\end{equation}
where $J_v$ is the volume flux, $A$ is the cross-sectional area
of the sample, and $\partial P / \partial z$ is the pressure gradient
as before. 


\subsubsection{Matching lattice and real units in porous media}


Following Succi \cite{succi} (Chapter~8) the following argument
can be made to match lattice quantities and real physical
quantities for a given system of interest.
Dimensionless lattice Boltzmann units set the lattice spacing
$\Delta x = 1$, the lattice time step $\Delta t = 1$ and the
mean density of the fluid to $\rho_0 = 1$. The speed of sound
in lattice units is $c_s = 1/ \sqrt{3}$. How do we match these
units to physical ones?

Suppose we have discretised an X-ray tomography image with a resolution
of 1 voxel equal to 1 $\mu$m. If we represent one voxel by one cubic
lattice with width $\Delta x$, then we can match $\Delta x = 1\mu$m. A
large data set might provide 1000 voxels on a side, giving
1000 lattice sites, which would represent 1 mm.
Similarly, if our real system is water at room temperature with density
$\rho = 1000$kg~m$^{-3}$, then we may equate the lattice density
$\rho_0 = 1$ to be equivalent to 1000~kg~m$^{-3}$. (This means the
mass of fluid at one lattice site of volume $\Delta x^3 = 1 \mu$m$^{3}$
is then 10$^{-15}$~kg, although this is not a very useful number.)

This leaves time. This is matched via the speed of sound. For the real
system, we take the speed of sound in water at 20$^o$C to be
1480 m~s$^{-1}$. So the speed of sound on the lattice
$(1/\sqrt{3}) \Delta x$ per $\Delta t$ matches 1480~m~s$^{-1}$,
or $\Delta t = 1/\sqrt{3} \times 10^{-6} / 1480 \approx 3.9 \times 10^{-10}$s.
Note that this is a small unit, i.e., it would require very many time
steps to reach a `macroscopic' time in any simulation.

We should also consider the Reynolds number $Re = \rho U L / \eta$.
Suppose our real structure has a characteristic pore size which is
just at the resolution of the X-ray tomography image: $L = h = 1\mu$m.
If the fluid (water) has viscosity $\eta = 10^{-3}$ Pa s, and
a typical flow is $U \sim 10^{-3}$ m~s$^{-1}$, then the pore scale
Reynolds number is $Re_h \sim 10^{-3}$. In the LB, if we choose a
lattice viscosity $\eta = 1/6$, and generate a typical flow in
lattice units of $10^{-4}$, then the Reynolds number in the
simulation (with $\rho_0 = 1$ and $L = \Delta x = 1$) is
$Re \sim 6 \times 10^{-4}$, which is similar.


There is one potential problem if we wish to study transport
phenomena where we might want to run a simulation long enough
for the flow to cross the whole sample. With $U \sim 10^{-4}$
in lattice units and a large sample size of, say,  1024$^3$ the
flow would take around 10$^7$ simulation time steps to propagate
information across the system. This is computationally expensive.
One solution is
to artificially raise the flow speed (and hence the Reynolds
number). This is acceptable as long as the Reynolds number
remains small compared with unity (all Reynolds numbers $< 0.1$
can be regarded as negligible \cite{batchelor}.) Here, for example,
if we raise the flow by two orders of magnitude to $10^{-2}$ in
lattice units,
the Reynolds number is still acceptable at $Re = 6\times 10^{-2}$,
and the number of simulation time steps is reduced to a more
managable $10^5$ (see also \cite{cates_scaling}).


% End examples
\vfill
\pagebreak
