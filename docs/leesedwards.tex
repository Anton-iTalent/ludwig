%%%%%%%%%%%%%%%%%%%%%%%%%%%%%%%%%%%%%%%%%%%%%%%%%%%%%%%%%%%%%%%%%%%%%%%%%%%%%%%
%
%  leesedwards.tex
%
%  Lees Edwards Sliding Perioidic Boundary Conditions
%
%  Edinburgh Soft Matter and Statistical Physics Group and
%  Edinburgh Parallel Computing Centre
%
%  (c) 2016 The University of Edinburgh
%
%  Contributing authors:
%  Kevin Stratford (kevin@epcc.ed.ac.uk)
%
%%%%%%%%%%%%%%%%%%%%%%%%%%%%%%%%%%%%%%%%%%%%%%%%%%%%%%%%%%%%%%%%%%%%%%%%%%%%%%%


\section{Lees Edwards Sliding Periodic Boundary Conditions}

\subsection{Background}

The idea of introducing a Galilean transformation in a periodic
computation to model uniform shear was introduced by Lees and
Edwards in 1972 \cite{lees-edwards1972}. It was first adapted
to the lattice Boltzmann picture by Wagner and Pagonabarraga
in 2000 \cite{wagner-pagonabarraga2002}. The current
implementation follows the description of Adhikari
et~al.\ \cite{adhikari-desplat2005}.

\subsection{Distributions crossing the LE planes}

With the planes conceptually half-way between lattice sites, the
propagation moves some of the distributions (namely, those with
$c_x = \pm1$) between different sliding blocks. While the collision
stage is unaffected, action must be taken to adjust these
distributions when the LE boundary conditions are active. This
is done in a two-stage process of reprojection and interpolation
implemented between the collision and propagation.

\subsubsection{Reprojection}
The post-collision distributions with
$c_x = \pm 1$ at sites adjacent to a boundary are modified to
take account of the velocity jump $\pm u^{LE}_y$ between the sliding
blocks. In terms of the hydrodynamic moments we have:
\begin{eqnarray}
\rho &\rightarrow& \rho', \\
\rho u_\alpha &\rightarrow& (\rho u_\alpha)' \pm \rho' u^{LE}_\alpha, \\
S_{\alpha\beta} &\rightarrow&
S'_{\alpha\beta} \pm (\rho u_\alpha)' u^{LE}_\beta
\pm (\rho u_\beta)' u^{LE}_\alpha + \rho' u^{LE}_\alpha u^{LE}_\beta.
\end{eqnarray}
If we work with the changes to the moments, then the density is
unaffected $\delta\rho =  0$, the velocity is changed by
$\delta u_\alpha = \pm u^{LE}_\alpha$, with an analogous expression for
the change in the stress
\begin{equation}
\delta S_{\alpha\beta} =
\pm (\rho u_\alpha)' u^{LE}_\beta
\pm (\rho u_\beta)' u^{LE}_\alpha + \rho' u^{LE}_\alpha u^{LE}_\beta.
\end{equation}
We can then work out the change in the distributions by reprojecting
via Eq.\ref{eq:}, i.e.,
\begin{equation}
f_i \rightarrow f_i' + w_i \Bigg(
\frac{\rho \delta u_\alpha c_{i\alpha}}{c_s^2} +
\frac{\delta S_{\alpha\beta}Q_{i\alpha\beta}}{2c_s^4} \Bigg).
\end{equation}
A similar set of expressions are given for the order parameter
distributions in Adhikari \textit{et al.} \cite{adhikari-desplat2005}.

\subsubsection{Interpolation of reprojected distributions}
As the sliding blocks are displaced continuously with
$\delta y = u_y^{LE}t$, which is not in general a whole number of
lattice spacings, an interpolation is also required before
propagation can take place. Consider the situation  from
a stationary frame where the frame above has translated a small
distance to the right ($\delta y < \Delta y$). In the stationary frame
we must interpolate the distributions with $c_x = 1$ to a
`departure point' from which the propagation will move the
distribution exactly to the appropriate lattice site in the moving
frame. Schematically,
\begin{equation}
f_i'(x, y + \delta y, z) = 
 (1 - \delta y) f_i^\star(x, y, z) +
\delta y f_i^\star(x, y + \Delta y, z),
\end{equation}
where $f_i'$ here represents the interpolated value and $f_i^\star$
is the reprojected post-collision quantity.

In the case where the displacement $\delta y > 1$, the relevant
lattice sites involved in the interpolation are displaced to the right
by the integral part of $\delta y$, while the relative weight given to
the two sites involves
the fractional part of $\delta y$. To take account of the periodic
boundary conditions
in the $y$-direction, all displacements are modulo $L_y$.


\subsection{Order parameter gradients}

When the order parameter gradient is computed at a given point
near the LE boundaries, the stencil may extend out of the stationary
frame. The process of interpolation is slightly different to that
for the distributions.

Here, we need to interpolate values in the moving frame to the position
which is seen by the rest frame. If a five-point
stencil is used to calculate the gradient at the central point
$(x,y,z)$ in the
rest frame. An interpolation of the $\phi$-field of the form
\begin{equation}
\phi'(x + \Delta x, y - \delta y, z) =
\delta y \phi(x + \Delta x,  y - \Delta y, z) + 
(1 - \delta y) \phi(x + \Delta x, y, z)
\end{equation}
must then take place. In practice, the interpolated values are stored
in the offset buffer. As the gradient calculation requires $\phi$
values in the halo regions, interpolated values for the buffer halo
region are also required.

In parallel, the interpolation requires data from at most two adjacent
processes in the along-plane $y-$direction as long as the $\phi$ halo
regions are up-to-date. The rank of the processes involved is
determined by the displacement $\delta y$ as a function of time.


\subsection{Velocities and finite-difference fluxes}

For the finite-difference implementation, the velocity field at
the cell boundaries is required. At the planes, this means an
interpolation which takes the same form as that for
$\phi$ is needed.

In computing the advective fluxes between cells, we must then
allow for the presence of the planes. This is done by computing
separately the 'east' and 'west' face fluxes in the $x$-direction
(the flux that crosses the plane). Away from the planes, face-flux
uniqueness $f_{ijk}^e = f_{i+1jk}^w$ ensures conservation of order
parameter. However, at the planes, the non-linear combination of
velocity field and order parameter field does not in general give
rise to consistent fluxes. In order to restore conservation, we
reconcile east and west face fluxes at the plane by taking an
average
\begin{eqnarray}
f_{ijk}^{e\star}&=&{\textstyle\frac{1}{2}}(f_{ijk}^e + f_{i+1j'k}^w),\quad 
f_{i+1j'k}^w  = \delta y f_{i+1 j-1 k}^w + (1 - \delta y) f_{i+1jk}^w,\\
f_{i+1 j k}^{w\star} &=& {\textstyle\frac{1}{2}}(f_{ij'k}^e + f_{i+1jk}^w),\quad
f_{ij'k}^e = (1 - \delta y) f_{ijk}^e + \delta y f_{ij+1k}^e.
\end{eqnarray}
In this way, we average the east face flux in the rest frame $f_{ijk}^e$
with the interpolated value of the west face flux in the moving frame
$f_{i+1j'k}^w$, and vice-versa. One can then easily show that when
integrated over the length of the plane, the average fluxes are
consistent, i.e.,
\begin{equation}
\sum_j (f_{ijk}^{e\star} - f_{i+1jk}^{w\star}) = 0 \quad\forall k,
\end{equation}
thus ensuring order parameter conservation.

A similar correction is required when computing the force on the fluid
directly via the divergence of the thermodynamic stress
$F_\alpha = \nabla_\beta P_{\alpha\beta}^{th}$. This is implemented
by computing fluxes of momentum at cell faces in each direction, and
then taking the divergence to compute the force. At the planes, an
averaging procedure is again used to ensure conservation of momentum.



% End section
\vfill
\pagebreak
