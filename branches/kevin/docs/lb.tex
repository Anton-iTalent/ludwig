
\section{Lattice Boltzmann Hydrodynamics}
\label{section:lb-hydrodynamics}

We review here the lattice Boltzmann method applied to a simple
Newtonian fluid with particular emphsis on the relevant
implementation in \textit{Ludwig}.

\subsection{The Navier Stokes Equation}

We seek to solve the isothermal Navier-Stokes equations which, often
written in vector form, express mass conservation
\begin{equation}
\partial_t \rho + \boldsymbol{\nabla}.(\rho\mathbf{u}) = 0
\label{eq_mass1}
\end{equation}
and the conservation of momentum
\begin{equation}
\partial_t (\rho\mathbf{u}) + \boldsymbol{\nabla}.(\mathbf{\rho uu}) =
-\boldsymbol{\nabla}p + \eta \nabla^2 \mathbf{u}
+\zeta \boldsymbol{\nabla}(\boldsymbol{\nabla}.\mathbf{u}).
\label{eq_momentum1}
\end{equation}
Equation~\ref{eq_mass1} expresses the local rate of change of the
density $\rho(\mathbf{r}; t)$ as the divergence of the flux of
mass associated with the velocity field $\mathbf{u}(\mathbf{r}; t)$.
Equation~\ref{eq_momentum1} expresses Newton's second law for
momentum, where the terms on the right hand side represent the
force on the fluid.

For this work, it is more convenient to rewrite these equations
in tensor notation, where Cartesian coordinates ${x,y,z}$ are
represented by indices $\alpha$ and $\beta$, viz
\begin{equation}
\partial_t \rho + \nabla_\alpha (\rho u_\alpha) = 0
\end{equation}
and
\begin{equation}
\partial_t (\rho u_\alpha) + \nabla_\beta (\rho u_\alpha u_\beta)
= -\nabla_\alpha p
+  \eta \nabla_\beta (u_\alpha \nabla_\beta + \nabla_\alpha u_\beta)
+ \zeta \nabla_\alpha (\nabla_\gamma u_\gamma).
\end{equation}
Here, repeated indices are understood to be summed over.
The conservation law is seem better if the forcing terms of the
right hand side are combined in the fluid stress $\Pi_{\alpha\beta}$
so that
\begin{equation}
\partial_t (\rho u_\alpha) +\nabla_\beta \Pi_{\alpha\beta} = 0.
\end{equation}
In this case the expanded expression for the stress tensor is
\begin{equation}
\Pi_{\alpha\beta} = p \delta_{\alpha\beta} + \rho u_\alpha u_\beta 
+ \eta \nabla_\alpha v_\beta + \zeta (\nabla_\gamma v_\gamma)\delta_{\alpha\beta}
\end{equation}
where $\delta_{\alpha\beta}$ is that of Kroneker. Th Navier-Stokes
equations have 10 degrees of freedom (or hydrodynamic modes) being
$\rho$, three components of the mass flux $\rho u_\alpha$, and 6 independent
modes from the (symmetric) stress tensor $\Pi_{\alpha\beta}$.


\subsection{The Distribution Function and its Moments}

In lattice Boltzmann, the density and velocity of the continuum fluid
are replaced by their discrete counterparrts, the distribution function
$f_i(\mathbf{r}; t)$ and the discrete velocity space $c_{i\alpha}$.
It is possible to relate the hydrodynamic modes to distribution
function via its moments, so
\begin{equation}
\rho = \sum_i f_i,  \,
\rho u_\alpha = \sum_i f_i c_{i\alpha},  \,
\Pi_{\alpha\beta} = \sum_i f_i c_{i\alpha} c_{i\beta.}
\end{equation}
here, the index of the summation is over the number of discrete
velocities used $n$, which in three dimensions is generally
denoted by D3Q$n$. We restrict our discussion to D3Q15 and
D3Q19 (see below).

In general, the number of modes supported in $n$ and these can
be written as
\begin{equation}
M^a(\mathbf{r};t) = \sum_i m_i^a f_i(\mathbf{r};t),
\end{equation}
where $m_i^a$ is the corresponding eigenvector of the collision
maxtrix (see section blah). For example, if $m_i^a = 1$ then the
mode $M^a$ is the density. Correspondingly, the distribution
function can be related to the modes $M^a$ via
\begin{equation}
f_i() = \sum_a w_i m_i^a M^a() N^a
\end{equation}

\subsubsection{Hydrodynamic modes}

\subsubsection{D3Q15 ghost modes}

\subsubsection{D3Q19 ghost modes}

\subsection{The Lattice Boltzmann Equation}

\subsection{The Collision}

\subsection{External Body Forces}

\subsection{Fluctuating LBE}

It is possible \cite{adhikari2005} to simulate fluctuating
hydrodynamics for an isothermal fluid via the inclusion of
a fluctuating stress $\sigma_{\alpha\beta}$:
\begin{equation}
\Pi_{\alpha\beta} = p\delta_{\alpha\beta} + \rho u_\alpha u_\beta
+ \eta_{\alpha\beta\gamma\delta} \nabla_\gamma u_\delta + \sigma_{\alpha\beta}.
\end{equation}
The fluctuation-dissipation theorem relates the magnitude of this
random stress to the isothermal temperature and the viscosity.

In the LBE, this translates to the addition of a random contribution
$\xi_i$ to the distribution at the collision stage, so that
\begin{equation}
\ldots + \xi_i.
\end{equation}

For the conserved modes $\xi_i = 0$. For all the non-conserved modes,
i.e., those with dissipation, the fluctuating part may be written
\begin{equation}
\xi_i (\mathbf{r}; t) = w_i m_i^a \hat{\xi}^a (\mathbf{r}; t) N^a
\end{equation}
where $\hat{\xi}^a$ is a noise termwhich has a variance determined
by the relaxation time for given mode
\begin{equation}
\left< \hat{\xi}^a \hat{\xi}^b \right> =
\frac{\tau_a + \tau_b + 1}{\tau_a \tau_b}
\left< \delta M^a \delta M^b \right>.
\label{eq_fvar}
\end{equation}

\subsubsection{Fluctuating stress}

For the stress, the random contribution to the distributions is
\begin{equation}
\xi_i = w_i \frac{Q_{i\alpha\beta} \hat{\sigma}_{\alpha\beta}}{4c_s^2}
\end{equation}
where $\hat{\sigma}_{\alpha\beta}$ is a symmtric matrix of random
variates drawn from a Gaussian distribution with variance given
by equation~\ref{eq_fvar}. In the case that the shear and bulk
viscosities are the same, i.e., there is a single relaxation
time, then the variances of the six independent components of
the matrix are given by
\begin{equation}
\left< \hat{\sigma}_{\alpha\beta} \hat{\sigma}_{\mu\nu} \right> =
\frac{2\tau + 1}{\tau^2}
(\delta_{\alpha\mu}\delta_{\beta\nu} + \delta_{\alpha\nu} \delta_{\beta\mu}).
\end{equation}


\subsubsection{Fluctuating ghost modes: D3Q15}

\subsubsection{Fluctuating ghost modes: D3Q19}



\subsection{The Propagation}



\subsection{Solid bodies and Bounce-Back on Links}

A very general method for the representation of solid objects
within the LB approach was put forward by Ladd \cite{l94a, l94b}.
Solid objects (of any shape) are defined by a boundary surface
which intersects some of the velocity vectors $\mathbf{c}_i$
joining lattice nodes. Sites inside are designated solid, while
sites outside remain fluid. The correct boundary condition is
defined by identifying \textit{links} between fluid and solid
sites, which allows those elements of the distribution which would
cross the boundary at the propagation step to be ``bounced-back''
into the fluid. This bounce-back on links is an efficient method
to obtain the  correct hydrodynamic interaction between solid
and fluid.

\subsubsection{Boundary walls}

\subsection{Model Details}

\subsubsection{D2Q9}

These are the eigenvalues and eigenvectors of the collision
matrix used for D2Q9


\begin{table}
\begin{tabular}{|l|r|rrrrrrrrr|r|l|}
\hline\hline
$M^a$ & & \multicolumn{9}{c|}{$m_i^a$} & $N^a$  &\\
\hline
$\rho$ & - & 1 &  1 &  1 &  1 &  1 &  1 &  1 &   1 &  1 & 1 &$\mathbf{1}$ \\
\hline
$\rho c_{ix}$ & - & 0 &  1 &  1 & 1 & 0 &  0 & -1 &  1 & -1 & 3 & $c_{ix}$ \\
\hline
$\rho c_{iy}$ & - & 0 & 1 &  0 &  -1 &  1 &  -1 & 1 & 0 & -1 & 3  &$c_{iy}$ \\
\hline
$Q_{xx}$ & 1/3 & -1 &  2 &  2 & 2 & -1 & -1 & 2 & 2 & 2 & 9/2 
& $c_{ix} c_{ix} - c_s^2$ \\
\hline
$Q_{xy}$ & - & 0 &  1 & 0 & -1 & 0 & 0 & -1 & 0 & 1 & 9 & $c_{ix} c_{iy}$ \\
\hline
$Q_{yy}$ & 1/3 & -1 &  2 & -1 & 2 & 2 & 2 & 2 & -1 & 2 & 9/2
& $c_{iy} c_{iy} - c_s^2$ \\
\hline\hline
$\chi^1$ & - &  1 & 4 & -2 & 4 & -2 & -2 & 4 & -2 & 4 & 1/4 & $\chi^1$ \\
\hline
$J_{ix}$ & - & 0 &  4 & 0 & -4 & -2 & 2 & 4 & 0 & -4 & 3/8
& $\chi^1 \rho c_{ix}$\\
\hline
$J_{iy}$ & - & 0 & 4 & 0 & -4 & -2 & 2 & 4 & 0 & -4 & 3/8
& $\chi^1 \rho c_{iy}$\\
\hline\hline
$w_i$ & 1/36 & 16 & 1 & 4 & 1 & 4 & 4 & 1 & 4 & 1 & & $w_i$\\
\hline\hline
\end{tabular}
\end{table}

\subsubsection{D3Q15}

These are the eigenvalues and eigenvectors of the collision
matrix used for D3Q15


\begin{table}
\begin{tabular}{|l|r|r|rrrrrr|rrrrrrrr|r|l|}
\hline\hline
$M^a$ & & \multicolumn{15}{c|}{$m_i^a$} & $N^a$  &\\
\hline
$\rho$ & - &
 1 &  1 &  1 &  1 &  1 &  1 &  1 &  1 &  1 &  1 &  1 &  1 &  1 &  1 &  1 &
1 &$\mathbf{1}$ \\
\hline
$\rho c_{ix}$ & - &
 0 &  1 & -1 &  0 &  0 &  0 &  0 &  1 & -1 &  1 & -1 &  1 & -1 &  1 & -1 &
3  & $c_{ix}$ \\
\hline
$\rho c_{iy}$ & - &
 0 &  0 &  0 &  1 & -1 &  0 &  0 &  1 &  1 & -1 & -1 &  1 &  1 & -1 & -1 &
3  &$c_{iy}$ \\
\hline
$\rho c_{iz}$ & - &
 0 &  0 &  0 &  0 &  0 &  1 & -1 &  1 &  1 &  1 &  1 & -1 & -1 & -1 & -1 &
3  & $c_{iz}$ \\
\hline
$Q_{xx}$ & 1/3 &
-1 &  2 &  2 & -1 & -1 & -1 & -1 &  2 &  2 &  2 &  2 &  2 &  2 &  2 &  2 &
9/2  & $c_{ix} c_{ix} - c_s^2$ \\
\hline
$Q_{yy}$ & 1/3 &
-1 & -1 & -1 &  2 &  2 & -1 & -1 &  2 &  2 &  2 &  2 &  2 &  2 &  2 &  2 &
 9/2 & $c_{iy} c_{iy} - c_s^2$ \\
\hline
$Q_{zz}$ & 1/3 &
-1 & -1 & -1 & -1 & -1 &  2 &  2 &  2 &  2 &  2 &  2 &  2 &  2 &  2 &  2 &
 9/2 & $c_{iz} c_{iz} - c_s^2$ \\
\hline
$Q_{xy}$ & - &
 0 &  0 &  0 &  0 &  0 &  0 &  0 &  1 & -1 & -1 &  1 &  1 & -1 & -1 &  1 &
9  & $c_{ix} c_{iy}$ \\
\hline
$Q_{yz}$ & - &
 0 &  0 &  0 &  0 &  0 &  0 &  0 &  1 &  1 & -1 & -1 & -1 & -1 &  1 &  1 &
9  & $c_{iy} c_{iz}$ \\
\hline
$Q_{zx}$ & - &
 0 &  0 &  0 &  0 &  0 &  0 &  0 &  1 & -1 &  1 & -1 & -1 &  1 & -1 &  1 &
9  & $c_{iz} c_{ix}$ \\
\hline\hline
$\chi^1$ & - &
-2 &  1 &  1 &  1 &  1 &  1 &  1 & -2 & -2 & -2 & -2 & -2 & -2 & -2 & -2 &
1/2 & $\chi^1$ \\
\hline
$J_{ix}$ & - &
 0 &  1 & -1 &  0 &  0 &  0 &  0 & -2 &  2 & -2 &  2 & -2 &  2 & -2 &  2 &
3/2 & $\chi^1 \rho c_{ix}$\\
\hline
$J_{iy}$ & - &
 0 &   0 &  0 &  1 & -1 &  0 &  0 & -2 & -2 &  2 &  2 & -2 & -2 &  2 &  2 &
3/2 & $\chi^1 \rho c_{iy}$\\
\hline
$J_{iz}$ & - &
 0 &   0 &  0 &  0 &  0 &  1 & -1 & -2 & -2 & -2 & -2 &  2 &  2 &  2 &  2 &
3/2 & $\chi^1 \rho c_{iz}$\\
\hline
$\chi^3$ & - &
 0 &   0 &  0 &  0 &  0 &  0 &  0 &  1 & -1 & -1 &  1 & -1 &  1 &  1 & -1 &
9 & $c_{ix} c_{iy} c_{iz}$ \\
\hline\hline
$w_i$ & 1/72 &
$16$ & 8 & 8 & 8 & 8 & 8 & 8 & 1 & 1 & 1 & 1 & 1 & 1 & 1 & 1 &
 & $w_i$\\
\hline\hline
\end{tabular}
\end{table}


\subsubsection{D3Q19}

\begin{table}
\begin{tabular}{|l||r|rrrrrr|rrrr|rrrr|rrrr|r||}
\hline\hline
$M^a$ & \multicolumn{19}{c||}{$m_i^a$} & $N^a$\\
\hline
$\rho $ & 1 &  1 &  1 &  1 &  1 &  1 &  1 & 
         1 &  1 &  1 &   1 &  1 &  1 &  1 & 1 & 1 & 1 & 1 & 1
& 1\\
\hline
$\rho c_{ix}$ & 0 &  1 &  -1 &  0 &  0 &  0 &  0 & 
         1 &  1 &  -1 &   -1 &  1 &  1 &  -1 & -1 & 0 & 0 & 0 & 0
& 3 \\
\hline
$\rho c_{iy}$ & 0 &  0 &  0 &  1 &  -1 &  0 &  0 & 
         1 &  -1 &  1 &   -1 &  0 &  0 &  0 & 0 & 1 & 1 & -1 & -1
& 3\\
\hline
$\rho c_{iz}$ & 0 &  0 &  0 &  0 &  0 &  1 &  -1 & 
         0 &  0 &  0 &   0 &  1 &  -1 &  1 & -1 & 1 & -1 & 1 & -1
& 3\\
\hline
$Q_{ixx}$ & -1 &  2 &  2 &  -1&  -1 &  -1 &  -1 & 
         2 &  2 &  2 &   2 &  2 &  2 &  2 & 2 & -1 & -1 & -1 & -1
& 9/2\\
\hline
$Q_{iyy}$ & -1 &  -1 &  -1 &  2&  2 &  -1 &  -1 & 
         2 &  2 &  2 &   2 &  -1 &  -1 &  -1 & -1 & 2 & 2 & 2 & 2
& 9/2\\
\hline
$Q_{izz}$ & -1 &  -1 &  -1 &  -1&  -1 &  2 &  2 & 
         -1 &  -1 &  -1 &   -1 &  2 &  2 & 2 & 2 & 2 & 2 & 2 & 2
& 9/2\\
\hline
$Q_{ixy}$ & 0 &  0 &  0 &  0&  0 &  0 &  0 & 
          1 &  -1 &  -1 &    1 &  0 &  0 & 0 & 0 & 0 & 0 & 0 & 0
& 9\\
\hline
$Q_{ixz}$ & 0 &  0 &  0 &  0&  0 &  0 &  0 & 
          0 &   0 &   0 &   0 &  1 & -1 & -1 & 1 & 0 & 0 & 0 & 0
& 9\\
\hline
$Q_{iyz}$ & 0 &  0 &  0 &  0&  0 &  0 &  0 & 
          0 &   0 &   0 &   0 &  0 & 0 & 0 & 0 & 1 & -1 & -1 & 1
& 9\\
\hline\hline
$\chi^1$ & 0 &  1 &  1 &  1 &  1 &  -2 &  -2 & 
         -2 &  -2 &  -2 &  -2 &  1 &  1 & 1 & 1 & 1 & 1 & 1 & 1
& 3/4\\
\hline
$\chi^1 \rho c_{ix}$ & 0 &  1 &  -1 &  0&  0 &  0 &  0 & 
         -2 &  -2 &  2 &  2 &  1 &  1 & -1 & -1 & 0 & 0 & 0 & 0
& 3/2\\
\hline
$\chi^1 \rho c_{iy}$ & 0 &  0 &  0 &  1&  -1 &  0 &  0 & 
         -2 &  2 &  -2 &  2 &  0 &  0 & 0 & 0 & 1 & 1 & -1 & -1
& 3/2\\
\hline
$\chi^1 \rho c_{iz}$ & 0 &  0 &  0 &  0&  0 &  -2 &  2 & 
         0 &  0 &  0 &  0 &  1 &  -1 & 1 & -1 & 1 & -1 & 1 & -1
& 3/2\\
\hline
$\chi^2$ & 0 &  1 &  1 &  -1&  -1 &  0 &  0 & 
         0 &  0 &  0 &  0 &  -1 &  -1 & -1 & -1 & 1 & 1 & 1 & 1
& 9/4\\
\hline
$\chi^2 \rho c_{ix}$ & 0 &  1 &  -1 &  0&  0 &  0 &  0 & 
         0 &  0 &  0 &  0 &  -1 &  -1 & 1 & 1 & 0 & 0 & 0 & 0
& 9/2\\
\hline
$\chi^2 \rho c_{iy}$ & 0 &  0 &  0 & -1&   1 &  0 &  0 & 
         0 &  0 &  0 &  0 &   0 &  0 & 0 & 0 & 1 &  1 & -1 & -1
& 9/2\\
\hline
$\chi^2 \rho c_{iz}$ & 0 &  0 &  0 &  0&  0 &  0 &  0 & 
         0 &  0 &  0 &  0 &  -1 &  1 & -1 & 1 & 1 & -1 & 1 & -1
& 9/2\\
\hline
$\chi^3$ & 1 &  -2 &  -2 &  -2&  -2 &  -2 &  -2 & 
         1 &  1 &  1 &  1 &  1 &  1 & 1 & 1 & 1 & 1 & 1 & 1
& 1/2\\
\hline\hline
$w_i$ & 12 & 2 & 2 & 2 & 2 & 2 & 2 & 
1 & 1 & 1 & 1 & 1 & 1 & 1 & 1 & 1 & 1 & 1 & 1
& \\
\hline\hline
\end{tabular}
\end{table}

\pagebreak
