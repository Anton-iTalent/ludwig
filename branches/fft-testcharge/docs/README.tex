\documentclass[12pt,a4paper]{article}

\usepackage{verbatim}

\begin{document}

%AC%\pagestyle{myheadings}
%AC%\markright{D.~S.~Henty}

%\title{A Latex thesis example}
%\author{D.~S.~Henty}
%\date{\today}

\setcounter{secnumdepth}{3}

\pagenumbering{roman}

%\begin{center}

\title{Ludwig electrokinetics module - FFT solver with P3DFFT:\\ Instructions}
\author{Ruairi Short}
\date{\today}

\maketitle

\tableofcontents
\clearpage

%\end{center}


\section{Installing libraries}
Two libraries need to be installed in order to use the FFT solver in Ludwig.
One of ESSL or FFTW must be installed and P3DFFT built on the appropriate one of these.

\subsection{Install FFTW (optional)}
If you choose to use FFTW and it is not already installed on your system the following is how to install it.
  \begin{itemize}
  \item extract the fftw-3.3.3.tar file of the FFTw library to a directory of your choice. This location will be needed to install P3DFFT.
  \item move to the new directory and run the following commands:
    \begin{verbatim}
      ./configure
      make
      make install
    \end{verbatim}
  \end{itemize}

For an overview on possible arguments to \texttt{configure} see the FFTW documentation: \textit{http://www.fftw.org/fftw3\_doc/}

\subsection{Install P3DFFT}
  This step has been tested on HECToR and BlueGene/Q. Specific instructions are given for these particular machines. For other machines the steps should be similar but it is not guaranteed to be simple.

\subsubsection{HECToR}
  \begin{itemize}
    \item extract the p3dfft.2.5.1.tar file of the library to a folder of your choice.
    \item go to the P3DFFT directory and run the following commands:
      \begin{verbatim}
        module switch PrgEnv-cray PrgEnv-pgi
        $ ./configure --prefix=$INSTALL_DESTINATION
          --enable-pgi --enable-stride1 --enable-fftw 
          --with-fftw=$FFTW_HOME FC=ftn CC=cc
         FCFLAGS=-O3 CFLAGS=-O3
        $ make
        $ make install
      \end{verbatim}
    \item The \texttt{module switch} command ensures the PGI compiler will be used.
    \item \texttt{INSTALL\_DESTINATION} is the location P3DFFT will be installed to. The default is \texttt{/usr/local}
    \item \texttt{FFTW\_HOME} is the location of the FFTW library. On HECToR the FFTW library is already installed and located at \texttt{/opt/fftw/3.3.0.1/interlagos}
    \item copy \texttt{config.h} from the P3DFFT directory to \texttt{INSTALL\_DESTINATION/include}.
  \end{itemize}

To see further flags that can be passed to \texttt{configure} see the P3DFFT users guide and wiki: \textit{http://code.google.com/p/p3dfft/wiki/Install}.

\subsubsection{BlueGene/Q}
  \begin{itemize}
    \item extract the p3dfft.2.5.1.tar file of the library to a folder of your choice.
    \item extract p3dfftbgqedit.tar to the same directory. This will overwrite some of the files in this directory in order to successfully install P3DFFT.
    \item go to the P3DFFT directory and run the following commands:
      \begin{verbatim}
        $ ./configure --prefix=$INSTALL_DESTINATION
          --enable-ibm --enable-stride1 --enable-essl 
          FC=mpixlf90_r CC=mpixlc_r
          FCFLAGS=-O3 CFLAGS=-O3
        $ make
        $ make install
      \end{verbatim}
    \item \texttt{INSTALL\_DESTINATION} is the location P3DFFT will be installed to. The default is \texttt{/usr/local}
    \item The ESSL library is located at \texttt{/opt/ibmmath/lib64/libesslbg.a}. This will be needed later.
    \item copy \texttt{config.h} from the P3DFFT directory to \texttt{INSTALL\_DESTINATION/include}.
  \end{itemize}
    
To see further flags that can be passed to \texttt{configure} see the P3DFFT users guide and wiki: \textit{http://code.google.com/p/p3dfft/wiki/Install}.

\section{Compiling Ludwig}
The steps in order to succesfully compile Ludwig and the unit tests are now outlined.

\subsection{Edit Makefile}
  The Makefiles in the \texttt{src/} and \texttt{tests/} directories contain the compiler flags necessary to compile Ludwig on HECToR and BlueGene/Q. The appropriate section should be uncommented.\\
  The user must also ensure that the link to the P3DFFT library is correct, along with the links to the FFTW or ESSL libraries.

\subsection{Compile Ludwig}
  Ludwig is compiled in the \texttt{src/} directory using the command:
  \begin{verbatim}
    $ make mpi
  \end{verbatim}
  This wil create an executable \texttt{Ludwig.exe} in this directory.

\subsection{Compile tests}
  To compile the unit tests, in the \texttt{src/} directory run the command
  \begin{verbatim}
    $ make libmpi
  \end{verbatim}
  Then in the \texttt{tests/} directory run
  \begin{verbatim}
    $ make tests_mpi
  \end{verbatim}
  If desired, and the option is available, the tests can be run on the front end by using
  \begin{verbatim}
    $ make do_tests_mpi
  \end{verbatim}
  in the \texttt{tests/} directory. This may take a long time to complete.


\section{Execute program}
  Ludwig can now be executed.
  Ludwig uses an input file to decide what parameters to use upon execution.
  By default it reads input from the file \texttt{input}, though the user can provide a different one at runtime.
  As an example, on HECToR the command
  \begin{verbatim}
    $ aprun -n 8 ./Ludwig.exe input
  \end{verbatim}
  will run Ludwig on 8 cores using the input file \texttt{input}.
  A file, \texttt{input.ref}, has been included as an example input file.
  Also included are the files used to run the Gouy Chapman and liquid junction potential problems, \texttt{GCinput} and \texttt{LJinput} respectively.\\
  Finally sample submission scripts are included for both HECToR and BlueGene/Q, \texttt{hectorludwig.pbs} and \texttt{bgqludwig.pbs} respectively. 

\end{document}

