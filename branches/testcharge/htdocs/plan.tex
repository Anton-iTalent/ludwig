%%%%%%%%%%%%%%%%%%%%%%%%%%%%%%%%%%%%%%%%%%%%%%%%%%%%%%%%%%%%%%%%%%%%%%%%%%%%%
%
%  plan.tex
%
%  Software Configuration Management Plan
%
%  $Id$
%
%  Edinburgh Soft Matter and Statistical Physics Group and
%  Edinburgh Parallel Computing Centre
%
%  Kevin Stratford (kevin@epcc.ed.ac.uk)
%  (c) 2011 The University of Edinburgh
%
%%%%%%%%%%%%%%%%%%%%%%%%%%%%%%%%%%%%%%%%%%%%%%%%%%%%%%%%%%%%%%%%%%%%%%%%%%%%%
\documentclass[11pt,twoside]{article}

\usepackage{amsmath}
\usepackage{moreverb}
\usepackage{lscape}
\usepackage{epic}
\usepackage[pdftex]{graphicx}
\usepackage{bm}
\usepackage{hyperref}
\usepackage{color}

\setlength{\hoffset}{-1in}
\setlength{\voffset}{-1in}

\setlength{\topmargin}{2cm}
\setlength{\evensidemargin}{3.1cm}
\setlength{\oddsidemargin}{3.1cm}

\setlength{\textwidth}{14.8cm}
\setlength{\textheight}{23cm}

\setlength{\parindent}{0pt}
\setlength{\parskip}{\smallskipamount}


\newcommand{\inputkey}[1]{\framebox{\textbf{\texttt{#1}}}}
\newcommand{\e}[1]{\cdot10^{#1}}
\newcommand{\beq}{\begin{equation}}
\newcommand{\eeq}{\end{equation}}
\newcommand{\beqa}{\begin{eqnarray}}
\newcommand{\eeqa}{\end{eqnarray}}
\newcommand{\com}[1]{\textcolor{red}{#1}}
\newcommand{\cur}[1]{{\textit{#1}}}

\begin{document}

\setcounter{page}{1}

\tableofcontents

\newpage

\setcounter{page}{1}

\section{Software Configuration Management Plan}

\subsection{Introduction}

\subsubsection{Purpose}

This section contains information on the \textit{Software Configuration
Management Plan} for the \textit{Ludwig} code.
It is to provide users of the code a background on the way in which
the code is developed, how bugs are dealt with, the way in which
new features may be added, and so on. It is therefore a part
of the process by which the code is maintained. The Software Configuration
Management Plan will be referred to as `The Plan' throughout the remainder
of the section.

\subsubsection{Scope}

The \textit{Ludwig} code is designed to study the hydrodynamic properties
of simple and complex fluids based around the numerical solution of the
Navier-Stokes equations. Complex fluids are dealt with via a free-energy
based finite difference approach, which couples explicitly to the
hydrodynamics. Examples of complex fluids include binary mixtures,
surfactants, liquid crystals, and active gels. Solid objects such as
porous media may be included, as well as moving particles (colloids).
Components of \textit{Ludwig} include the main program, unit and
regression tests, and a small number of utility programs used to prepare
input and post-process output. The Plan is limited to these
software components.

\textit{Ludwig} is specifically designed for parallel computers and
relies on the Message Passing Interface \cite{mpi-standard}. The
code will also run in serial, and is designed to be independent of
the platform it is run on. The Plan is therefore not concerned with
hardware or system management activities.

This document is meant to be fairly informal, but will be updated as
the need arises. 
\textit{Ludwig} is an on-going research project, and relies mainly
on funding for the United Kingdom Engineering and Physical Sciences Research
Council to provide support for staff time to work on maintenance and
development. As such, we cannot commit to any and all user requests!

\subsubsection{Definition of key terms}

\subsubsection{References}

This section is based on the IEEE standard for Software Configuration
Management Plans IEEE 828-2005 \cite{ieee-208}, and follows the format
set out therein.
Relevant references are included and can be found in the main References
section at the end of the document.

\subsection{Management}

\subsubsection{Organisations}

\textit{Ludwig} is developed by a team based in the School of Physics at
The University of Edinburgh. This involves two groups: the Soft Matter
Physics Group under the leadership of Prof. Mike Cates FRS, and EPCC in
the person of Kevin Stratford. We collaborate with a number of workers at
different Universities around the world on different aspects of the code.

\subsubsection{Responsibilities}

While the Soft Matter Physics Group is responsible for the overall
scientific direction and development of the code and concomitant
priorities, all
the Software Configuration Management activities will be the
responsibility of EPCC.

\subsubsection{Procedures}

There are currently no external constraints on The Plan.

\subsubsection{Process Management}

EPCC is responsible for the Software Configuration Management process.
It is anticipated that the cost of this process will be negligible, and
will not require monitoring. There is currently no independent
surveillance of activities to ensure compliance with The Plan.

\subsection{Activities}

\subsubsection{Identification}

The material under control of the project includes three sets of files:
this documentation, the source code, and the test suite which is used
to monitor the status of the code.
Control of project files is via a single on-line repository which
uses the subversion (SVN) revision control system. The SVN repository
is currently located at

\texttt{http://ccpforge.cse.rl.ac.uk/}

which is maintained at the Rutherford Appleton Laboratory on behalf
of Collaborative Computational Physics 5 (The Computer Simulation of
Condensed Phases). Subversion identifies different revisions by a
unique revision number. A given version of a file is then uniquely
identified by its path in the repository as it exists at a given
revision number. For example, SVN revision 1416 contains the
following directories:

\texttt{/trunk/current} a precursor code not under control

\texttt{/trunk/htdocs} source files for this documentation

\texttt{/trunk/mpi\_s} MPI stub library (scheduled to move)

\texttt{/trunk/src} source code

\texttt{/trunk/tests} unit and regression test suite

\texttt{/trunk/util} utility programs

At present access is restricted. However, we have committed to making
the code available to the public by October 2014. We expect this to
be via the same repository, referred to as CCPForge.
All members of the \textit{Ludwig} project are currently registered with
a CCPForge user account, and fall into one of three categories:
Administrator, Developer, and User. Only the first two categories have
permission to change files in the repository. Users have read-only
access. The exact mechanism for public access is yet to be decided,
but may still require users to register with CCPForge.

As CCPForge is not physically under our control, we undertake a
complete dump of the repository itself to a local resource on a
weekly basis for security.

\subsubsection{Control}

Requests for changes or additions to the code should be made via
the issue tracker at CCPForge. The project team will then decide
whether the change is possible. If so, the implementation and
testing of the change will take place, and the new code committed
back to the repository.


\subsubsection{Status}

Can we account for the status of the code and tests?

\subsubsection{Evaluation and review}

How do we evaluate and audit the code?

\subsubsection{Interface control}

At the moment, we do not explicitly interface to any code outside The
Plan (except MPI). We could extend coverage to include data export
to e.g., Paraview \cite{paraview}.

\subsubsection{Subcontractor / vendor control}

None required.

\subsubsection{Release management and delivery}

There is currently no formal release mechanism. If such a mechanism were
introduced, management and  delivery would be via the SVN repository.

\subsection{Schedules}

Sequence of events for SCM process.

\subsection{Resources}

For each active, state what resources (infrastructure, software tools,
personnel etc) are required.


\subsection{Plan Maintenance}

The Plan is currently in a draft form. A complete version of The Plan
will be in place to coincide with the public release.

\newpage
\section{Coordination Meetings}

We may want to keep a log here of various issues and points we discussed.

\subsection{Meeting 16 April 2013}

Alan, Kevin, Davide and Oliver were participating.
We had a general discussion about the scientific agenda for the 
next years, the status quo of the code, necessary development 
steps and open issues.

\subsubsection{Scientific Agenda}

The scientific agenda with Ludwig for the next few years could look like this:

\begin{itemize}
\item Colloids:\\
On the CPU colloids in LC are fully supported and the boundary condition for 
the anchoring has been recently corrected. This functionality should 
not require any further work on the CPU.

The GPU version could be interesting for very large colloids with respect to 
the pitch length. An upper limit of currently $2048^3$ lattice sites was mentioned.  

The GPU version could be also interesting for hydrodynamic aspects of 
glass formers. The typical model system would be a 20:80 Kob-Anderson 
mixture of radii 1:1.4.

\item Liquid crystal droplets and emulsions:\\
LC-emulsions are currently being developed in the hybrid code as Ludwig 
proved to be unstable. This may be due to spurious currents that do not
occur in the hybrid code as some terms are not modelled as divergence of a stress tensor.  
It would be maybe necessary to merge the models with Ludwig if possible 
once clarity has been gained how 3D-droplets can be modelled and
what causes the instability in Ludwig.

\item Electrokinetic phenomena:\\
The electrokinetics branch is currently emerging. A first project will be
on spinodal decomposition in the presence of charges. We will continure
with asymmetrically charged ionic liquids. Modelling the electrophoretic
behaviour of bacteria via colloids that have a conducting cell wall will
be another line of research.

\item Polymers:\\
Polymers are currently not supported, but are on the agenda. 
At the moment we don't have connectivity between the colloids,
which could form a polymer-bead-spring model. As Ludwig supports subgrid particles 
having polymers consisting of pointlike particles would be an interesting 
alternative in more dilute regimes.
 
A possible route to polymers in Ludwig could be Burkhard Duenweg's 
approach (reference missing here).

Another possible realisation could be to develp an add-on for 
LAMMPS which is based on Ludwig and LAMMPS would be used for the entire MD-part. 
At the moment it is not clear how the coupling could be achieved. Similar 
ideas have already surfaced (see F.E. Mackay, C. Denniston, 
'Coupling MD particles to a lattice-Boltzman fluid through the use of 
conservative forces', J. Comput. Phys. 237, 289-298 (2013).)  

\end{itemize}

\subsubsection{Status Quo and Open Issues}

There are currently three major branches in the repository:

\begin{itemize}
\item Main branch:\\
We are still working on fixes for a few issues with main branch. 
More precisely the anchoring of the LC at the 
surface of the colloids resulted in anisotropic states and cyclic permutation 
of the Cartesian coordinates did not lead to the same results. 
This issue has been fixed and appropriate unit test were suggested.
The question has been raised how we want to test new functionalities in general.
A possible solution might be to think about a test case against which
the new functionality can be tested, e.g. isotropy of the free energy density 
against permutation of the coordinates in the above case.
 
Another issue was the bug in the trace calculation which also has been removed.  
Also, incompressibility is now enforced after each time step.

We are still in the process of comparing Ludwig's output 
with Davide's hybrid code. This flagged 
up another problem related to the predictor-corrector algorithm which is 
currently not implemented in Ludwig, resulting in different convergence 
behaviour of specific test case presumable due to a lack of accuracy.  
Possible solutions are RK2, RK4 or semi-implicit methods.

The fluid-solid interaction via bounce-back is perhaps not state-of-the-art
anymore. In a discussion with Ignacio Pagonabarraga earlier this year
it turned out that alternative realisations like Yamamoto's approach 
(Y. Nakayama, R. Yamamoto, PRE 71, 036707 (2005).) with
a smooth profile may have advantages, but could make it more difficult 
to impose a variety of boundary conditions. Immersed boundary conditions
are generally unsuitable for complex fluids. 

\item GPU branch:\\
The GPU branch has diverged from the main branch during summer 2012.
We are currently thinking how it can be merged with the other branches.

\item Testcharge branch:\\
The testcharge branch with its electrokinetic functionality is still 
in statu nascendi. We have derived a Maxwell stress tensor whose divergence
is giving the force input into Ludwig. This still has to be tested.

The testcharge branch has seen a major technology refresh and will be merged with
the main branch.

\end{itemize}

\subsubsection{Upcoming Development Steps}

\begin{itemize}
\item We need to improve the accuracy in the finite difference part of the code. 
A first approach will use an RK2 or RK4 integrator because of simplicity.

\item The solution of the Poisson equation in the electrokinetic branch needs to be generalised
so that non-homogeneous dielectric permittivities can be used. We are currently
trying to identify possible solutions. An iterative procedure will have to be
found. 
Moreover, we need to test the implementation of the Maxwell stress tensor against the 
force input via gradients of the chemical potential.

Later this year the testcharge branch will be merged with the main branch in a major
technical update.

\item The merging of the GPU and the main branch requires suitable and coherent concepts
of abstraction to avoid code duplication. The dummy approach used for the serial
version of the code, which replaces MPI directives with a MPI neutral dummy, will
prove difficult to copy as CUDA C has some limitations.

\item It would be desireable to increase the internal user base so that issues 
can be spotted more easily.

\item For the planned public release in 2014/2015 we need to think about an appropriate
license.

\end{itemize} 

\end{document}
