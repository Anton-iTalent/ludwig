\section{Polar Active Gel}


Here we consider a polar vector field model to study active gels. 
The order parameter is this time a vector $P_\alpha$
(with variable magnitude) as there is no longer head-tail symmetry
in the system as was the case for the ${\bf Q}$ tensor formulation. 
The equation of motion as:
\begin{equation}
\partial_t P_\alpha  \partial_\beta (u_{\beta} + wP_{\beta}) P_{\alpha}
= \lambda D_{\alpha\beta} P_{\beta} - \omega_{\alpha\beta} P_{\beta}
+\Gamma' h_{\alpha}.
\end{equation}

In this equation, $w$ is an active term, due to swimming, which causes
self-advection of the order parameter, while $\lambda$ is a material
dependent constant -- positive for rod-like molecules. If $|\lambda|>1$
the liquid crystalline passive phase is flow-aligning, otherwise it is
flow-tumbling. 

Note that the convection for the Leslie-Ericksen is that the
velocity gradient tensor should be writen $W_{\alpha\beta}
= \partial_\alpha u_\beta$ (and not the other way round, which
is usual).

The molecular field is given by
$h_{\alpha}=-\delta {\cal F}_{\rm pol}/\delta p_{\alpha}$
where ${\cal F}_{\rm pol}$ is the free energy for a polar active nematic,
whose density we can take to be:
\begin{equation}
f =  \frac{\alpha_2}{2} |{\mathbf P}|^2 + \frac{\alpha_4}{4}|{\mathbf P}|^4
+ \frac{K}{2}\left(\partial_\alpha P_{\beta}\right)^2 + 
\frac{K_{LC}}{2}\left[\partial_\alpha (P_{\beta}P_{\gamma})\right]^2 
\end{equation}
where $K$ and $K_{LC}$ are elastic constants, $\alpha_4>0$, and $\alpha_2$ 
can be positive (isotropic phase) or negative (polar nematic phase).

The molecular field is explicitly given by:
\begin{equation}
h_{\alpha}=-\alpha_2 P_{\alpha} -\alpha_4 P_{\alpha} P_{\beta}P_{\beta}+
K \partial_{\beta}\partial_{\beta} P_{\alpha}+2 K_{LC}
P_{\gamma}\partial_{\beta}\partial_{\beta} (P_{\gamma}P_{\alpha}).
\end{equation}


The Navier-Stokes equation is:
\begin{equation}
\rho \left(\partial_t+v_{\beta}\partial_{\beta}\right) v_{\alpha} =
\partial_{\beta} \Pi_{\alpha\beta} 
+\eta \partial_{\beta}\partial_{\beta} v_{\alpha}.
\end{equation}


The stress tensor this time is 
\begin{eqnarray}
\Pi_{\alpha\beta} =
\frac{1}{2}\left(P_\alpha h_\beta - P_\beta h_\alpha \right)
- \lambda\left( (P_\alpha h_\beta + P_\beta h_\alpha)/2 
- (1/3)P_\gamma P_\gamma \delta_{\alpha\beta} \right)
\\
-  \zeta ( P_\alpha P_\beta - (1/3)P_\gamma P_\gamma \delta_{\alpha\beta})
- K \partial_\alpha P_\gamma \partial_\beta P_\gamma.
\end{eqnarray}
The active term in the stress tensor, proportional
to $\zeta$, has the same form for apolar and polar active gels. 

\subsection{User input}

\inputkey{free\_energy polar\_active}

This models a polar active gel with vector order parameter $P_\alpha$.
The free energy density is:


The corresponding input parameters are:

\inputkey{polar\_active\_a}

\inputkey{polar\_active\_b}

\inputkey{polar\_active\_k}

\inputkey{polar\_active\_klc}

\inputkey{polar\_active\_zeta}

\inputkey{polar\_active\_lambda}

\inputkey{leslie\_ericksen\_gamma}

\inputkey{leslie\_ericksen\_swim}


