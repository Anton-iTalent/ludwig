%%%%%%%%%%%%%%%%%%%%%%%%%%%%%%%%%%%%%%%%%%%%%%%%%%%%%%%%%%%%%%%%%%%%%%%%%%%%%
%
%  porous.tex
%
%  Some considerations for porous media.
%
%  $Id$
%
%  Edinburgh Soft Matter and Statistical Physics Group and
%  Edinburgh Parallel Computing Centre
%
%  Kevin Stratford (kevin@epcc.ed.ac.uk)
%  (c) 2011 The University of Edinburgh
%
%%%%%%%%%%%%%%%%%%%%%%%%%%%%%%%%%%%%%%%%%%%%%%%%%%%%%%%%%%%%%%%%%%%%%%%%%%%%%



\section{Porous Media}

Porous media calculations can be undertaken when appropriate
solid/fluid status information is supplied. The are a number
of switches available in the input file:

\inputkey{porous\_media\_file}

specifies the file stub name to be read at the start of execution.
(If the stub name is \texttt{file} then the code will expect to
find \texttt{file.001-001} in the current directory.

\inputkey{porous\_media\_format}

is either \texttt{ASCII} or \texttt{BINARY} as appropriate. Note that
in parallel, a single data file can be supplied, but it must be binary.
The default is \texttt{BINARY}.

\inputkey{porous\_media\_type}

is either \texttt{status\_only} (the default) or \texttt{status\_with\_h}.

\subsection{File format}

In all cases the status (fluid/solid) information is represented by a
single \texttt{char} (or integer), which must be supplied via the
porous media file. A single file should contain data matching the
current system size, and have the $z-$direction running fastest,
followed by the $y-$direction. Note that in non-periodic directions,
the structure must be 'closed', i.e., all the points at the edge
should be solid.

Fluid sites are designated by \texttt{0} and boundary or solid sites
by \texttt{1}. These data should be of type \texttt{char} in binary,
and may be integer in ASCII.

Where wetting information is required, the free energy parameter $H$
can be supplied by using the \texttt{status\_with\_h} switch. In this
case, the \texttt{char} status is augmented by a single \texttt{double}
value which is the local value of $H$. The order is then
$s_1,h_1, s_2, h_2, \ldots$.

An example of how to construct a porous media file is provided in
\texttt{util/capillary.c}, which builds an appropriate file for
a square or circular capillary tube. Please see the comments in
the file for further details. Note that the allowed
solid/fluid status values are defined in \texttt{src/site\_map.h}.
A solid boundary is \texttt{BOUNDARY}, while fluid is \texttt{FLUID}.

\subsection{Permeability calculations}

Single fluid permeability calculations for a given proous structure
can be undertaken by driving a flow via the fluid body force. Note
that the structure must be periodic in the direction of the force
(this may mean duplicating a 'reflected' version of a given sample
to create the correct input). The body force can be specified so
that, e.g., to drive a flow in the positive $x-$direction

\texttt{force 0.001\_0.000\_0.000}

The force should not be so large that the maximum velocity generated
threatens the Mach number constraint. To get a measurement of the
flow at equilibrium, the calculation should be run at least the
momentum diffusion time for the system ($L^2/\eta$ in LB time steps).

The net flow can be measured by combining the statistics for the
total momemtum and the total density (which is equal to the volume
with $\rho_0 = 1$).

Note that in the case of porous media with narrow channels at the grid
scale, the wall velocity can be dependent on the viscosity of the fluid.
This is an artefact of the bounce-back on links and will result in a
viscosity-dependent permeability (see, e.g., \cite{lipanmiller}).
To minimise this effect, e.g.,  Ginzburg and d'Humi\`eres \cite{ginzburg}
corrects the viscosity-dependence of the apparent boundary position
using a three relaxation time scheme. This is currently under
investigation.


\subsection{Example: circular capillaries}

The utility \texttt{capillary.c} can be used to generate various
porous media input files corresponding to various simple circular
or square capillaries.

For an infinite  capillary of circular cross section radius $a$,
the conductance is known analytically\cite{papanastasiou}. The
flow per unit area $J$ is given by
\begin{equation}
J = - \frac{1}{8\eta} \frac{\partial p}{ \partial x} a^2. 
\end{equation}
If the pressure gradient is replaced by a uniform body force
$-\partial p/ \partial x = \rho g$, then one can define a
viscosity-independent conductance $C$ via $ J = C \rho g / \eta$,
i.e., $C = a^2/8$. So, by measuring the rate of change of flow
with applied body force, one can compute an estimate of the conductance
to compare with this result.

If the total momentum is measured from the code for a given force
(in say the $x$-direction), then the flux per unit area is
$J = \sum_i \rho v_x / \sum_i \rho = \sum_i \rho v_x / V$
where the sum is over all fluid lattice sites $i$, with $V$
the fluid volume. The rate of change of flux with applied force
$\Delta J / \Delta\rho g$ can then be combined with the viscosity
to give the conductance.



\subsection{Example: rectangular capillaries}

An analtyical expression is also available for the conductance
of a square or rectugular capillary. For an infinite capillary
of square cross section width $w \times h$ (where $h$ is the
longer),
the volume flux per unit area $J$ is expected to be
\cite{papanastasiou,edo1}
\begin{equation}
J = - \frac{1}{3\eta} \frac{\partial p}{\partial x} (h/2)^2
\left[
1 - 6(h/w) \sum_{k=1}^{\infty} \frac{\tanh(\alpha_k w/h)}{\alpha_k^5} 
\right]
\end{equation}
where $\alpha_k = (2k - 1)\pi/2$. The pressure gradient
$-\partial p / \partial x$ is replaced by the body force $\rho g$ and
one can difine the viscosity-independnet conductance $C$ via
$J = C\rho g / \eta$. The calculation procedes as above.


\subsection{Matching lattice and real units in porous media}

Following Succi \cite{succi} (Chapter~8) the following argument
can be made to match lattice quantities and real physical
quantities for a given system of interest. Suppose we have
a structure with characteristic pore size $h = 1$ micron
and characteristic sample size of 1 cm. If the fluid
(e.g., water) has viscosity $\eta = 10^{-3}$ Pa s and density
$\rho = 10^3$ kg~s$^{-1}$, and a typical flow is
$U \sim 10^{-3}$ m~s$^{-1}$, then the pore scale Reynolds
number is $Re_h \sim 10^{-3}$.

If the speed of sound in water is 1500~m~s$^{-1}$,
then the corresponding time scale related to the
speed of sound and the pore size is $T = h/ c_s$,
which is $\sim 10^{-10}-10^{-9}$~s. (Exact matching
is acheived via the dimensionless group $\Delta x / c_s\Delta t$,
with the lattice speed of sound $1/\sqrt{3}$).

This, effectively very small, lattice
time step is clearly a problem in reaching any respectable 
macroscopic time in a simulation.
One step that can be taken is to increase both the lattice
viscosity and the lattice velocity so that the Reynolds number
remains unchanged. (Note that the lattice viscosity matching
the real viscoity might be $(\eta/\rho)T/h^2$ which, using the
above figures, is $\sim 10^{-4}$.) This can gain 2--3 orders
of magnitude while keeping the viscosity below $O(1)$. Ultimately,
it may also be necessary to raise the Reynolds number, which may
be considered small compared to unity as long as it remains
in practice less than $O(10^{-1})$ \cite{cates_scaling}.

