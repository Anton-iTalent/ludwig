%%%%%%%%%%%%%%%%%%%%%%%%%%%%%%%%%%%%%%%%%%%%%%%%%%%%%%%%%%%%%%%%%%%%%%%%%%%%%%
%
%  surfactant.tex
%
%  Some notes on surfactant work
%
%%%%%%%%%%%%%%%%%%%%%%%%%%%%%%%%%%%%%%%%%%%%%%%%%%%%%%%%%%%%%%%%%%%%%%%%%%%%%%

\section{Surfactants}

\subsection{Note on Experimental Measurements}

Experimental measurements have generally been based on mechanical
methods which allow the determination of an interfacial tension
(for recent reviews, see \cite{chang,eastoe}). The degree of surface
activity is then accessed by measuring the the interfacial tension
$\sigma$ as a function of the bulk concentration $\psi_b$, and then
using the Gibbs equation
\begin{equation}
\psi_{0,\mathrm{eq}} = -\frac{1}{nRT}
\left( \frac{\partial \sigma}{\partial \ln \psi_b} \right)_T
\end{equation}
where $\psi_{0,\mathrm{eq}}$ is the equilibrium surface excess
concentration. The curve of $\psi_{0,\mathrm{eq}}$ against $\psi_b$
so obtained is thus relevant for a fixed temperature and is generally
refered to as the \textit{adsorption isotherm}. More recently, the
method of neutron reflection has been used to to access the interfacial
concentration directly (with good agreement with the mechanical
methods).

\subsubsection{Dynamic surface tension}

A newly formed interface (e.g., after shaking) in a surfactant
solution will have interfacial tension close to the bare value
$\sigma_0$. As the surfactant concentration at the interface
approaches its equilibrium value, the interfacial tension is
reduced. The quantity of interest is then the \textit{dynamical
surface tension} $\sigma(t)$. This process takes place on
timescales typically 10$^{-3}$--1 second, and is important in
many industrial processes such as coatings and formation of
soap lather.

At equilibrium, the final change in interfacial tension can
be obtained from Gibbs' equation together with a suitable
isotherm relating $\psi_0$ and $\psi_b$. 

\subsubsection{Adsorption isotherms}

A simple choice of isotherm is Henry's law:
\begin{equation}
\psi_{0,\mathrm{eq}} = K_H \psi_b
\label{eq:iso:henry}
\end{equation}
where $K_H$ is the equilibrium adsorption constant. Note this has the
dimensions of length, and represents the thickness of bulk solution
holding the same quantity of surfactant as an interface with
concentration $\psi_{0,\mathrm{eq}}$. In this way it is also a
measure of the strength of surface activity for a given surfactant.
Henry's law has a number of drawbacks: there is no upper limit
on $\psi_0$, and is only valid at low concentrations.

A more practical choice is the Langmuir equilibrium isotherm
\begin{equation}
\psi_{0,\mathrm{eq}} = \psi^\star \frac{K_L \psi_b}{1 + K_L \psi_b}
\label{eq:iso:langmuir}
\end{equation}
where $\psi^\star$ is the maximum concentration supported at the
interface and the Langmuir equilibrium adsorption constant is $K_L$
(with units mol$^{-1}$~m$^3$). At low concentrations $ K_L\psi_b << 1$,
the Langmuir isotherm reduces to the Henry isotherm with
$K_H = \psi^\star K_L$.

One further case of interest here is the Frumkin isotherm
\begin{equation}
\psi_{0, \mathrm{eq}} = \psi^\star
\frac{K_F \psi_b}{K_F\psi_b + e^{-B\psi_{0,\mathrm{eq}} / \psi^\star}}
\label{eq:iso:frumkin}
\end{equation}
where the Frumkin equilibrium adsorption constant $K_F$ again
has units of mol$^{-1}$~m$^3$. The dimensionless parameter $B$
 is a measure of the degree to which the surfactant behaviour
at the interface is non-ideal. For $B=0$, the Frumkin isotherm
reduces to that of Langmuir.

\subsection{Note on the Theory of Diamant and Andelman 1996}

As we ultimately adopt a free energy approach, we first review one
free energy based work --- that of Diamant and Andelman
\cite{diamant96}. This introduces a uniform system with interface
of zero thickness at $z = 0$. A uniform dilute solution of
surfactant occupies $z > 0$. The excess free energy per unit area
of interface is then written
\begin{equation}
\Delta\sigma [\psi] = \int_0^\infty  \big\{ \Delta f[\psi (z)] +
f_0 [\psi (z) ] \delta(z)  \big\} \mathrm{d}z. 
\end{equation}
The first term here is a contribution from the bulk while the
second is that from the interface. 

The uniform bulk solution has an ideal entropy of mixing
\begin{equation}
\Delta f(\psi) = (1/a^3)
\big\{ kT[\psi \ln\psi - \psi - (\psi_b \ln\psi_b - \psi_b)]
- \mu_b (\psi - \psi_b) \big\}.
\label{eq:da2}
\end{equation}
Here, $a$ is the characteristic size of the surfactant molecules.
The chemical potential $\mu_b$ and volume fraction of surfactant
in the bulk are fixed (with $\psi_b < 1$). The final term in
Eq.~\ref{eq:da2} is a contact term with a reservoir at $z = \infty$.

At the interface, we have a surfactant concentration $\psi_0$, which
is no longer small:
\begin{equation}
f_0 (\psi_0) = (1/a^2) \big\{
kT [ \psi_0 \ln\psi_0 + (1 - \psi_0) \ln(1 - \psi_0)]
- \alpha\psi_0 - {\textstyle\frac{1}{2}}\beta\psi_0^2 - \mu_1\psi_0 \big\}.
\end{equation} 
The first term is the full entropy of mixing. The second term represents
the reduction ($\alpha$ is positive) in surface energy owing to the
presence of surfactant. The third term represents an attractive
interaction between surfactants at the interface ($beta$ again positive).
The final term is a contact term between the interface and the adjacent
solution (at $z\rightarrow 0$).

\subsubsection{Equilibrium}

At equilibrium, where the chemical potential is the same throughout
the system $\mu (z) = \mu_b$, the DA recover the adsorption isotherm
\begin{equation}
\psi_{0,\mathrm{eq}} = \frac{\psi_b}{\psi_b
+ e^{-(\alpha + \beta\psi_{0,\mathrm{eq}})/kT}}.
\label{eq:iso:diamant}
\end{equation}
Comparison with the experiment forms Eq.~\ref{eq:iso:langmuir} and
Eq.~\ref{eq:iso:frumkin} reveals that $K_F$ is equivalent to
$e^{\alpha /kT}$ in the Frumkin isotherm. In addition, the equilibrium
equation of state is
\begin{equation}
\Delta \gamma_{\mathrm{eq}} = (1/a^2) [ kT \ln (1 - \psi_0) 
+ {\textstyle\frac{1}{2}}\beta \psi_0^2 ].
\end{equation}

\subsubsection{Out of Equilibrium}

DA96 go on to derive a variant of the Ward-Tordai equation \cite{wardtordai}
which describes the time evolution of the concentration of surfactant
at the interface:
\begin{equation}
\psi_0 (t) = \sqrt{D/\pi a^2}
\Bigg[ 2\psi_b \sqrt{t} 
- \int_0^\infty \frac{\psi_1 (\tau)}{\sqrt{t - \tau}} \mathrm{d}\tau \Bigg]
+ 2\psi_b - \psi_1.
\label{eq:wardtordai}
\end{equation}
When the time-scale for equilibration at the interface is fast compared
with that in the bulk, $\psi_0$ responds immediately to changes in
$\psi_1$. This is the case of \textit{diffusion limited adsorption} (DLA).
Here, the diffusive problem Eq.~\ref{eq:wardtordai} can be closed
using the isotherm Eq.~\ref{eq:iso:diamant} with $\psi_b$ replaced by
$\psi_1$. This can be solved numerically for $\psi_0 (t)$. If one further
makes the assumption that the equilibrium equation of state holds
approximately out of equililbrium, then one can also recover
$\Delta\gamma (t)$, that is, the dynamic surface tension.

\subsection{Current Model}

The current model is based the standard symmetric binary fluid
free energy to represent two fluids separated by a finite
interface of thickness $\xi_0$. This free energy is the usual
functional of the compositional order parameter $\phi$. This is
referred to as
\begin{equation}
f_\phi = {\textstyle\frac{1}{2}}A\phi^2
+ {\textstyle\frac{1}{4}}B\phi^4
+ {\textstyle\frac{1}{2}}\kappa (\partial_\alpha \phi)^2.
\end{equation}
Note that here, the excess free energy related to $f_\phi$ may
be integrated analytically to provide an expression for the
intefacial tension $\sigma_0 = 4\kappa\phi^\star/3\xi_0$ at
equilibrium.
(In one dimension the equiibrium profile
is $\phi(z) = \phi^\star \tanh(z/\xi_0)$ with $\phi^\star = \sqrt{-A/B}$
and $\xi_0^2 = -2\kappa/A$.)

To this we add a separate order parameter to represent the
surfactant $\psi$ \cite{smangraaf}. The free energy density
is made up of two parts following DA96:
\begin{equation}
f_{\psi} = kT[\psi\ln\psi  + (1 - \psi) \ln (1 - \psi)]
\end{equation}
represents the entropy of mixing, valid in both the bulk and at
the interface. The interfacial part of the free energy density is
\begin{equation}
f_{\psi,i} = -{\textstyle\frac{1}{2}}\epsilon\psi (\partial_\alpha \phi)^2
- {\textstyle\frac{1}{2}} \beta \psi^2 (\partial_\alpha \phi)^2.
\end{equation}

We note that the excess free energy can no longer be integrated
exactly. However, it can be integrated numerically for a
given profile $\psi(z)$ (see below).

\subsubsection{Equilibrium}

We can compute the chemical potential associated with each order
parameter by taking the functional derivative with respect to
$\phi$ and $\psi$ respectively:
\begin{equation}
\mu_\phi = A\phi + B\phi^3 - \kappa \partial_\alpha^2 \phi
+ \epsilon \partial_\alpha \phi \partial_\alpha \psi
+ \epsilon \psi \partial_\alpha^2 \phi
+ 2\beta\psi \partial_\alpha \phi \partial_\alpha \psi
+ \beta\psi^2 \partial_\alpha^2 \phi,
\label{eq:mu:phi}
\end{equation}
and
\begin{equation}
\mu_\psi = kT[\ln\psi - \ln(1 - \psi)]
- {\textstyle \frac{1}{2}} \epsilon (\partial_\alpha \phi)^2
- \beta\psi (\partial_\alpha \phi)^2.
\label{eq:mu:psi}
\end{equation}

At equilibrium, we can set the chemical potential in the bulk and
at the interface to be equal. We assume that the presence of
surfactant does not affect the form of the equilibrium
composition profile so that in one dimension
$\phi(z) = \phi^\star \tanh(z/\xi_0)$ as before.
In the bulk we assume the concentration $\psi_b << 1$ so that
Eq.~\ref{eq:mu:psi} gives $\mu_{\psi,b} = kT\ln \psi_b$. At
the interface we ahve $\mu_{\phi,i} = kT[\ln\psi_0 - \ln(1 - \psi_0)]
- (1/2)\epsilon(\partial_\alpha \phi)^2
- \beta\psi_0 (\partial_\alpha \phi)^2$.
If we equate $\mu_{\psi, b}$ and $\mu_{\psi,i}$ we can recover the
Frumkin isotherm
\begin{equation}
\psi_{0,\mathrm{eq}} = \frac{\psi_b}{\psi_b +
e^{-({\scriptscriptstyle \frac{1}{2}}\epsilon + \beta\psi_{0,\mathrm{eq}})
(\phi^\star)^2 / \xi_0^2 kT}}.
\end{equation}
This is equivalent to Eq.~\ref{eq:iso:diamant} of DA96, where the
interfacial width has has appeared in place of $a$ (absorped into
the definition of the free energy density by DA96).


In the following we have $D = M_\psi \psi ( 1 - \psi)$. The
initial interfacial width is $\xi = 1.13$ (narrower than the
equilibrium value). This is a one-dimensional system of
length 100 lattice units. We choose to fix the binary fluid
parameters, and fix $\epsilon$ in relation to $\kappa$. The
isotherm is then varied by adjusting the value of $kT$.

\subsubsection{Parameters with $\mathbf{\xi_0 = 1.13}$, $\epsilon = \kappa/4$}

Here we keep the standard binary fluid parameters, having interfacial
width $\xi_0 = 1.13$.

\begin{table}[h]
\begin{center}
\begin{tabular}{|c|c|c|c|c|c|c|c|c|c|}
\hline
Name & $\eta$ & $-A,B$ & $\kappa$ & $kT$ & $\epsilon$ & $\beta$ & $W$
     & $M$ & $M_\psi$\\
\hline
R008a & 1/6 & 0.0625 & 0.04 & 1.6965$\times 10^{-3}$ & 0.01 & 0.0000 & 0.0 
      & 0.15 & 1.0 \\
R008d &  &  &  & & & 0.0025 &  &  &  \\
R008f &  &  &  & & & 0.0050 &  &  &  \\
\hline
R006a & 1/6 & 0.0625 & 0.04 & 8.48231$\times 10^{-4}$ & 0.01 & 0.0000 & 0.0 
      & 0.15 & 1.0 \\
R006d & &  &  &  & & 0.0025 &  &  &  \\
R006f & &  &  &  & & 0.0050 &  &  &  \\
\hline
\end{tabular}
\caption{Parameters}
\end{center}
\end{table}




\subsubsection{Parameters with $\mathbf{\xi_0 = 2.26}$, $\epsilon = \kappa/4$}

Here we relax the interfaical width by a factor of 2, but
keep $A,-B$ as before so that the surface tension is increased
by a factor of $\sqrt{2}$.

\begin{table}[h]
\begin{center}
\begin{tabular}{|c|c|c|c|c|c|c|c|c|c|}
\hline
Name & $\eta$ & $-A,B$ & $\kappa$ & $kT$ & $\epsilon$ & $\beta$ & $W$
     & $M$ & $M_\psi$\\
\hline
R002a & 1/6 & 0.0625 & 0.16 & 1.6965$\times 10^{-3}$ & 0.04 & 0.00 & 0.0 
      & 0.15 & 1.0 \\
R002d &  &  &  & & & 0.01 &  &  &  \\
R002f &  &  &  & & & 0.02 &  &  &  \\
\hline
R003a & 1/6 & 0.0625 & 0.16 & 5.6549$\times 10^{-4}$ & 0.04 & 0.00 & 0.0 
      & 0.15 & 1.0 \\

R003d & &  & &  &  & 0.01 &  &  & 1.0\\
R003e & &  & &  &  & 0.01 &  &  & 0.5\\
R003f & &  & &  &  & 0.02 &  &  & 1.0\\
\hline
R004a & 1/6 & 0.0625 & 0.16 & 8.48231$\times 10^{-4}$ & 0.04 & 0.00 & 0.0 
      & 0.15 & 1.0 \\
R004d & &  &  &  & & 0.01 &  &  &  \\
R004f & &  &  &  & & 0.02 &  &  &  \\

\hline
\end{tabular}
\caption{Parameters for the surfactant free energy model. There are
three values of the (Langmuir) isotherm $(1/K_L) = 0.1$ (R002a),
0.01 (R004a) and 0.001 (R003a).
See section X for notes on $W$.}
\end{center}
\end{table}


\begin{table}
\begin{center}
\begin{tabular}{|c|c|c|c|c|c|}
\hline
Name & $\bar{\psi}$ & $\psi_b$ & $\psi_{0\mathrm{eq}}$
     & $\sigma / \sigma_0$  & Comment \\
\hline
R008a &  $2.0\times 10^{-1}$ & $1.859\times 10^{-1}$ & $5.143\times10^{-1}$
      & 0.76 & ok\\
R008a &  $1.0\times 10^{-1}$ & $9.093\times 10^{-2}$ & $3.033\times10^{-1}$
      & 0.80 & ok\\
R008a &  $5.0\times 10^{-2}$ & $4.490\times 10^{-2}$ & $1.643\times10^{-1}$
      & 0.82 & ok\\
R008a &  $2.5\times 10^{-2}$ & $2.231\times 10^{-2}$ & $8.540\times10^{-2}$
      & 0.84 & ok\\
R008a &  $1.0\times 10^{-2}$ & $8.888\times 10^{-3}$ & $3.494\times10^{-2}$
      & 0.86 & ok\\
R008a &  $1.0\times 10^{-3}$ & $8.867\times 10^{-4}$ & $3.540\times10^{-3}$
      & 0.87 & ok\\
\hline
R008d &  $2.0\times 10^{-1}$ &  &  &  & unstable \\
R008d &  $1.0\times 10^{-1}$ &  &  &  & unstable \\
R008d &  $5.0\times 10^{-2}$ &  &  &  & unstable \\
R008d &  $2.5\times 10^{-2}$ &  &  &  & no equilibrium \\
R008d &  $1.0\times 10^{-2}$ & $8.455\times 10^{-3}$ & $4.535\times10^{-2}$
      & 0.85 & ok \\
R008d &  $1.0\times 10^{-3}$ & $8.835\times 10^{-4}$ & $3.617\times10^{-3}$
      & 0.87 & ok \\
\hline
R006a &  $2.0\times 10^{-1}$ & $1.713\times 10^{-1}$ & $8.510\times10^{-1}$
      & 0.71 & ok\\
R006a &  $1.0\times 10^{-1}$ & $7.556\times 10^{-2}$ & $6.657\times10^{-1}$
      & 0.74 & ok\\
R006a &  $5.0\times 10^{-2}$ & $3.406\times 10^{-2}$ & $4.211\times10^{-1}$
      & 0.78 & ok\\
R006a &  $2.5\times 10^{-2}$ & $1.616\times 10^{-2}$ & $2.310\times10^{-1}$
      & 0.81 & ok\\
R006a &  $1.0\times 10^{-2}$ & $6.286\times 10^{-3}$ & $9.653\times10^{-2}$
      & 0.84 & ok\\
R006a &  $1.0\times 10^{-3}$ & $6.192\times 10^{-4}$ & $9.868\times10^{-3}$
      & 0.87 & ok\\
\hline
R006d &  $2.0\times 10^{-1}$ & $1.664\times 10^{-1}$ & $9.791\times10^{-1}$
      & 0.64 & ok\\
R006d &  $1.0\times 10^{-1}$ &  & &  & unstable\\
R006d &  $5.0\times 10^{-2}$ &  & &  & unstable\\
R006d &  $2.5\times 10^{-2}$ &  & &  & unstable\\
R006d &  $1.0\times 10^{-2}$ & $2.9\times 10^{-4}$ & $4.75\times10^{-1}$
      & 0.80 & unsymmetric\\
R006d &  $1.0\times 10^{-3}$ & $5.822\times 10^{-4}$ & $1.077\times10^{-2}$
      & 0.87 & ok \\
\hline
\end{tabular}
\label{table:xi113}
\caption{Equilibrium results for $\xi_0 = 1.13$, $\epsilon = \kappa/4$.}
\end{center}
\end{table}




\begin{table}
\begin{center}
\begin{tabular}{|c|c|c|c|c|c|}
\hline
Name & $\bar{\psi}$ & $\psi_b$ & $\psi_{0\mathrm{eq}}$
     & $\sigma / \sigma_0$  & Comment \\
\hline
R002a &  $2.0\times 10^{-1}$ & $1.701\times 10^{-1}$ & $7.423\times10^{-1}$
      & 0.88 & ok\\
R002a &  $1.0\times 10^{-1}$ & $8.008\times 10^{-2}$ & $5.023\times10^{-1}$
      & 0.90 & ok\\
R002a &  $5.0\times 10^{-2}$ & $3.871\times 10^{-2}$ & $2.886\times10^{-1}$
      & 0.93 & ok\\
R002a &  $2.5\times 10^{-2}$ & $1.904\times 10^{-2}$ & $1.530\times10^{-1}$
      & 0.95 & ok\\
R002a &  $1.0\times 10^{-2}$ & $7.456\times 10^{-2}$ & $6.305\times10^{-2}$
      & 0.96 & ok \\
R002a &  $1.0\times 10^{-3}$ & $7.506\times 10^{-3}$ & $6.407\times10^{-3}$
      & 0.97 & ok \\
\hline
R002d &  $2.0\times 10^{-1}$ & $1.598\times 10^{-1}$ & $9.443\times10^{-1}$
      & 0.85 & ok \\
R002d &  $1.0\times 10^{-1}$ & $7.267\times 10^{-2}$ & $7.540\times10^{-1}$
      & 0.88 & ok \\
R002d &  $5.0\times 10^{-2}$ & $3.615\times 10^{-2}$ & $3.894\times10^{-1}$
      & 0.92 & ok \\
R002d &  $2.5\times 10^{-2}$ & $1.841\times 10^{-2}$ & $1.776\times10^{-1}$
      & 0.94 & ok \\
R002d &  $1.0\times 10^{-2}$ & $7.449\times 10^{-3}$ & $6.680\times10^{-2}$
      & 0.96 & ok \\
R002d &  $1.0\times 10^{-3}$ & $7.497\times 10^{-4}$ & $6.444\times10^{-3}$
      & 0.97 & ok \\
\hline
R002f &  $2.0\times 10^{-1}$ & $1.512\times 10^{-1}$ & $9.972\times10^{-1}$
      & 0.80 &  ok\\
R002f &  $1.0\times 10^{-1}$ & $5.994\times 10^{-2}$ & $9.751\times10^{-1}$
      & 0.80 & ok \\
R002f &  $5.0\times 10^{-2}$ & $3.002\times 10^{-2}$ & $6.947\times10^{-1}$
      & 0.90 & ok \\
R002f &  $2.5\times 10^{-2}$ & $1.746\times 10^{-2}$ & $2.202\times10^{-1}$
      & 0.95 & ok \\
R002f &  $1.0\times 10^{-2}$ & $7.338\times 10^{-3}$ & $7.132\times10^{-2}$
      & 0.96 & ok \\
R002f &  $1.0\times 10^{-3}$ & $7.487\times 10^{-4}$ & $6.480\times10^{-3}$
      & 0.97 &  ok\\
\hline
R003a &  $1.0\times 10^{-1}$ & $4.168\times 10^{-2}$ & $9.979\times10^{-1}$
      & 0.85 & ok\\
R003a &  $1.0\times 10^{-2}$ & $5.7\times 10^{-4}$ & $3.563\times10^{-1}$
      & 0.95 & ok\\
R003a &  $1.0\times 10^{-3}$ & $5.7\times 10^{-5}$ & $3.590\times10^{-2}$
      & 0.97 & ok\\
\hline
R003d & $1.0\times 10^{-1}$& & & & unstable\\
R003e & $1.0\times 10^{-1}$& & & & unstable\\
R003f & $1.0\times 10^{-1}$& & & & unstable\\
\hline
R004a &  $1.0\times 10^{-1}$ & $5.475\times 10^{-2}$ & $9.476\times10^{-1}$
      & 0.87 & ok\\
R004a &  $5.0\times 10^{-2}$ & $1.960\times 10^{-2}$ & $7.935\times10^{-1}$
      & 0.89 & ok\\
R004a &  $2.5\times 10^{-2}$ & $8.113\times 10^{-3}$ & $5.029\times10^{-1}$
      & 0.93 & ok\\
R004a &  $1.0\times 10^{-2}$ & $3.043\times 10^{-3}$ & $2.160\times10^{-1}$
      & 0.95 & ok\\
R004a &  $1.0\times 10^{-3}$ & $2.999\times 10^{-4}$ & $2.191\times10^{-2}$
      & 0.97 & ok\\
\hline
R004d &  $1.0\times 10^{-1}$ & $4.365\times 10^{-2}$ & $9.995\times10^{-1}$
      & 0.82 & ok\\
R004d &  $5.0\times 10^{-2}$ & $9.260\times 10^{-3}$ & $9.829\times10^{-1}$
      & 0.84 & ok\\
R004d &  $2.5\times 10^{-2}$ & $3.896\times 10^{-3}$ & $8.405\times10^{-1}$
      & 0.91 & ok\\
R004d &  $1.0\times 10^{-2}$ & $2.198\times 10^{-3}$ & $2.854\times10^{-1}$
      & 0.95 & ok\\
R004d &  $1.0\times 10^{-3}$ & $2.927\times 10^{-4}$ & $2.246\times10^{-2}$
      & 0.97 & ok\\
\hline
R004f &  $1.0\times 10^{-1}$ &  &    &  & unstable\\
R004f &  $5.0\times 10^{-2}$ & $9.183\times 10^{-3}$ & $1.000\times10^{0}$
      & 0.80 & ok\\
R004f &  $2.5\times 10^{-2}$ & &   &  & no equilibrium\\
R004f &  $1.0\times 10^{-2}$ & $1.045\times 10^{-3}$ & $3.965\times10^{-1}$
      & 0.94 & ok\\
\hline
\end{tabular}
\label{table:xi226}
\caption{Equilibrium results for $\xi_0 = 2.26$, $\epsilon = \kappa/4$.}
\end{center}
\end{table}


\subsubsection{Dynamics}

Solve the Ward-Tordai equation.



\vfill
\pagebreak
