%% ****** Start of file apstemplate.tex ****** %
%%
%%
%%   This file is part of the APS files in the REVTeX 4 distribution.
%%   Version 4.1r of REVTeX, August 2010
%%
%%
%%   Copyright (c) 2001, 2009, 2010 The American Physical Society.
%%
%%   See the REVTeX 4 README file for restrictions and more information.
%%
%
% This is a template for producing manuscripts for use with REVTEX 4.0
% Copy this file to another name and then work on that file.
% That way, you always have this original template file to use.
%
% Group addresses by affiliation; use superscriptaddress for long
% author lists, or if there are many overlapping affiliations.
% For Phys. Rev. appearance, change preprint to twocolumn.
% Choose pra, prb, prc, prd, pre, prl, prstab, prstper, or rmp for journal
%  Add 'draft' option to mark overfull boxes with black boxes
%  Add 'showpacs' option to make PACS codes appear
%  Add 'showkeys' option to make keywords appear
%\documentclass[aps,prl,preprint,groupedaddress]{revtex4-1}
%\documentclass[aps,prl,preprint,superscriptaddress]{revtex4-1}
%\documentclass[aps,prl,reprint,groupedaddress]{revtex4-1}
\documentclass[11pt, oneside, a4paper]{article}

\usepackage{algorithm}
\usepackage{algorithmic}
\usepackage{amsmath}

% You should use BibTeX and apsrev.bst for references
% Choosing a journal automatically selects the correct APS
% BibTeX style file (bst file), so only uncomment the line
% below if necessary.
%\bibliographystyle{apsrev4-1}

\begin{document}

% Use the \preprint command to place your local institutional report
% number in the upper righthand corner of the title page in preprint mode.
% Multiple \preprint commands are allowed.
% Use the 'preprintnumbers' class option to override journal defaults
% to display numbers if necessary
%\preprint{}

%Title of paper
\title{A Parallel Geometric Multigrid Algorithm}

% repeat the \author .. \affiliation  etc. as needed
% \email, \thanks, \homepage, \altaffiliation all apply to the current
% author. Explanatory text should go in the []'s, actual e-mail
% address or url should go in the {}'s for \email and \homepage.
% Please use the appropriate macro foreach each type of information

% \affiliation command applies to all authors since the last
% \affiliation command. The \affiliation command should follow the
% other information
% \affiliation can be followed by \email, \homepage, \thanks as well.
\author{Oliver Henrich$^1$, Kevin Stratford$^1$, S\'ebastien Loisel$^2$\\
$^1$ EPCC, School of Physics and Astronomy,\\
University of Edinburgh, Edinburgh EH9 3JZ, UK\\
$^2$ Department of Mathematics,\\ 
Herriot-Watt University, Edinburgh EH14 4AS \\
}
%\email[]{Your e-mail address}
%\homepage[]{Your web page}
%\thanks{}
%\altaffiliation{}
%\affiliation{}
\date{\today}
%Collaboration name if desired (requires use of superscriptaddress
%option in \documentclass). \noaffiliation is required (may also be
%used with the \author command).
%\collaboration can be followed by \email, \homepage, \thanks as well.
%\collaboration{}
%\noaffiliation


% insert suggested PACS numbers in braces on next line
%\pacs{}
% insert suggested keywords - APS authors don't need to do this
%\keywords{}
%\begin{abstract}
%Blablabla
%\end{abstract}

%\maketitle must follow title, authors, abstract, \pacs, and \keywords
\maketitle

% body of paper here - Use proper section commands
% References should be done using the \cite, \ref, and \label commands
\section{Definitions}

\begin{eqnarray*}
k&\dots&\mbox{grid level with $k=1$ being the coarsest grid}\\
\Omega_k &\dots& \mbox{regular grid with spacing $h(k)$}\\
\partial \Omega_k &\dots& \mbox{boundary of computational domain}\\
m&\dots& \mbox{iteration index}\\
u_k&\dots& \mbox{exact solution on $\Omega_k$}\\ 
v_k^m&\dots& \mbox{approximate solution on $\Omega_k$ after $m$ iterations}\\
e_k^m&\dots& \mbox{error on $\Omega_k$ after $m$ iterations}\\ 
R_h^H, R_k^{k-1}&\dots& \mbox{restriction operator: mapping functions on grid level $k$ with} \\
&& \mbox{spacing $h$ to functions on the next coarser level $k-1$ with spacing $H=2h$}\\ 
I_H^h, I_{k-1}^k&\dots& \mbox{interpolation operator: mapping functions on grid level $k-1$ with }\\
&&\mbox{spacing $H=2h$ to functions on the next finer level $k$ with spacing $h$}\\ 
\end{eqnarray*}


\section{Transfer Operators}

\subsection{Restriction}

We use half weighting (HW) as this restriction operator has reduced communication requirements. In stencil notation the effect of this operator is

\begin{eqnarray*}
f_H(x,y,z) &=& R_h^H f_h(x,y,z) = \frac{1}{12}[ f_h(x-h,y,z)+f_h(x+h,y,z)+f_h(x,y-h,z)\\
&& +f_h(x,y+h,z)+f_h(x,y,z-h)+f_h(x,y,z+h)+6\;f_h(x,y,z)].
\end{eqnarray*}

The operations are independent of each other and can be performed in parallel. The computation can also be overlapped with communication of the halo data.

\subsection{Interpolation}

We use trilinear interpolation. Note that the coordinate indices $x,y$ and $z$ refer to the indices on the fine grid $\Omega_h$.

\begin{equation*}
f_h(x,y,z) = I_H^h f_H(x,y,z)
\end{equation*}

\begin{equation*}
f_h(x,y,z) =\\
\begin{cases}
\;\;\;f_H(x,y,z)& , \mbox{ case } 1\\
\frac{1}{2} [f_H(x,y-h,z)+f_H(x,y+h,z)] &,  \mbox{ case } 2\\
\frac{1}{2} [f_H(x-h,y,z)+f_H(x+h,y,z)] &,  \mbox{ case } 3\\
\frac{1}{4} [f_H(x-h,y-h,z)+f_H(x+h,y-h,z) \\
\;\;\;+f_H(x-h,y+h,z)+f_H(x+h,y+h,z)] &,  \mbox{ case } 4\\
\frac{1}{8} [f_H(x-h,y-h,z-h)+f_H(x+h,y-h,z-h)\\
\;\;\;+f_H(x-h,y+h,z-h)+f_H(x+h,y+h,z-h)\\
\;\;\;+f_H(x-h,y-h,z+h)+f_H(x+h,y-h,z+h)\\
\;\;\;+f_H(x-h,y+h,z+h)+f_H(x+h,y+h,z+h) &, \mbox{ case } 5
\end{cases}
\vspace*{0.5cm}
\end{equation*}

We distinguish the following cases:

\begin{enumerate}
\item points with $x,y,z\in\Omega_H$ 
\item points with $x,z\in \Omega_H$, $y\notin \Omega_H$  
\item points with $y,z\in \Omega_H$, $x\notin \Omega_H$
\item points with $z\in \Omega_H$, $x,y\notin \Omega_H$ 
\item points with $x,y,z\notin \Omega_H$. 
\end{enumerate}

Also these operations are independent of each other and can be performed in parallel. However, it might not be possible to overlap computation and communication.

\section{Relaxation}

The equation we like to solve is the Poisson Equation $A u = f$ with $A=L=-\Delta$. 
We use the standard 7-point discretisation for the Laplacian:

\begin{eqnarray*}
A_h u_h(x,z,z) &=& \frac{1}{h^2} [6 u_h(x,y,z) - u_h(x+h,y,z) - u_h(x-h,y,z)-u_h(x, y+h,z) \\
 &-& u_h(x,y-h,z) - u_h(x,y,z+h) - u_h(x,y,z-h)] = f_h(x,y,z).
\end{eqnarray*}
 
We use a red-black Gauss-Seidel with overrelaxation parameter. An overrelaxtion parameter of $\omega=1.15$ was reported to lead to an optimal convergence rate. The iteration scheme for the approximation after $m+1$ iterations on grid $\Omega_h$ is

\begin{eqnarray*}
v^{m+1}_h(x,y,z) &=& \frac{1}{6} [h^2\;f_h(x,y,z)\\
&&+\;u_h^m(x+h,y,z)+u_h^m(x-h,y,z)\\
&&+\; u_h^m(x,y+h,z)+u_h^m(x,y-h,z)\\
&&+\; u_h^m(x,y,z+h)+u_h^m(x,y,z-h)]\\
u^{m+1}_h(x,y,z) &\leftarrow& u^m_h(x,y,z)+\omega[v^{m+1}_h(x,y,z)-u^m_h(x,y,z)]
\end{eqnarray*}

where the arrow indicates straight overwriting of the last approximation $u^m_h$ with the new one $u^{m+1}_h$ as it becomes available. All 'red' components on the lefthand side are independent of the 'black' components on the righthand side and can be updated in parallel.

In order overlap computation and communication we relax first the grid points at the boundaries $\partial\Omega_h$ followed by halo swaps via non-blocking MPI calls and compute then the inner grid points $\Omega_h\setminus\partial\Omega_h$.

\section{Coarse-grid communication and solution strategies}

Several coarse-grid communication strategies are suggested in the literature which are supposed to alleviate the communication overhead that emerges when the decreasing ratio of grid points and halo sites starts to have a deteriorating effect on the algorithm. 

Probably the easiest way to deal with this issue is to truncate the multigrid hierarchy so that there are always plenty of grid points on each process. This can have a detrimental effect on the overall-convergence of the algorithm as the lowest grid level is probably still too fine to achieve rapid convergence.

Another strategy is to use a fast, parallel iterative solver on the coarstest grid level like conjugate-gradient solver for instance. 

It is also possible to gather grid points that are initially allocated on different processes, solve only on a subset of processes with the others idling and scatter afterwards. Alternatively the data from all processes can be distributed over the entire topology by means of an MPI All-to-All call. The identical calculation can then be performed simulataneously on all processes, making further communication after each iteration step unnecessary. The grid can be also further coarsed down to the level where a direct solutin is possible.


\section{Multigrid Cycle}

In the following paragraph we describe a multigrid V-cycle in pseudocode for one iteration of the approximate solution $v_k^m\rightarrow v_k^{m+1}$. For convenience we drop the iteration index $m$.

The multigrid algorithm can be very efficently formulated in terms of a nested procedure. This is because solving $A_k u_k = f_k$ with arbitrary intial guess $v_k$ is equivalent to solving the defect equation $A_{k-1} e_{k-1} = r_{k-1}$ on the next coarser level with $e_{k-1}=0$ as initial guess. This allows mapping onto the original problem with $e_{k-1}$ and $r_{k-1}$ taking now the roles of a solution vector and a right-side vector, respectively. It is thus convenient to re-label the error and residual according to $e_{k-1}\rightarrow v_{k-1}$ and $r_{k-1}\rightarrow f_{k-1}$. This, however, does not affect the meaning of these quantities, in particular that from the first restriction on the solution vector is actually an error and the right-side vector is a residual. 
 
\begin{algorithm}
\caption{Multigrid V-cycle for one iteration step $v_k^m\rightarrow v_k^{m+1}$}
\begin{algorithmic} 
\STATE{}
\STATE{1. Presmoothing:}
\FOR{$\nu_1$ steps}
\STATE{$\triangleright$ Request halo data via non-blocking MPI-call}
\STATE{$\triangleright$ Relax $A_k v_k = f_k$ on $\partial\Omega_k$ with initial guess $v_k$}
\STATE{$\triangleright$ Send halo data via non-blocking MPI-call}
\STATE{$\triangleright$ Relax $A_k v_k = f_k$ on $\Omega_k\setminus \partial\Omega_k$  with initial guess $v_k$}
\ENDFOR
\STATE{}
\STATE{2. Compute residual:}
\STATE{$\triangleright$ Compute $r_k=f_k-A_k v_k \rightarrow f_k$}
\STATE{}
\STATE{3. Restrict residual:}
\STATE{$\triangleright$ Compute $f_{k-1}=R^{k-1}_k f_k$}
\STATE{}
\STATE{4. Coarse solve:}
\IF{$k-1>1$}
\FOR{$\nu_1$ steps}
\STATE{$\triangleright$ Request halo data via non-blocking MPI-call}
\STATE{$\triangleright$ Relax $A_{k-1}v_{k-1}=f_{k-1}$ on $\partial\Omega_{k-1}$ with initial guess $v_{k-1}=0$ }
\STATE{$\triangleright$ Send halo data via non-blocking MPI-call}
\STATE{$\triangleright$ Relax $A_{k-1}v_{k-1}=f_{k-1}$ on $\Omega_{k-1}\setminus \partial\Omega_{k-1}$ with initial guess $v_{k-1}=0$ }
\ENDFOR
\STATE{$\triangleright$ Set $k-1 \rightarrow k$; goto step 2.}
\ENDIF
\STATE{}
\IF{$k-1=1$}
\STATE{$\triangleright$ Scatter all data via MPI All-to-All call}
\STATE{$\triangleright$ On each process solve $A_1 v_1=f_1$ on entire domain $\Omega_1$, either exactly or using a fast iterative solver with inital guess $v_1=0$} 
\ENDIF
\STATE{}
\STATE{5. Apply coarse grid correction to inital guess:}
\STATE{$\triangleright$ Compute $\hat{v}_k=v_k + I_{k-1}^k v_{k-1}$ on $\Omega_k$}
\STATE{}
\STATE{6. Postsmoothing:}
\FOR{$\nu_2$ steps}
\STATE{$\triangleright$ Request halo data via non-blocking MPI-call}
\STATE{$\triangleright$ Relax $A_k \hat{v}_k= f_k$ on $\partial\Omega_k \rightarrow v_k$ new approximate solution}
\STATE{$\triangleright$ Send halo data via non-blocking MPI-call}
\STATE{$\triangleright$ Relax $A_k \hat{v}_k= f_k$ on $\Omega_k\setminus\partial\Omega_k \rightarrow v_k$ new approximate solution}
\ENDFOR
\STATE{}
\end{algorithmic}
\end{algorithm}


%\STATE <text>
%\IF{<condition>} \STATE{<text>} \ENDIF
%\FOR{<condition>} \STATE{<text>} \ENDFOR
%\FOR{<condition> \TO <condition> } \STATE{<text>} \ENDFOR
%\FORALL{<condition>} \STATE{<text>} \ENDFOR
%\WHILE{<condition>} \STATE{<text>} \ENDWHILE
%\REPEAT \STATE{<text>} \UNTIL{<condition>}
%\LOOP \STATE{<text>} \ENDLOOP
%\REQUIRE <text>
%\ENSURE <text>
%\RETURN <text>
%\PRINT <text>
%\COMMENT{<text>}
%\AND, \OR, \XOR, \NOT, \TO, \TRUE, \FALSE

\end{document}

