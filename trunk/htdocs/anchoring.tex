%%%%%%%%%%%%%%%%%%%%%%%%%%%%%%%%%%%%%%%%%%%%%%%%%%%%%%%%%%%%%%%%%%%%%%%%%%%%%
%
%  anchoring.tex
%
%  $Id$
%
%  Edinburgh Soft Matter and Statistical Physics Group and
%  Edinburgh Parallel Computing Centre
%
%  Kevin Stratford (kevin@epcc.ed.ac.uk)
%  (c) 2011 The University of Edinburgh
%
%%%%%%%%%%%%%%%%%%%%%%%%%%%%%%%%%%%%%%%%%%%%%%%%%%%%%%%%%%%%%%%%%%%%%%%%%%%%%

\section{Cholesteric Anchoring}

We have a fluid free energy density
\begin{equation}
f = 
{\textstyle\frac{1}{2}} \kappa_0 (\partial_\beta Q_{\alpha\beta})^2
+ {\textstyle\frac{1}{2}}
 \kappa_1 (\epsilon_{\alpha\gamma\sigma} \partial_\gamma
Q_{\sigma\beta} + 2q_0 Q_{\alpha\beta})^2
\end{equation}
where we have ignored the bulk terms for the time being, but we retain
two elastic constants $\kappa_0$ and $\kappa_1$ in the distortion term.
The cholesteric pitch $p = 2\pi/q_0$.

There is also a surface term (an area density)
\begin{equation}
f_s = {\textstyle\frac{1}{2}} w_1 (Q_{\alpha\beta} - Q_{\alpha\beta}^0)^2
\end{equation}
where $Q^0_{\alpha\beta}$ is the preferred order parameter configuration
at the surface (in the case of normal anchoring or fixed planar anchoring).
For degenerate planar anchoring,
following Fournier and Galatola \cite{fournier2005}, we have
\begin{equation}
f_s = {\textstyle\frac{1}{2}} w_1 (\tilde{Q}_{\alpha\beta}
    - \tilde{Q}^\perp_{\alpha\beta})^2
    + {\textstyle\frac{1}{2}} w_2 (\tilde{Q}_{\alpha\beta}
                      \tilde{Q}_{\alpha\beta} - S^2_0)^2
\end{equation}
where $\tilde{Q}_{\alpha\beta} + (1/3)S_0 \delta_{\alpha\beta}$, $S_0$ is
is a fixed surface amplitude; to obtain a prefered degenerate
configuration we take the fluid order parameter $Q_{\alpha\beta}$ and
project $\tilde{Q}^\perp_{\alpha\beta} = P_{\alpha\gamma}
\tilde{Q}_{\gamma\sigma}
P_{\sigma\beta}$, where $P_{\alpha\beta} = \delta_{\alpha\beta}
- n_\alpha n_\beta$ and $n_\alpha$ is the local unit surface normal.

The boundary condition we wish to apply are the Euler-Lagrange equations. 
They are at the surface of the particle
\begin{equation}
n_\gamma \frac{\partial f}{\partial Q_{\alpha\beta,\gamma}} = 0
\end{equation}
where $Q_{\alpha\beta,\gamma} = \partial_\gamma Q_{\alpha\beta}$. Here,
$n_\gamma$ is the outward unit normal at the surface (pointing into the
fluid) \cite{skarabot}. Note that compared to the regular minimization as  
explained in \cite{wright} this is a surface integral term that would 
usually vanish in the bulk.
With the
addition of the surface term we obtain
\begin{equation}
n_\gamma \frac{\partial f}{\partial Q_{\alpha\beta,\gamma}}
+ \frac{\partial f_s}{\partial Q_{\alpha\beta}} = 0.
\end{equation}
The derivative of the distortion free energy with respect to
$Q_{\alpha\beta,\gamma}$ gives
\begin{equation}
\kappa_0 n_\beta \partial_\gamma Q_{\alpha\gamma}
+ \kappa_1 n_\gamma
(\partial_\gamma Q_{\alpha\beta} - \partial_\alpha Q_{\gamma\beta})
- 2\kappa_1 q_0 n_\gamma \epsilon_{\alpha\gamma\sigma} Q_{\sigma\beta}.
\end{equation}
However, if we want a symmetric form (the derivative with respect to
$Q_{\beta\alpha,\gamma}$ is just as good) we can write this as
\begin{eqnarray}
{\textstyle\frac{1}{2}} \kappa_0 (n_\alpha \partial_\gamma Q_{\beta\gamma}
+ n_\beta \partial_\gamma Q_{\alpha\gamma})
+ \kappa_1 n_\gamma \partial_\gamma Q_{\alpha\beta}
- {\textstyle\frac{1}{2}} \kappa_1 n_\gamma ( \partial_\alpha Q_{\gamma\beta}
+ \partial_\beta Q_{\gamma\alpha})
\nonumber
\\
- \kappa_1 q_0 n_\gamma (\epsilon_{\alpha\gamma\sigma} Q_{\sigma\beta}
+ \epsilon_{\beta\gamma\sigma}Q_{\sigma\alpha}).
\end{eqnarray}
If we add the derivative of the surface free energy,
where we assume that the preferred orientation $Q_{\alpha\beta}^0$ is
independent of $Q_{\alpha\beta}$, we get a full boundary condition
for the gradient of the tensor order parameter at the solid-fluid boundary:
\begin{eqnarray}
{\textstyle\frac{1}{2}} \kappa_0 (n_\alpha \partial_\gamma Q_{\beta\gamma}
+ n_\beta \partial_\gamma Q_{\alpha\gamma})
+ \kappa_1 n_\gamma \partial_\gamma Q_{\alpha\beta}
- {\textstyle\frac{1}{2}} \kappa_1 n_\gamma ( \partial_\alpha Q_{\gamma\beta}
+ \partial_\beta Q_{\gamma\alpha})
\nonumber
\\
- \kappa_1 q_0 n_\gamma (\epsilon_{\alpha\gamma\sigma} Q_{\sigma\beta}
+ \epsilon_{\beta\gamma\sigma}Q_{\sigma\alpha})
- w_1(Q_{\alpha\beta} - Q_{\alpha\beta}^0) = 0.
\label{cholesteric_bc}
\end{eqnarray}
Note that for planar generate anchoring, the final term (the `surface
molecular field') is replaced by
\begin{equation}
- w_1 (\tilde{Q}_{\alpha\beta} - \tilde{Q}^\perp_{\alpha\beta})
- 2w_2 (\tilde{Q}_{\gamma\sigma} \tilde{Q}_{\gamma\sigma} - S^2_0)
  \tilde{Q}_{\alpha\beta}. 
\end{equation}

\subsection{Discrete implementation}

In three dimensions, the boundary condition Eq.~\ref{cholesteric_bc}
contains (potentially) 15 unknown derivatives
$\partial_\gamma Q_{\alpha\beta}$
corresponding to the 5 independent elements of the order parameter
tensor $Q_{xx}$, $Q_{xy}$, $Q_{xz}$, $Q_{yy}$, and $Q_{yz}$. At a
flat surface with, e.g., $n = (1, 0, 0)$, this is reduced to five unknown
gradients $\partial_x Q_{\alpha\beta}$ at the
boundary if we assume the tangential gradients $\partial_y Q_{\alpha\beta}$
and $\partial_z Q_{\alpha\beta}$ can be approximated
using the standard differencing method involving only fluid values of
$Q_{\alpha\beta}$. We proceed by
computing the constant terms
$$
- \kappa_1 q_0 n_\gamma (\epsilon_{\alpha\gamma\sigma} Q_{\sigma\beta}
+ \epsilon_{\beta\gamma\sigma}Q_{\sigma\alpha})
- w(Q_{\alpha\beta} - Q_{\alpha\beta}^0)
$$
using $Q_{\alpha\beta}$ from the adjacent fluid site, and an estimate
of $Q^0_{\alpha\beta}$ for the appropriate anchoring type. To these
constant terms are added the tangential gradients.
The gradients at the surface are then computed by solving a 5x5
linear algebra problem for $\partial_x Q_{\alpha\beta}$. This
allows the full gradient at the adjacent fluid site to be constructed.

At concave edges or corners, where it is not possible to compute the
tangential gradients from the usual stencil as for a flat interface,
a different approach is required. We note than an attempt to solve a
10x10 or 15x15 linear algebra problem for the full set of unknown
gradients has proven unreliable: in the case where $n_\gamma$ is
normal to the coordinate directions there is simply no solution
available. Instead we adopt an iterative approach.

Consider the case of an edge where the solid neighbours
are in the $x$ and $y$ directions.  We first make an approximation
to the tangential gradients in the $x$ and $y$ directions by using
a one-sided derivative from the fluid points available. This
tangential approximation in $y$ is then used to find a first
estimate of the gradients $\partial_x Q_{\alpha\beta}$ using the
procedure described above. The tangential estimate in $x$ is then
used to obtain $\partial_y Q_{\alpha\beta}$. These surface gradients
are then used to improve the estimate of the tangential gradients,
and the process is iterated.

{\small
\begin{table}
\begin{center}

\begin{tabular}{|c|ccccc|}

\hline
&
$Q_{xx,x}$ & $Q_{xy,x}$ & $Q_{xz,x}$ & $Q_{yy,x}$ & $Q_{yz,x}$ \\
\hline
$Q_{xx}$ &
$\kappa_0 n_x$ & $-\kappa_1 n_y$ & $-\kappa_1 n_z$ & & \\
$Q_{xy}$ &
$\kappa_0 n_y$ & $\kappa' n_x$ & & $-\kappa_1 n_y$  & $-\kappa_1 n_z$  \\
$Q_{xz}$ &
$\kappa' n_z$ & & $\kappa' n_x$ & $\kappa_1 n_z$ & $-\kappa_1 n_y$ \\
$Q_{yy}$ &
 & $\kappa_0 n_y$ & & $\kappa_1 n_x$ & \\
$Q_{yz}$ &
 & $\kappa_0 n_z$ & $\kappa_0 n_y$ & & $2\kappa_1 n_x$ \\
\hline
\hline
&
$Q_{xx,y}$ & $Q_{xy,y}$ & $Q_{xz,y}$ & $Q_{yy,y}$ & $Q_{yz,y}$ \\
\hline
$Q_{xx}$ &
$\kappa_1 n_y$ & $\kappa_0 n_x$ & & & \\
$Q_{xy}$ &
$-\kappa_1 n_x$ & $\kappa' n_y$ & $-\kappa_1 n_z$ & $\kappa_0 n_x$ & \\
$Q_{xz}$ &
 & $\kappa_0 n_z$ & $2\kappa_1 n_y$ & & $\kappa_0 n_x$ \\
$Q_{yy}$ &
 & $-\kappa_1 n_x$ & & $\kappa_0 n_y$ & $-\kappa_1 n_z$\\
$Q_{yz}$ &
$\kappa_1 n_z$ & & $-\kappa_1 n_x$ & $\kappa' n_z$ & $\kappa' n_y$\\
\hline
\hline
&
$Q_{xx,z}$ & $Q_{xy,z}$ & $Q_{xz,z}$ & $Q_{yy,z}$ & $Q_{yz,z}$ \\
\hline
$Q_{xx}$ &
$\kappa_1 n_z$ & & $\kappa_0 n_x$ & & \\
$Q_{xy}$ &
 & $2\kappa_1 n_z$ & $\kappa_0 n_y$ & & $\kappa_0 n_x$ \\
$Q_{xz}$ &
$-\kappa' n_x$ & $-\kappa_1 n_y$ & $\kappa' n_z$ & $-\kappa_0 n_x$ & \\
$Q_{yy}$ &
 & & & $\kappa_1 n_z$ & $\kappa_0 n_y$  \\
$Q_{yz}$ &
$-\kappa_0 n_y$ & $-\kappa_1 n_x$ & & $-\kappa' n_y$ & $\kappa' n_z$ \\
\hline
\end{tabular}
\end{center}
\caption{Coefficients of the various derivatives of the order parameter
tensor appearing in five equations for the independent elements of the
order parameter. Note $\kappa_0 + \kappa_1 = \kappa'$ and all the
coefficients have been multiplied by a factor of 2.}
\label{tab:cholesteric_bcs}
\end{table} 
}
 
