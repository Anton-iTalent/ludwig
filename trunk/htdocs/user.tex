%%%%%%%%%%%%%%%%%%%%%%%%%%%%%%%%%%%%%%%%%%%%%%%%%%%%%%%%%%%%%%%%%%%%%%%%%%%%%
%
%  user.tex
%
%  The 'user' section which holds an overview, and discusses
%  compilation and running.
%
%  $Id$
%
%  Edinburgh Soft Matter and Statistical Physics Group and
%  Edinburgh Parallel Computing Centre
%
%  Kevin Stratford (kevin@epcc.ed.ac.uk)
%  (c) 2010 The University of Edinburgh
%
%%%%%%%%%%%%%%%%%%%%%%%%%%%%%%%%%%%%%%%%%%%%%%%%%%%%%%%%%%%%%%%%%%%%%%%%%%%%%

\section{Introduction}

\subsection{Overview}

We aim to provide a robust and portable code, written in C, which
can be used to perform serial and scalable parallel simulations of
complex fluid systems based around hydrodynamics via the Lattice
Boltzmann method. Complex fluids are introduced via a free energy,
of which a number of different examples are available. The preferred
method of dealing with the corresponding order parameter equations
is now by using finite difference. However, for the case of a binary
fluid, the `full LB' approach, using two distributions, is retained
as an option for the time being.

\subsection{Availability}

Under the auspices of the CCPs (Collaborative Computational Projects),
and in particular CCP5, we have a CCPForge project account at

\texttt{http://ccpforge.cse.rl.ac.uk/}

This provides controlled access to known persons and uses version
control via SVN, and includes bug
tracking, and so on. The CCPForge SVN is not available to the public
at the present time. You need to create an account at CCPForge, and
let me or Juho know the username so you can be added to the access list.

To download the entire code, you need
\begin{verbatim}
svn checkout --username <user> http://ccpforge.cse.rl.ac.uk/svn/ludwig
\end{verbatim}
where \texttt{<user>} is replaced by your CCPForge user name.
Note that this will download all the existing tags and branches
into a new directory named \texttt{ludwig}, as well as the main code!
On success, you should see three directories
\begin{verbatim}
$ ls ludwig
branches  tags      trunk
\end{verbatim}
The main code is in the \texttt{trunk} subdirectory. This documentation is
found in the file \texttt{trunk/htdocs/ludwig.tex}, unit tests are in
the directory \texttt{trunk/tests}, and the utilities are in
\texttt{trunk/util}, and so on.

If you do not want all the tags and branches (which take up a certain amount
of space), you can specify that you
want only the trunk, e.g.,
\begin{verbatim}
svn co --username <user> http://ccpforge.cse.rl.ac.uk/svn/ludwig/trunk
\end{verbatim}
in which case you will get a new directory called \texttt{trunk} containing
the source files. All the instructions below refer to paths relative to
this trunk directory.

If you have problems, it may be worth while looking at the issue
tracker on the project web page. It would be useful if all issues
could be reported via this mechanism. If you e-mail me, I will just
insert your e-mail into the issue tracker!

\subsection{Compilation}

Locate your local C compiler. The following example uses \texttt{gcc}
in serial, and \texttt{mpicc} in parallel. For MPI in particular,
local details may vary.

The other thing required by the Makefile is the utility
\texttt{svnversion}, which allows the compile-time SVN version number to
be included in the executable (and hence the output of the program). This
is useful to remember which version you were using when looking at
results at a later date. If
\texttt{svnversion} is not available (and it
should be if you have downloaded the code via SVN), you may get an
error message, but the code should still compile.

\subsubsection{Serial}

First, compile the MPI stub library in the \texttt{mpi\_s}
directory. Do this by editing the Makefile, and checking the compiler
is appropriate. You should be able to build the library and run the
tests using, e.g.,:

\begin{verbatim}
$ make libc
gcc -Wall -I.  -c mpi_serial.c
ar -cru libmpi.a mpi_serial.o
$ make testc
gcc -Wall   mpi_tests.c -L. -lmpi
./a.out
Running mpi_s tests...
Finished mpi_s tests ok.
\end{verbatim}

Now compile the main code in the \texttt{src} directory. Again
edit the Makefile to check the compiler. At this stage, you also
need to choose the LB model using exactly one of the preprocessor
options for D2Q9 (fluid only problems), D3Q15, or D3Q19. Now type,
e.g., using D3Q19:

\begin{verbatim}
$ make serial
gcc  -D SVN_REVISION='"'`svnversion`'"' -c svn.c
gcc  -D_D3Q19_ -O3 -Wall -I. -I../mpi_s -c d2q9.c
gcc  -D_D3Q19_ -O3 -Wall -I. -I../mpi_s -c d3q15.c
gcc  -D_D3Q19_ -O3 -Wall -I. -I../mpi_s -c d3q19.c
gcc  -D_D3Q19_ -O3 -Wall -I. -I../mpi_s -c model.c
...
\end{verbatim}

This should provide an executable \texttt{Ludwig.exe} which is linked
against the MPI stub library.

\subsubsection{Parallel}

Here, you do not need to compile the stub library. Simply compile
the main code with, e.g.,:
\begin{verbatim}
$ make mpi
mpicc -D_D3Q19_ -O3 -Wall -I. -c d2q9.c
mpicc -D_D3Q19_ -O3 -Wall -I. -c d3q15.c
mpicc -D_D3Q19_ -O3 -Wall -I. -c d3q19.c
mpicc -D_D3Q19_ -O3 -Wall -I. -c model.c
...
\end{verbatim}

Invoke the executable with the MPI launcher in the usual way for your
system, e.g., \texttt{mpirun -np 8 ./Ludwig.exe} for 8 processors,
and so on.

\subsubsection{ANSI C}

The code is ANSI C (1989), with the single exception of calls to the
library function \texttt{erfc()} in the Ewald sum code. This
can sometimes cause a problem, which can usually be solved with
a compiler-specific work-around.

\subsubsection{C assertions}

The code makes quite a lot of use of standard C assertions, which
are useful to prevent errors. They do result in a considerably
slower execution in some instances, so production runs should
switch off the assertions with the standard \texttt{NDEBUG}
preprocessor flag.


\section{User Input}

The executable will look for an input file at run time.
By default, this file is named \texttt{input} and should be in the
current working directory. The alternative is to specify the name
of the file explicitly on the command line:
\begin{verbatim}
$ ./Ludwig.exe input_file
\end{verbatim}
A reference input file \texttt{input.ref} is provided as a template.

The contents of the input file are made up of \textit{key value}
pairs with control the run-time behaviour. Blank lines and lines
begining with \# are ignored as comments. The following describes
the effect of various keys. Most keys have a default value which
will be used by the code if the corresponding key is not present
(or commented out) in the input file.

\subsection{Setting up the input}

Decide what free energy you want and check the relevant sections
of this dcoumentation. If you want a simple fluid only, the free
energy should be set to \texttt{none}.

You should set at least the basic fluid parameters, the system
size, and number of steps to be run.

\textbf{IMPORTANT:} Although many incompatible choices of input
parameters are trapped at run time, you must be careful in how
the input file is set out. If you mistakenly set or unset some
options in the input, erroneous results will surely result...


\subsection{Basic run parameters}

\inputkey{N\_cycles}

The number of lattice Boltzmann time steps to execute. Default
value: 0.

\inputkey{N\_start}

By default the code will start from time step $t = 0$. If you 
wish to restart from a previously saved configuration, set
the appropriate value for \texttt{N\_start}. The code will
execute \texttt{N\_cycles} steps starting from this point.

\inputkey{size}

Controls the number of lattice points in the $(x, y, z)$ directions,
respectively. Default value: 64\_64\_64. If a two-dimensional system
is required, the $z$-direction can be set to 1.

\inputkey{grid}

In parallel, this sets the processor Cartesian decomposition, i.e.,
the number of processors in each dimension of the Cartesian communicator.
The default value may be implementation dependent as it is that returned
by \texttt{MPI\_Dims\_create()}. For
example:
\begin{verbatim}
size 64_64_64
grid 4_2_1
\end{verbatim}
gives a total (physical) system size of 64 lattice sites in each
direction, decomposed across 4 processors in the x-direction, 2 in
the y-direction and 1 (no decomposition) in the z-direction. The
total number of processors must therefore be 8. The local domain
size per processor is then 16x32x64.


\subsection{Fluid properties}

\inputkey{free\_energy}

This sets the free energy in use, and hence has many consequences.
The default value is \texttt{none}, i.e., use a simple fluid only.
The available choices are for symmetric binary fluids, Brazovskii
smectics, polar active gels, and liquid crystals. The surfactant
model is currently being rebuilt.

\inputkey{viscosity}

Sets the LB fluid shear viscosity $\eta$ (and related relaxation time).
Safe values are roughly $0.2 > \eta > 0.0001 $. Default value is 1/6
(i.e., relaxation time equal to unity).

\inputkey{viscosity\_bulk}

Sets the LB bulk viscosity $\zeta$ (and related relaxation time).
This can be useful if you are concerned with incompressiblity
violations, in which case $\zeta >> \eta$ can be useful. The
default value is $\zeta = \eta$.

\inputkey{isothermal\_fluctuations}

This switches on the fluctuationing hydrodynamics. Default is off.

\inputkey{temperature}

The LB 'temperature' using fluctuating hydrodynamics. Safe values
are $0.0001 > kT > 0$. 

\inputkey{ghost\_modes}

Allows you to switch the ghost modes off in the collision stage.
Default is on.


\subsection{Symmetric binary fluid}

\inputkey{free\_energy symmetric}

The following applies for the binary fluid problem with
compositional order parameter $\phi$ and free energy
(excluding the term in the density $\rho$):
\begin{equation}
 F[\phi] = 
\int dr \left(
{\textstyle \frac{1}{2}}A\phi^2
+ {\textstyle \frac{1}{4}}B\phi^4
+ {\textstyle \frac{1}{2}}\kappa (\mathbf{\nabla}\phi)^2 \right).
\end{equation}

This is described in some detail
by Kendon et al \cite{viv}. The first two terms represent the bulk
contribution, whereas the term in $\kappa$ penalises curvature in
the interface.

\inputkey{A} The parameter $A$. Note $A < 0$.

\inputkey{B} The parameter $B$. Note $B = -A$ for common usage,
although this is not enforced.

\inputkey{K} The parameter $\kappa$, which is positive.

The order parameter evolution is determined by a Cahn-Hilliard equation
with mobility set by

\inputkey{mobility} Sets mobility $M$ (uniform in space).

The following parameters control the initialisation of the order
parameter.

\inputkey{phi\_initialisation}

Determines how the compositional order parameter is initialised
at the start of the run. If set to \texttt{spinodal} the value
is set to $\phi_0$ as set above plus or minus a random noise, the
magnitude of which is set by the value of the \texttt{noise} key.
If set to \texttt{block} a one-dimensional profile is set up in
the $z$-direction representing two blocks of fluid with
$\phi = \pm 1$. The two interfaces are set at $z = L_z/4$ and
$z = 3L_z/4$ with the equilibrium $\tanh(z/\xi_0)$ profile having 
appropriate width. The $\phi = -1$ section is in the middle.

\inputkey{phi0}

The mean compositional order parameter roughly $-0.5 < \phi_0 < 0.5$.
The default value is zero, i.e., a symmetric 50:50 mixture by volume.

\inputkey{noise}

The magnitude of the initial fluctuations in $\phi$ used to
initiate spinodal decomposition.

\subsubsection{Binary fluid using two distributions}

\inputkey{symmetric\_lb}

This is the special case where the composition is represented
by a second LB distribution, and an appropriate lattice kinetic
equation approximating the Cahn-Hilliard equation is solved.
In this case the above parameters have the same meaning.


\subsection{Brazovskii}

\inputkey{free\_energy brazovskii}

This is similar to the symmetric free energy, but with one extra term
in a higher derivative of $\phi$.
\begin{equation}
 F[\phi] = 
\int dr \left(
{\textstyle \frac{1}{2}}A\phi^2
+ {\textstyle \frac{1}{4}}B\phi^4
+ {\textstyle \frac{1}{2}}\kappa (\mathbf{\nabla}\phi)^2
+ {\textstyle \frac{1}{2}} C (\nabla^2 \phi)^2 \right).
\end{equation}

The parameters now include $C$. For $A<0$, phase separation occurs
with a result depending on $\kappa$: ones get two symmetric phases
for $\kappa >0$ (cf. symmetric) or a lamellar phase for
$\kappa < 0$.

\inputkey{A} Bulk parameter $A < 0$.

\inputkey{B} Bulk parameter $B = -A$.

\inputkey{K} Negative for lamellar phase.

\inputkey{C} Positive.

Other parameters, including the mobility, are set as for the
symmetric free energy (see above).


\subsection{Surfactant in binary fluid}

\inputkey{free\_energy surfactant}

Currently under reconstruction.

This is for use with a two order parameter model the first of
which is the composition, as for the symmetric free energy,
while the second $\psi$ represents surfactant concentration.
The free energy density is made up of a number of terms:

\begin{equation}
{\textstyle \frac{1}{2}}A\phi^2
+ {\textstyle \frac{1}{4}}B\phi^4
+ {\textstyle \frac{1}{2}}\kappa (\mathbf{\nabla}\phi)^2
\end{equation}
is the standard symmetric term related to the composition.
These parameters are represented in the input as

\inputkey{surf\_A}
\inputkey{surf\_B}
\inputkey{surf\_kappa}

\begin{equation}
D \left(\psi \ln\psi + (1 - \psi) \ln(1-\psi)\right)
\end{equation}
represents the energy of surfactant in the bulk phase with surfactant
concentration $0 < \psi < 1$ with a single parameter $D$;
\begin{equation}
-{\textstyle \frac{1}{2}} \epsilon \psi (\mathbf{\nabla} \phi)^2
-{\textstyle \frac{1}{2}} \beta \psi^2  (\mathbf{\nabla} \phi)^2
\end{equation}
where the term in $\epsilon$ represents the energy reduction by
adsorbing surfactant at the interface and the term in $\beta$
provides an additional favourable term modelling the fact that
the molecules like to line up; finally, there is an extra term
\begin{equation}
+{\textstyle \frac{1}{2}} W \psi \phi^2
\end{equation}
which is included to stabilise the interface \cite{theissengompper}.

The corresponding keys for the input file are:

\inputkey{D} is the bulk surfactant contribution parameter

\inputkey{epsilon} is the parameter $\epsilon$

\inputkey{W} is the parameter $W$

\inputkey{beta} is the value of the parameter $\beta$.

\inputkey{phi\_b} the initial, uniform, background surfactant
concentration $\psi_b$.

\inputkey{mobility\_psi} Sets the (uniform) mobility for the surfactant
as appropriate for the Cahn-Hilliard equation.


\subsection{Polar Active Gel}

\inputkey{free\_energy polar\_active}

This models a polar active gel with vector order parameter $P_\alpha$.
The free energy density is:


The corresponding input parameters are:

\inputkey{polar\_active\_a}

\inputkey{polar\_active\_b}

\inputkey{polar\_active\_k}

\inputkey{polar\_active\_klc}

\inputkey{polar\_active\_zeta}

\inputkey{polar\_active\_lambda}

\inputkey{leslie\_ericksen\_gamma}

\inputkey{leslie\_ericksen\_swim}


\subsection{Liquid Crystal}

\inputkey{free\_energy lc\_blue\_phase}

Here, we have a tensor order parameter $Q_{\alpha\beta}$.
The free energy density is
\begin{eqnarray}
f = {\textstyle\frac{1}{2}}A_0(1 - \gamma/3)Q^2_{\alpha\beta}
  - {\textstyle\frac{1}{3}}A_0 \gamma
Q_{\alpha\beta}Q_{\beta\delta}Q_{\delta\alpha}
 + {\textstyle\frac{1}{4}}A_0 \gamma (Q^2_{\alpha\beta})^2
\nonumber
\\
+ {\textstyle\frac{1}{2}} \Big(
\kappa_0 (\epsilon_{\alpha\delta\sigma} \partial_\delta Q_{\sigma\beta} +
2q_0 Q_{\alpha\beta})^2 + \kappa_1(\partial_\alpha Q_{\alpha\beta})^2 \Big)
\end{eqnarray}

The corresponding input parameters are:

\inputkey{lc\_a0} The bulk free energy parameter $A_0$

\inputkey{lc\_gamma} The bulk free energy parameter $\gamma$

\inputkey{lc\_q0} The pitch wavenumber $q_0 = 2\pi / p$, where p
is the pitch.

\inputkey{lc\_kappa0} Elastic constant $\kappa_0$ in distortion free energy

\inputkey{lc\_kappa1} Elastic constant $\kappa_1$ in distortion free energy

Note that the code currently enforces the `one elastic constant'
approximation ($\kappa_0 = \kappa_1$), so both these values must
be equal in the input. This constraint may be relaxed in the future.


\inputkey{lc\_Gamma} The rotational diffusion constant appearing in
the Beris Edwards equation (note \texttt{lc\_gamma} and \texttt{lc\_Gamma}
are different).

\inputkey{lc\_xi} Aspect ratio for rod lie molecules appearing in the
stress tensor.

\inputkey{lc\_active\_zeta} Apolar liquid crystal activity constant.

\subsubsection{Initial LC order parameter $Q_{\alpha\beta}$}

\inputkey{lc\_q\_initialisation}

There are a number of useful choices here. The key value
\texttt{twist} initialises a simple cholesteric with the
helical axis in the $z-$direction;  \texttt{o8m}
gives equilibrium BPI, \texttt{o2} gives BPII. The appropriate
choices for the other parameters must be made correspondingly.

An additional \texttt{lc\_q\_init\_amplitude} value sets the
scalar order parameter  `amplitude' in each case.

\subsubsection{The redshift}

\inputkey{lc\_init\_redshift}

Sets the initial value of the redshift parameter. For example,
for the \texttt{o8m} initialisation, the redshift can be set to 0.83,
and for the \texttt{o2} initialisation, the redshift can be set to 0.91.
The default redshift is unity.

\inputkey{lc\_redshift\_update [0|1]}

Switches on (default is off) the dynamic computation of the redshift
from the order parameter gradient terms in the free energy. The
caluclation is performed at every time step, and override any
initial value specified in the above. It is also robust to restarts.

\subsection{Colloidal particles}

The are a number of important options required for colloids. If
none of these keys are present, no colloids will be used.

\inputkey{colloid\_init}

This determines how the code initialises particles (or not). As stated, the
default is to have no particles (value \texttt{no\_colloids}). There
are two other options at the moment.

There are two options:

\inputkey{colloid\_init    from\_file}

requires a preprepared file of colloid information. To understand
how to create such a file in the correct format,
see the example program \texttt{colloid\_file.c} in the util
directory.

\inputkey{colloid\_init   random}

This may be useful to initialise a limited number of colloids at
low volume fraction. Note that there is no guarantee that the
particles do not overlap, in which case the initialisation will
terminate. The number of particles and their properties
are determined by the value of the keys described below.

\inputkey{colloid\_type}

This controls the the type of particle. Appropriate values are
\texttt{inactive}, giving a standard fully resolved LB colloid;
\texttt{active} switches on the correction to BBL appropriate
for active particles (expecting parameters $b_0$ and $b_1$ to be
set); \texttt{subgrid} gives unresolved particles (radius $< 1$
lattice unit) and no BBL.

If more than one particle is used, all positions are set at random.


\inputkey{colloid\_gravity  0.0\_0.0\_-0.0001}

Sets a body force on each particle. Useful, for example, for
sedimentation. For example, the above gives a constant force in
the negative $z$-direction on each particle.
The code automatically computes the compensating body force on fluid
nodes required to give no net change in the total momentum of the
system at each time step.

\subsubsection{Interactions and the cell list}

In each local domain, colloidal particles are stored in a cell list.
This is a common structure in molecular dynamics-like problems. Here,
it also serves as the basis for parallel communication of colloid
information, so there are a number of constraints which it is important
to understand. In particular, parallel communication imposes the
constraint that there be at least 2 cells in the local domain in
each coordinate direction.

The size of the cells is computed automatically by the code based on
the local domain size, particle size, and the cut-off distance of any
pairwise interactions that are relevant. This will usually mean trying
to maximise the number of cells (to reduce pairwise interaction
calculation). The user must set the minimum allowable cell width.

\inputkey{colloid\_cell\_min}

This sets the minimum cell list width. This key must be set and there
must be at least two cells per processor domain. If the minimum does
not capture the specified interactions, a fatal error will result.
For spherical particles, the minimum width will typically be
$2a_h + h_c$ where $h_c$ is an interaction cut-off distance.


In the case where very large particles are required, it can be useful
to switch of the cell list. This is done via

\inputkey{colloid\_cell\_list\_interactions no}

The constraint in this case is that $2a_h + h_c < L_{local}$, ie.,
there is effectively one cell in each local domain. This option
should not be required in normal circumstances.


\subsubsection{Initialisation from file}

A file containing initial state properties for one or more colloids
can be read at run file. This file must be of the correct from, but
can be ASCII or binary. To help to create such a file, an example
is given in \texttt{./util/colloid\_file.c}. This file will be
read correctly in both serial and in parallel. Colloid positions
should always be in terms of the global coordinates. By default,
the input expected is \texttt{config.cds.init.001-001}.

All colloid output is in serial (ie., a single file is produced)
at the moment. Again, it may be ASCII or binary.

\subsubsection{Initialisation at random}

\inputkey{colloid\_random\_no} Sets the total number of particles

\inputkey{colloid\_random\_a0}

\inputkey{colloid\_random\_ah}

\inputkey{colloid\_random\_dh}

The parameter $a_0$ the nominal radius of the particle on the lattice.
This is used to construct the links for BBL. The smallest acceptable LB
particle is $a_0 = 1.25$ (approximately). There are 'magic' values
including 1.25, 2.3, 3.7, 4.77 which minimise discretisation effects.
The hydrodynamic radius of the particles is $a_h$. For all
reasonable applications choose $a_h = a_0$. This is the radius for
all physical purposes. The \texttt{colloid\_random\_dh} parameter
specifies the smallest allowed surface-surface separation if two or
more particles (or walls) are required. This is useful to prevent
very close particles experiencing large and destabilising interactions
at start time.

The other state for the initial colloids can be set uniformly by
means of the keys:

\inputkey{colloid\_random\_v0} initial velocity vector

\inputkey{colloid\_random\_w0} initial angular velocity vector

\inputkey{colloid\_random\_s0} initial spin vector

\inputkey{colloid\_random\_m0} initial direction of motion vector for active
particles

\inputkey{colloid\_random\_b1} active parameter $b_1$

\inputkey{colloid\_random\_b2} active parameter $b_2$ (should be combined
with the \texttt{active} particle type key as discussed above).

There is one special case. If you want exactly one particle, you may
set its initial position with

\inputkey{colloid\_random\_r0} the initial position vector.

\subsubsection{Restarting at random!}

Be careful if you are restarting a simulation that used the random
initialisation at $t=0$. You must change to 

\texttt{colloid\_init  from\_file}

to avoid generating a completely new set of random positions
(which will not be consistent with the density and order parameter
fields at the restart time). This is almost certainly not what you
intended!

\subsubsection{Hydrodynamic radius}

There is an important issue to understand about the (nominally)
spherical particles in LB. The input radius $a_0$ is used to
construct the link boundary conditions. However, discretisation
errors in the boundary conditions means that the `real' particle
size may be different. This real radius is the hydrodynamic radius $a_h$.
As a further complication, the hydrodynamic radius can also
be a function of viscosity for a given input radius. Some values
are given in Table~X. This values are computed using a calibration
procedure described by Nguyen and Ladd \cite{nguyen2001}.

\begin{table}[h]
\begin{center}
\begin{tabular}{|c|c|c|c|}
\hline
$\eta$ & \multicolumn{3}{c|}{$a_0$}\\ \cline{2-4} 
       & 1.25 & 2.30 & 4.77 \\
\hline
1/6  & 1.09 (1.05) & 2.23 (2.20) & 4.76 \\
1/100  & 1.40 (1.34) & 2.50 (2.46) & 5.01 \\
1/1000  & 1.63 & 2.71 & 5.23 \\
\hline
\end{tabular}
\end{center}
\caption{A table of calibrated hydrodynamic radii as a function of input
radius and fluid viscosity $\eta$. These figures are for D3Q19 (and D3Q15
in brackets).}
\end{table}

If a calibration is required for a given radius and viscosity,
use the \texttt{calibration on} key pair in the input file. Some
examples are given in the \texttt{tests/calibration} directory.
An apppropriate
number of time steps must be used to allow the system to get over
the initial transient (we use the momentum diffusion time for the
system $L^2/\eta$) and to allow at least one
particle Stokes time ($a/U$). The input radius and hydrodynamic
radius should be set the same in the input file.

% PENDING Ginzburg d'Humieres correct to ghost mode relaxation times

\subsection{Boundary conditions}

Solid planar boundary walls can be added at the extremities of the
system (i.e., the volume of fluid is exactly that specified by
the \texttt{size} keyword). The wall can be in one, two or all three
coordinate directions, i.e., a 'slab' geometry, and 'duct' geometry
or a 'box' geometry, respectively.

The boundary walls are switched on via the keyword

\inputkey{boundary\_walls}

and with a complementatary specification of the key \texttt{periodicity}.

For example, walls at $x = x_{\min}$  and $x = x_{\max}$  are specified
by

\inputkey{boundary\_walls 1\_0\_0}

\inputkey{periodicity 0\_1\_1}

Both must be present. Note that there are a number of constraints on
valid choices, and the code will stop if there is an invalid choice, or
the two keys above do not match. Common choices are:
\texttt{boundary\_walls 0\_0\_1},
which gives periodic boundaries in the $x-$ and $y-$directions, with
walls at either edge in the $z-$direction; \texttt{0\_1\_1}
gives periodic conditions in the $x-$ direction only, while
\texttt{1\_1\_1} is a fully enclosed box.

In the case of the slab geometry with walls at $z = z_{\min}$ and
$z= z_{\max}$ (only),
two further values are used so that shear may be imparted to the system:

\inputkey{boundary\_speed\_bottom}

\inputkey{boundary\_speed\_top}

These values set the $x-$component of the wall velocity at
$z = z_{\min}$ and $z = z_{\max}$, respectively. Like all
velocities, these values must respect the low Mach number
constraint $u_x < c_s$.

There is a further option for use with the special case in which
top and bottom walls in the $z-direction$ are used to drive a shear
flow (typically with the top speed positive and the bottom speed
negative). The key

\inputkey{boundary\_shear\_init}

may be switched on (set to 1) to provide initialisation of the
LB distributions appropriate for the specified shear rate based
on a linear velocity profile in the $z-$direction. The default
is to initialise the fluid at rest.

\subsection{Lees-Edwards planes}

\inputkey{N\_LE\_plane}

sets the total number of Lees-Edwards planes. Default is zero.
The placing is as followings. The number of planes $n$ must
divide the lattice size in the $x$-direction to give an integer
$\delta x$. Planes are then placed at $\delta x / 2, 3\delta x/2, \ldots$.

\inputkey{LE\_plane\_vel}

sets the velocity of each plane relative to the lattice. All planes
have the same, constant, velocity.

\inputkey{LE\_init\_profile}

if set to 1, the fluid velocity is initialise to reflect a steady
state shear flow appropriate for the number of planes at the
given velocity. If set to zero, the fluid is initialised with
zero velocity.

Note that when the Lees Edwards boundaries are in place, output
files containing lattice quantities must be `unrolled' to remove
the time-dependent displacement of the planes. This is done using
the extraction utility (see section on parallel I/O).

The code works out the
current displacement of the planes by computing $U_{LE} t$, where
$t$ is the current time step. A shear run should then start from
$t = 0$, otherwise an offset is required.

\inputkey{LE\_time\_offset}

It is often convenient to run an equilibration with no shear, and
then to start an experiment after some number of steps. This
key allows you to offset the start of the Lees-Edwards motion.
It should then take the value of the start time (in time steps)
corresponding to the restart at the end of the equilibration
period. You also need to take this value into account when unrolling
the output.

\subsubsection{Lees-Edwards planes in parallel} 

There are a couple of additional constraints to use the Lees-Edwards
planes in parallel. In particular, the planes cannot fall at a
processor boundary in the $x$-direction. This means you should
arrange an integer number of planes per process in the $x$-direction.
(For example, use one plane per process; this will also ensure the number
of planes
still evenly divides the total system size.)
This will interleave the planes with the processor decomposition.
The $y$-direction and $z$-direction may be decomposed without
further constraint.

Note that this means a simulation with one plane will only work
if there is one process in the $x$ decomposition.

\subsection{Input/Output}

\subsubsection{Serial I/O and standard output}

A variety of information is printed to standard output during the
course of the simulation. This should allow the user to keep track
of basic progress in terms of integrated quantities, total mass,
total momentum and so on. Frequency of output is controlled by

\inputkey{freq\_statistics}

which is an integer. Note that these global statistics require
global communication, and so can affect performance in parallel.
It is recommended that the frequency is kept to a minimum for
large systems (e.g., every 500 or 1000 time steps). This
consideration is true for most types of output (see section
on parallel I/O).

\subsubsection{Configurations}

It may be appropriate to save entire model configurations from
time to time, and certainly when a restart is wanted. For each lattice
quantity, in addition to the data themselves, the program produces a
single meta-data file (ending in
\texttt{.meta}) which is used to describe these data. Colloid
data does not require a meta-data file at the moment.

\inputkey{freq\_config}

This is an integer and sets the frequency of full configuration dumps.
These may be large depending on the size of the system.
Full configurations should consist of the LB distributions,
order parameter (if present) and colloids (if present).


\inputkey{config\_at\_end [yes | no]} 

forces (or switches off) a configuration dump at the end of the run.
The default is to produce a configuration at the end of the run.

\subsubsection{Other lattice quantities}

A number of other lattice-based quantities are available as output.

\inputkey{freq\_measure}

sets the frequency of recording of colloid/order parameter output.
This includes the combined scalar order parameter and director
field output in the case of the liquid crystal free energy.

\inputkey{freq\_phi}

set frequency of order parameter output.

\inputkey{freq\_vel}

set frequency of velocity output. Note that all the hydrodynamic
quantities can be reconstructed from the distribution output, if
available. If only the velocity field is required, this method
allows a considerable saving in storage.


\subsubsection{I/O format}


\inputkey{vel\_format}

\inputkey{phi\_format}

May be either \texttt{ASCII} or \texttt{BINARY}. Default is \texttt{BINARY}.
Configuration I/O is always in binary. It is recommended to use
binary output for speed and disk space considerations.

\subsubsection{Parallel I/O}

In serial, each output quantity appears in one file (\texttt{phi-}
and so on). In parallel, this is also the case by default. However,
this does not scale, owing to the fact that each process has to
write to the same file in turn (ie., in serial). The solution is
to make the I/O parallel. This is done by splitting the domain into
different ``I/O grids'' each of which writes to a separate file.

The default I/O grid can be set using, e.g.,

\inputkey{default\_io\_grid 2\_2\_2}

meaning that the domain is decomposed into a 2x2x2 Cartesian
decomposition for I/O purposes. This will give rise to a total
of 8 files per I/O event with extensions \texttt{.008-001} to
\texttt{.008-008}. These files must be recombined if analysis
is required. However, restarting can be performed with the
parallel files.

Individual lattice quantities can use separate I/O grids. This
may be particularly useful for the distribution data, which are
much larger than the other lattice quantities. This means a
finer decomposition may be appropriate. Clearly, the size of the
I/O decomposition cannot exceed the processor decomposition.
See the advice in section \cite{section-advice} for guidelines.



\subsection{Miscellaneous}

\inputkey{random\_seed}

sets the random number generator seed.

\subsection{Dealing with the Parallel I/O files}

When running in parallel, output of lattice-based (and particle)
quantities takes place in parallel. As a consequence, the order
of the output in the files is not in the same order as would be
the case in serial. This means the parallel output must be
manipulated into the correct order before any analysis.

Parallel I/O subdivides the processors into one or more groups,
each of which write data to a separate file. Information on the
content of each of these files is contained in `meta-data' file
for each quantity, which are produced automatically at the start
of each run. The files are called, e.g., \texttt{phi.001-001.meta}.
If there are 8 I/O groups, the files will be named \texttt{008-001}
to \texttt{008-008} for the different groups.

The actual data is stored for each relevant time step as, e.g.,
\texttt{phi-000100.001-001} where the extension refers to the
I/O. To recombined a single, or multiple, data files into a
single file in the correct order, a utility program
\texttt{extract.c} is provided.

The executable takes two arguments, e.g.,

\inputkey{./extract phi.001-001.meta phi-000100.001-001}

where  the first is the first meta-data file in the relevant series,
and the second is the first data file for the relevant time step. The
program will combine the relevant quantities in the correct order
and produce a single new file \texttt{phi-000100} (in the above case).

\subsection{Restarting}

When restarting in parallel, the processor decomposition must be
preserved. An attempt to change the processor decomposition will
cause the configuration files to be read in the incorrect order,
and results will be wrong.

\section{Parallel Performance}

\subsection{General Comments}

The basic LB calculation should scale well in parallel, that is,
the time taken for fixed problem size will decrease linearly
with the number of MPI processes (strong scaling), or a larger
problem can be run on a proportionally larger number processes
in the same time (weak scaling). To ensure performance is retained
in parallel, it is useful to understand some basic considerations.
Some of these are discussed below.

\textit{Ludwig} has been run successfully on up to 131,072 MPI
tasks with close to ideal scaling, and there is no problem in
principle to run larger decompositions.

\subsection{Things to Remember and Things to Avoid}
\ref{section-advice}

There are a number of simple issues to get right:
\begin{itemize}
\item
Make sure the assertions are switched off and appropriate optimisations
are switched on for production runs (see Makefile).
\item
Do not use too small a local domain size. Typically, 16$^3$ to 32$^3$
cubic local domains will give resaonble scaling. Anything smaller may
be inefficient. The best size may depend on the exact nature of the
calculation and the hardware. Note that this limitation will always
ultimately limit strong scaling.
\item
Don't have diagnsotic output (freq\_statistics) more often than is
necessary. This output requires a global communication which is slow.
\item
Use BINARY output options; ASCII output is slower, and produces larger
files.
\item
Do not use serial I/O. This will have an increasingly high overhead as
the number of processors is increases (it's probably reasonable to
128--256 MPI tasks). This is because each MPI tasks writes its data,
in turn, to a single file; this in inherently serial. To avoid this,
you need to specify the I/O decomposition for the various output
quantities of interest. This splits the system into regular blocks
(in a similar approach to the MPI decomposition), the data for which
are written to separate files in parallel. For example, if the MPI
decomposition is 8\_8\_8 (512 MPI tasks), then a reasonable I/O
decomposition may be 2\_2\_2 or 4\_4\_4. The only way to find out
what works best is to try some tests for the problem size you want to
address.
\item
Don't use the Ewald sum! This simply does not scale.
An alternative algorithm is required.
\end{itemize}
