%%%%%%%%%%%%%%%%%%%%%%%%%%%%%%%%%%%%%%%%%%%%%%%%%%%%%%%%%%%%%%%%%%%%%%%%%%%%%
%
%  plan.tex
%
%  Software Configuration Management Plan
%
%  Edinburgh Soft Matter and Statistical Physics Group and
%  Edinburgh Parallel Computing Centre
%
%  Kevin Stratford (kevin@epcc.ed.ac.uk)
%  (c) 2011-2015 The University of Edinburgh
%
%%%%%%%%%%%%%%%%%%%%%%%%%%%%%%%%%%%%%%%%%%%%%%%%%%%%%%%%%%%%%%%%%%%%%%%%%%%%%

\documentclass[11pt,twoside]{article}

\setlength{\hoffset}{-1in}
\setlength{\voffset}{-1in}

\setlength{\topmargin}{2cm}
\setlength{\evensidemargin}{3.1cm}
\setlength{\oddsidemargin}{3.1cm}

\setlength{\textwidth}{14.8cm}
\setlength{\textheight}{23cm}

\setlength{\parindent}{0pt}
\setlength{\parskip}{\smallskipamount}


\newcommand{\inputkey}[1]{\framebox{\textbf{\texttt{#1}}}}
\newcommand{\e}[1]{\cdot10^{#1}}
\newcommand{\beq}{\begin{equation}}
\newcommand{\eeq}{\end{equation}}
\newcommand{\beqa}{\begin{eqnarray}}
\newcommand{\eeqa}{\end{eqnarray}}
\newcommand{\com}[1]{\textcolor{red}{#1}}
\newcommand{\cur}[1]{{\textit{#1}}}

\begin{document}

\setcounter{page}{1}

\tableofcontents

\newpage

\setcounter{page}{1}

\section{Software Configuration Management Plan}

\subsection{Introduction}

\subsubsection{Purpose}

This section contains information on the \textit{Software Configuration
Management Plan} for the \textit{Ludwig} code.
It is to provide users of the code a background on the way in which
the code is developed, how bugs are dealt with, the way in which
new features may be added, and so on. It is therefore a part
of the process by which the code is maintained. The Software Configuration
Management Plan will be referred to as `The Plan' throughout the remainder
of the section.

\subsubsection{Scope}

The \textit{Ludwig} code is designed to study the hydrodynamic properties
of simple and complex fluids based around the numerical solution of the
Navier-Stokes equations. Complex fluids are dealt with via a free-energy
based finite difference approach, which couples explicitly to the
hydrodynamics. Examples of complex fluids include binary mixtures,
surfactants, liquid crystals, and active gels. Solid objects such as
porous media may be included, as well as moving particles (colloids).
Components of \textit{Ludwig} include the main program, unit and
regression tests, and a small number of utility programs used to prepare
input and post-process output. The Plan is limited to these
software components.

\textit{Ludwig} is specifically designed for parallel computers and
relies on the Message Passing Interface \cite{mpi-standard}. The
code will also run in serial, and is designed to be independent of
the platform it is run on. The Plan is therefore not concerned with
hardware or system management activities.

This document is meant to be fairly informal, but will be updated as
the need arises. 
\textit{Ludwig} is an on-going research project, and relies mainly
on funding for the United Kingdom Engineering and Physical Sciences Research
Council to provide support for staff time to work on maintenance and
development. As such, we cannot commit to any and all user requests!

\subsubsection{Definition of key terms}

\subsubsection{References}

This section is based on the IEEE standard for Software Configuration
Management Plans IEEE 828-2012 \cite{ieee-828-2012}, and follows the format
set out therein.
Relevant references are included and can be found in the main References
section at the end of the document.

\subsection{Management}

\subsubsection{Organisations}

\textit{Ludwig} is developed by a team based in the School of Physics at
The University of Edinburgh. This involves two groups: the Soft Matter
Physics Group  under the leadership of Prof. Mike Cates FRS, and EPCC in
the person of Kevin Stratford. We collaborate with a number of workers at
different Universities around the world on different aspects of the code.

\subsubsection{Responsibilities}

While the Soft Matter Physics Group is responsible for the overall
scientific direction and development of the code and concomitant
priorities, all the Software Configuration Management activities will
be the responsibility of EPCC.

\subsubsection{Procedures}

There are currently no external constraints on The Plan.

\subsubsection{Process Management}

EPCC is responsible for the Software Configuration Management process.
It is anticipated that the cost of this process will be negligible, and
will not require monitoring. There is currently no independent
surveillance of activities to ensure compliance with The Plan.

\subsection{Activities}

\subsubsection{Identification}

The material under control of the project includes three sets of files:
this documentation, the source code, and the test suite which is used
to monitor the status of the code.
Control of project files is via a single on-line repository which
uses the subversion (SVN) revision control system. The SVN repository
is currently located at

\texttt{http://ccpforge.cse.rl.ac.uk/}

which is maintained at the Rutherford Appleton Laboratory on behalf
of Collaborative Computational Physics 5 (The Computer Simulation of
Condensed Phases). Subversion identifies different revisions by a
unique revision number. A given version of a file is then uniquely
identified by its path in the repository as it exists at a given
revision number. For example, SVN revision 1416 contains the
following directories:


\texttt{/trunk/doc} source files for this documentation;

\texttt{/trunk/mpi\_s} an MPI stub library;

\texttt{/trunk/src} source code

\texttt{/trunk/targetDP} thread parallelism abstraction layer, currently
under development;

\texttt{/trunk/tests} unit and regression test suite

\texttt{/trunk/util} utility programs

All access is via the CCPForge repository. Users can obtain the code
via anonymous svn access at CPPForge.
All developers of the \textit{Ludwig} project are currently registered with
a CCPForge user account, and fall into one of two categories.
Administrator or Developer. Only the account-holding developers have
permission to change files in the repository. Users have read-only
access.

As CCPForge is not physically under our control, we undertake a
complete dump of the repository itself to a local resource on a
regular basis for security.

\subsubsection{Control}

Requests for changes or additions to the code should be made via
the issue tracker at CCPForge. The project team will then decide
whether the change is possible. If so, the implementation and
testing of the change will take place, and the new code committed
back to the repository.


\subsubsection{Status}

Can we account for the status of the code and tests.

\subsubsection{Evaluation and review}

How do we evaluate and audit the code?

\subsubsection{Interface control}

At the moment, we do not explicitly interface to any code outside The
Plan (except MPI). We could extend coverage to include data export
to e.g., Paraview \cite{paraview}.

\subsubsection{Subcontractor / vendor control}

None required.

\subsubsection{Release management and delivery}

There is currently no formal release mechanism. If such a mechanism were
introduced, management and  delivery would be via the SVN repository.

\subsection{Schedules}

Sequence of events for SCM process.

\subsection{Resources}

For each active, state what resources (infrastructure, software tools,
personnel etc) are required.


\subsection{Plan Maintenance}

The Plan is currently in a draft form. A complete version of The Plan
will be in place to coincide with the public release.

\newpage
\section{Coordination Meetings}


\begin{thebibliography}{99}
\bibitem{ieee-828-2012} IEEE standard for Configuration Management
\bibitem{mpi-standard} Message passing ineterface.
\bibitem{paraview} Paraview is a standard visualisation package. 
\end{thebibliography}

\end{document}
