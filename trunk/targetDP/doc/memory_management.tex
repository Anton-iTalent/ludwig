\chapter{Memory Management}



\newpage
\section{targetMalloc}

\subsection{Description}

The \verb+targetMalloc+ function allocates memory on the target.

\subsection{Syntax}
\begin{verbatim}
void targetMalloc(void **targetPtr, size_t n);
\end{verbatim}

\begin{itemize}
\item \verb+targetptr+: A pointer to the allocated memory.
\item \verb+n+: The number of bytes to be allocated.
\end{itemize}


\subsection{Example}
See Line \ref{example:scalelaunch:targetMalloc} in Figure \ref{fig:scalelaunch} in Section \ref{chapter:examples}.
\subsection{Implementation}
\subsubsection{C}
\verb+malloc+
\subsubsection{CUDA}
\verb+cudaMalloc+

\newpage
\section{targetCalloc}

\subsection{Description}

The \verb+targetCalloc+ function allocates, and initializes to zero, memory on the target.

\subsection{Syntax}
\begin{verbatim}
void targetCalloc(void **targetPtr, size_t n);
\end{verbatim}

\begin{itemize}
\item \verb+targetptr+: A pointer to the allocated memory.
\item \verb+n+: The number of bytes to be allocated.
\end{itemize}


\subsection{Example}
Analogous to Line \ref{example:scalelaunch:targetMalloc} in Figure \ref{fig:scalelaunch} in Section \ref{chapter:examples}.

\subsection{Implementation}
\subsubsection{C}
\verb+calloc+
\subsubsection{CUDA}
\verb+cudaMalloc+ followed by \verb+cudaMemset+

\newpage
\section{targetFree}

\subsection{Description}

The \verb+targetFree+ function deallocates memory on the target.

\subsection{Syntax}
\begin{verbatim}
void targetFree(void *targetPtr);
\end{verbatim}

\begin{itemize}
\item \verb+targetPtr+: A pointer to the memory to be freed.
\end{itemize}


\subsection{Example}
See Line \ref{example:scalelaunch:targetFree} in Figure \ref{fig:scalelaunch} in Section \ref{chapter:examples}.

\subsection{Implementation}
\subsubsection{C}
\verb+free+
\subsubsection{CUDA}
\verb+cudaFree+

\newpage
\section{copyToTarget}

\subsection{Description}

The \verb+copyToTarget+ function copies data from the host to the target.

\subsection{Syntax}
\begin{verbatim}
void copyToTarget(void *targetData, const void *data, size_t n);
\end{verbatim}

\begin{itemize}
\item \verb+targetData+: A pointer to the destination array on the target.
\item \verb+data+: A pointer to the source array on the host.
\item \verb+n+: The number of bytes to be copied.
\end{itemize}


\subsection{Example}
See Line \ref{example:scalelaunch:copyToTarget} in Figure \ref{fig:scalelaunch} in Section \ref{chapter:examples}.

\subsection{Implementation}
\subsubsection{C}
\verb+memcpy+
\subsubsection{CUDA}
\verb+cudaMemcpy+

\newpage
\section{copyFromTarget}

\subsection{Description}

The \verb+copyFromTarget+ function copies data from the target to the host.

\subsection{Syntax}
\begin{verbatim}
void copyFromTarget(void *data, const void *targetData, size_t n);
\end{verbatim}

\begin{itemize}
\item \verb+data+: A pointer to the destination array on the host.
\item \verb+targetData+: A pointer to the source array on the target.
\item \verb+n+: The number of bytes to be copied.
\end{itemize}


\subsection{Example}
See Line \ref{example:scalelaunch:copyFromTarget} in Figure \ref{fig:scalelaunch} in Section \ref{chapter:examples}.

\subsection{Implementation}
\subsubsection{C}
\verb+memcpy+
\subsubsection{CUDA}
\verb+cudaMemcpy+


\newpage
\section{targetConst}
\subsection{Description}

The \verb+__targetConst__+ keyword is used in a variable or array declaration
to specify that the corresponding data can be treated as constant (read-only) on the target.

\subsection{Syntax}
\begin{verbatim}
__targetConst__ type variableName
\end{verbatim}

\begin{itemize}
\item \verb+variableName+: The name of the variable or array.
\item \verb+type+: The type of variabe or array.
\end{itemize}


\subsection{Example}
**TO DO**.
\subsection{Implementation}
\subsubsection{C}
Holds no value
\subsubsection{CUDA}
\verb+__constant__+

\newpage
\section{copyConstToTarget}

\subsection{Description}

The \verb+copyConstToTarget+ function copies data from the host to the target, where the data will remain constant (read-only) during the execution of functions on the target.

\subsection{Syntax}
\begin{verbatim}
void copyConstToTarget(void *targetData, const void *data, size_t n);
\end{verbatim}

\begin{itemize}
\item \verb+targetData+: A pointer to the destination array on the target. This must have been declared using the \verb+__targetConst__ + keyword.
\item \verb+data+: A pointer to the source array on the host.
\item \verb+n+: The number of bytes to be copied.
\end{itemize}


\subsection{Example}
See Line \ref{example:scalelaunch:copyConstToTarget} in Figure \ref{fig:scalelaunch} in Section \ref{chapter:examples}.
\subsection{Implementation}
\subsubsection{C}
\verb+memcpy+
\subsubsection{CUDA}
\verb+cudaMemcpyToSymbol+

\newpage
\section{copyConstFromTarget}

\subsection{Description}

The \verb+copyConstFromTarget+ function copies data from a constant data location on the target to the host.

\subsection{Syntax}
\begin{verbatim}
void copyConstToTarget(void *targetData, const void *data, size_t n);
\end{verbatim}

\begin{itemize}
\item \verb+data+: A pointer to the destination array on the host.
\item \verb+targetData+: A pointer to the source array on the target. This must have been declared using the \verb+__targetConst__ + keyword.
\item \verb+n+: The number of bytes to be copied.
\end{itemize}


\subsection{Example}
Analogous to Line \ref{example:scalelaunch:copyConstToTarget} in Figure \ref{fig:scalelaunch} in Section \ref{chapter:examples}.

\subsection{Implementation}
\subsubsection{C}
\verb+memcpy+
\subsubsection{CUDA}
\verb+cudaMemcpyFromSymbol+

\newpage
\section{targetConstAddress}

\subsection{Description}

The \verb+targetConstAddress+ function provides the target address for a constant object.

\subsection{Syntax}
\begin{verbatim}
void targetConstAddress(void **address, objectType object);
\end{verbatim}

\begin{itemize}
\item \verb+address+ (output): The pointer to the constant object on the target.
\item \verb+objectType+: The type of the object.
\item \verb+object+ (input): The constant object on the target. This should have been declared using the \verb+__targetConst__+ keyword.
\end{itemize}


\subsection{Example}
**TO DO**.

\subsection{Implementation}
\subsubsection{C}
Explicit copying of address.
\subsubsection{CUDA}
\verb+cudaGetSymbolAddress+

\newpage
\section{targetInit3D}

\subsection{Description}

The \verb+targetInit3D+ initialises the environment required to perform any of following the targetDP 3D lattice operations.

\subsection{Syntax}
\begin{verbatim}
void targetInit3D(size_t extent, size_t nFields);
\end{verbatim}

\begin{itemize}
\item \verb+extent+: The total extent of data parallelism (e.g. the number of lattice sites).
\item \verb+nFields+: The extent of data resident within each parallel partition (e.g. the number of fields per lattice site).
\end{itemize}


\subsection{Example}
**TO DO**.
\subsection{Implementation}
\subsubsection{C}
**TO DO**.
\subsubsection{CUDA}
**TO DO**.

\newpage
\section{targetFinalize3D}

\subsection{Description}

The \verb+targetFinalize3D+ finalizes the targetDP 3D environment.

\subsection{Syntax}
\begin{verbatim}
void targetFinalize3D();
\end{verbatim}

\subsection{Example}
**TO DO**.

\subsection{Implementation}
\subsubsection{C}
**TO DO**.
\subsubsection{CUDA}
**TO DO**.


\newpage
\section{copyToTargetBoundary3D}

\subsection{Description}

The \verb+copyToTargetBoundary3D+ function copies the data corresponding to the boundaries of a 3D lattice from the host to the target.

\subsection{Syntax}
\begin{verbatim}
void copyToTargetBoundary3D(void *targetData, const void *data, size_t extent3D[3], size_t nField, size_t offset, size_t depth);
\end{verbatim}

\begin{itemize}
\item \verb+targetData+: A pointer to the destination array on the target.
\item \verb+data+: A pointer to the source array on the host.
\item \verb+extent3D+: An array of 3 integers corresponding to the 3D dimensions of the lattice.
\item \verb+nFields+: The number of fields per lattice site.
\item \verb+offset+: The number of sites from the lattice edge at which each boundary face should start.
\item \verb+depth+: The depth of each boundary face.
\end{itemize}


\subsection{Example}
**TO DO**.
\subsection{Implementation}
\subsubsection{C}
**TO DO**.
\subsubsection{CUDA}
**TO DO**.

\newpage
\section{copyFromTargetBoundary3D}

\subsection{Description}

The \verb+copyFromTargetBoundary3D+ function copies the data corresponding to the boundaries of a 3D lattice from the target to the host.

\subsection{Syntax}
\begin{verbatim}
void copyFromTargetBoundary3D(void *data, const void *targetData, size_t extent3D[3], size_t nField, size_t offset, size_t depth);
\end{verbatim}

\begin{itemize}
\item \verb+data+: A pointer to the destination array on the host.
\item \verb+targetData+: A pointer to the source array on the target.
\item \verb+extent3D+: An array of 3 integers corresponding to the 3D dimensions of the lattice.
\item \verb+nFields+: The number of fields per lattice site.
\item \verb+offset+: The number of sites from the lattice edge at which each boundary face should start.
\item \verb+depth+: The depth of each boundary face.
\end{itemize}


\subsection{Example}
**TO DO**.
\subsection{Implementation}
\subsubsection{C}
**TO DO**.
\subsubsection{CUDA}
**TO DO**.
\newpage
\section{copyToTargetPointerMap3D}

\subsection{Description}

The \verb+copyToTargetPointerMap3D+ function copies a subset of lattice data from the host to the target. The sites to be included are defined using an array of pointers passed as input.

\subsection{Syntax}
\begin{verbatim}
void copyToTargetPointerMap3D(void *targetData, const void *data, 
           size_t extent3D[3], size_t nField, 
           int includeNeighbours, void** pointerArray);
\end{verbatim}

\begin{itemize}
\item \verb+targetData+: A pointer to the destination array on the target.
\item \verb+data+: A pointer to the source array on the host.
\item \verb+extent3D+: An array of 3 integers corresponding to the 3D dimensions of the lattice.
\item \verb+nField+: The number of fields per lattice site.
\item \verb+includeNeighbours+: A boolean switch to specify whether each included site should also have it's neighbours included (in the 19-point 3D stencil).
\item \verb+pointerArray+: An array of \verb+nSite+ pointers, where \verb+nSite+ is the total number of lattice sites. Each lattice site should be included unless the pointer corresponding to that site is \verb+NULL+.  
\end{itemize}


\subsection{Example}
**TO DO**.
\subsection{Implementation}
\subsubsection{C}
**TO DO**.
\subsubsection{CUDA}
**TO DO**.

\newpage
\section{copyFromTargetPointerMap3D}

\subsection{Description}

The \verb+copyFromTargetPointerMap3D+ function copies a subset of lattice data from the target to the host. The sites to be included are defined using an array of pointers passed as input.

\subsection{Syntax}
\begin{verbatim}
void copyFromTargetPointerMap3D(void *data, const void *targetData, 
           size_t extent3D[3], size_t nField, 
           int includeNeighbours, void** pointerArray);
\end{verbatim}

\begin{itemize}
\item \verb+data+: A pointer to the destination array on the host.
\item \verb+targetData+: A pointer to the source array on the target.
\item \verb+extent3D+: An array of 3 integers corresponding to the 3D dimensions of the lattice.
\item \verb+nField+: The number of fields per lattice site.
\item \verb+includeNeighbours+: A boolean switch to specify whether each included site should also have it's neighbours included (in the 19-point 3D stencil).
\item \verb+pointerArray+: An array of \verb+nSite+ pointers, where \verb+nSite+ is the total number of lattice sites. Each lattice site should be included unless the pointer corresponding to that site is \verb+NULL+.  
\end{itemize}


\subsection{Example}
**TO DO**.

\subsection{Implementation}
\subsubsection{C}
**TO DO**.
\subsubsection{CUDA}
**TO DO**.

