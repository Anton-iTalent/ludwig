%%%%%%%%%%%%%%%%%%%%%%%%%%%%%%%%%%%%%%%%%%%%%%%%%%%%%%%%%%%%%%%%%%%%%%%%%%%%%%
%
%  input.tex
%
%  Instructions for contents of the user input file.
%
%  Edinburgh Soft Matter and Statistical Physics Group and
%  Edinburgh Parallel Computing Centre
%
%  (c) 2016-2017 The University of Edinburgh
%
%%%%%%%%%%%%%%%%%%%%%%%%%%%%%%%%%%%%%%%%%%%%%%%%%%%%%%%%%%%%%%%%%%%%%%%%%%%%%%

\section{The Input File}

\subsection{General}

By default, the run time expects to find user input in a file
\texttt{input} in the current working directory. If a different
file name is required, its name
should be provided as the sole command line argument, e.g.,
\begin{lstlisting}
./Ludwig.exe input_file_name
\end{lstlisting}
If the input file is not located in the current working directory
the code will terminate immediately with an error message.

When an input file is located, its content is read by a single MPI
task, and its contents then broadcast to all MPI relevant tasks.
The format of the file is plain ASCII text, and its contents are
parsed on a line by line basis. Lines may contain the following:
\begin{itemize}
\item comments introduced by \texttt{\#}.
\item key value pairs separated by white space.
\end{itemize}
Blank lines are treated as comments. The behaviour of the code is
determined by a set of key value pairs. Any given key may appear
only once in the input file; unused key value pairs are not reported.
If the key value pairs are not correctly formed, the code will terminate
with an error message and indicate the offending input line.

Key value pairs may be present in the input file, but have no effect for
any given run: they are merely ignored. Relevant control parameters for
given input are reported in the standard output.

\subsubsection{Key value pairs}

Key value pairs are made up of a key --- an alphanumeric string with no
white space --- and corresponding value following white space. Values
may take on the follow forms:
\begin{lstlisting}
key_string           value_string

key_integer_scalar  1
key_integer_vector  1_2_3

key_double_scalar    1.0
key_double_vector    1.0_2.0_3.0
\end{lstlisting}
Values which are strings should contain no white space. Scalar parameters
may be integer values, or floating point values with a decimal point
(scientific notation is also allowed).  Vector parameters are introduced
by a set of three values (to be interpreted as $x,y,z$ components of the
vector in Cartesian coordinates) separated by an underscore. The identity
of the key will specify what type of value is expected. Keys and (string)
values are case sensitive.


Most keys have an associated default value which will be used if
the key is not present. Some keys must be specified: an error will
occur if they are missing. The remainder of this part
of the guide details the various choices for key value pairs,
along with any default values, and any relevant constraints.

\subsection{The Free Energy}
\label{input-free-energy}

The choice of free energy is determined as follows:
\begin{lstlisting}
free_energy   none
\end{lstlisting}
The default value is \texttt{none}, i.e., a simple Newtonian fluid is used.
Possible values of the \texttt{free\_energy} key are:
\begin{lstlisting}
#  none                     Newtonian fluid [DEFAULT]
#  symmetric                Symmetric binary fluid (finite difference)
#  symmetric_lb             Symmetric binary fluid (two distributions)
#  brazovskii               Brazovskii smectics
#  surfactant               Surfactants
#  polar_active             Polar active gels
#  lc_blue_phase            Liquid crystal (nematics, cholesterics, BPs)
#  lc_droplet               Liquid crystal emulsions
#  fe_electro               Single fluid electrokinetics
#  fe_electro_symmetric     Binary fluid electrokinetics
\end{lstlisting}

The choice of free energy will control automatically a number of factors
related to choice of order parameter, the degree of parallel communication
required, and so on. Each free energy has a number of associated parameters
discussed in the following sections.

Details of general (Newtonian) fluid parameters, such as viscosity,
are discussed in Section~\ref{input-fluid-parameters}.

\subsubsection{Symmetric Binary Fluids}

We recall that the free energy density is, as a function of compositional
order $\phi$:
\[
{\textstyle \frac{1}{2}} A\phi^2 +
{\textstyle \frac{1}{4}} B\phi^4 +
{\textstyle \frac{1}{2}} \kappa (\mathbf{\nabla}\phi)^2.
\]

Parameters are introduced by (with default values):
\begin{lstlisting}
free_energy  symmetric
A            -0.0625                         # Default: -0.003125
B            +0.0625                         # Default: +0.003125
K            +0.04                           # Default: +0.002
\end{lstlisting}
Common usage has $A < 0$ and $B = -A$ so that $\phi^\star = \pm 1$.
The parameter $\kappa$
(key \texttt{K}) controls
the interfacial energy penalty and is usually positive.

\subsubsection{Brazovskii smectics}
\label{input-brazovskki-smectics}
The free energy density is:
\[
{\textstyle \frac{1}{2}} A\phi^2 +
{\textstyle \frac{1}{4}} B\phi^4 +
{\textstyle \frac{1}{2}} \kappa (\mathbf{\nabla}\phi)^2 +
{\textstyle \frac{1}{2}} C (\nabla^2 \phi)^2 
\]

Parameters are introduced via the keys:
\begin{lstlisting}
free_energy  brazovskii
A             -0.0005                        # Default: 0.0
B             +0.0005                        # Default: 0.0
K             -0.0006                        # Default: 0.0
C             +0.00076                       # Default: 0.0
\end{lstlisting}
For $A<0$, phase separation occurs with a result depending on $\kappa$:
one gets two symmetric phases for $\kappa >0$ (cf.\ the symmetric case)
or a lamellar phase for $\kappa < 0$. Typically, $B = -A$ and the
parameter in the highest derivative $C > 0$.

\subsubsection{Surfactants}
\label{input-surfactants}

The surfactant free energy should not be used at the present time.

\subsubsection{Polar active gels}
\label{input-polar-active-gels}

The free energy density is a function of vector order parameter $P_\alpha$:
\[
{\textstyle \frac{1}{2}} A P_\alpha P_\alpha +
{\textstyle \frac{1}{4}} B (P_\alpha P_\alpha)^2 +
{\textstyle \frac{1}{2}} \kappa (\partial_\alpha P_\beta)^2
\]

There are no default parameters:
\begin{lstlisting}
free_energy        polar_active
polar_active_a    -0.1                       # Default: 0.0
polar_active_b    +0.1                       # Default: 0.0
polar_active_k     0.01                      # Default: 0.0
\end{lstlisting}
It is usual to choose $B > 0$, in which case $A > 0$ gives
an isotropic phase, whereas $A < 0$ gives a polar nematic phase.
The elastic constant $\kappa$ (key \texttt{polar\_active\_k})
is positive.

\subsubsection{Liquid crystal}
\label{input-liquid-crystal}
The free energy density is a function of tensor order parameter
$Q_{\alpha\beta}$:
\begin{eqnarray}
{\textstyle\frac{1}{2}} A_0 (1 - \gamma/3)Q^2_{\alpha\beta} -
{\textstyle\frac{1}{3}} A_0 \gamma
                        Q_{\alpha\beta}Q_{\beta\delta}Q_{\delta\alpha} +
{\textstyle\frac{1}{4}} A_0 \gamma (Q^2_{\alpha\beta})^2
\nonumber \\
+ {\textstyle\frac{1}{2}} \Big(
\kappa_0 (\epsilon_{\alpha\delta\sigma} \partial_\delta Q_{\sigma\beta} +
2q_0 Q_{\alpha\beta})^2 + \kappa_1(\partial_\alpha Q_{\alpha\beta})^2 \Big)
\nonumber
\end{eqnarray}

The corresponding \texttt{free\_energy} value, despite its name, is
suitable for nematics and cholesterics, and not just blue phases:
\begin{lstlisting}
free_energy      lc_blue_phase
lc_a0            0.01                       # Deafult: 0.0
lc_gamma         3.0                        # Default: 0.0
lc_q0            0.19635                    # Default: 0.0
lc_kappa0        0.00648456                 # Default: 0.0
lc_kappa1        0.00648456                 # Default: 0.0
\end{lstlisting}
The bulk free energy parameter $A_0$ is positive and controls the
energy scale (key \texttt{lc\_a0}); $\gamma$ is positive and
influences the position in the phase diagram relative to the
isotropic/nematic transition (key \texttt{lc\_gamma}).
The two elastic constants must be equal, i.e., we enforce the
single elastic constant approximation (both keys \texttt{lc\_kappa0} and
\texttt{lc\_kappa1} must be specified).

Other important parameters in the liquid crystal picture are:
\begin{lstlisting}
lc_xi            0.7                         # Default: 0.0
lc_Gamma         0.5                         # Default: 0.0
lc_active_zeta   0.0                         # Default: 0.0
\end{lstlisting}
The first is $\xi$ (key \texttt{lc\_xi}) is the effective molecular
aspect ratio and should be in the range $0< \xi< 1$. The rotational
diffusion constant is $\Gamma$ (key \texttt{lc\_Gamma}; not to be
confused with \texttt{lc\_gamma}). The (optional) apolar activity
parameter is $\zeta$ (key \texttt{lc\_active\_zeta}).

\subsubsection{Liquid crystal anshoring}

Different types of anchoring are available at solid surfaces, with
one or two related free energy parameters depending on the type.
The type of anchoring may be set independently for stationary
boundaries (walls) and colloids.
\begin{lstlisting}
lc_anchoring_strength     0.01               # free energy parameter w1
lc_anchoring_strength_2   0.0                # free energy parameter w2
lc_wall_anchoring         normal             # ``normal'' or ``planar''
lc_coll_anchoring         normal             # ``normal'' or ``planar''
\end{lstlisting}
See section \ref{section-lc-anchoring} for details of surface anchoring.

\subsubsection{Liquid crystal emulsion}
\label{input-liquid-crystal-emulsion}

This an interaction free energy which combines the symmetric and liquid
crystal free energies. The liquid crystal free energy constant $\gamma$
becomes a
function of composition via $\gamma(\phi) = \gamma_0 + \delta(1 + \phi)$,
and a coupling term is added to the free energy density:
\[
WQ_{\alpha\beta} \partial_\alpha \phi \partial_\beta \phi.
\]
Typically, we might choose $\gamma_0$ and $\delta$ so that
$\gamma(-\phi^\star) < 2.7$ and the $-\phi^\star$ phase is isotropic,
while $\gamma(+\phi^\star) > 2.7$ and the
$+\phi^\star$ phase is ordered (nematic, cholesteric, or blue phase).
Experience suggests that a suitable choice is $\gamma_0 = 2.5$ and
$\delta = 0.25$.

For anchoring constant $W > 0$, the liquid crystal anchoring at the
interface is planar, while for $W < 0$ the anchoring is normal. This
is set via key \texttt{lc\_droplet\_W}.

Relevant keys (with default values) are:
\begin{lstlisting}
free_energy            lc_droplet

A                      -0.0625
B                      +0.0625
K                      +0.053

lc_a0                   0.1
lc_q0                   0.19635
lc_kappa0               0.007
lc_kappa1               0.007

lc_droplet_gamma        2.586                # Default: 0.0
lc_droplet_delta        0.25                 # Default: 0.0
lc_droplet_W           -0.05                 # Default: 0.0
\end{lstlisting}
Note that key \texttt{lc\_gamma} is not set in this case.

\subsection{System Parameters}
\label{input-system-parameters}

Basic parameters controlling the number of time steps
and the system size are:
\begin{lstlisting}
N_start      0                              # Default: 0
N_cycles     100                            # Default: 0
size         128_128_1                      # Default: 64_64_64
\end{lstlisting}
A typical simulation will start from time zero (key \texttt{N\_start})
and run for a certain number of time steps (key \texttt{N\_cycles}).
The system size (key \texttt{size}) specifies the total number of
lattice sites in each dimension. If a two-dimensional system is
required, the extent in the $z$-direction must be set to unity, as
in the above example.

If a restart from a previous run is required, the choice of parameters
may be as follows:
\begin{lstlisting}
N_start      100
N_cycles     400
\end{lstlisting}
This will restart from data previously saved at time step 100, and
run a further 400 cycles, i.e., to time step 500.
% TODO Cross-reference to I/O.

\subsubsection{Parallel decomposition}

In parallel, the domain decomposition is closely related to the
system size, and is specified as follows:
\begin{lstlisting}
size         64_64_64
grid         4_2_1
\end{lstlisting}
The \texttt{grid} key specifies the number of MPI tasks required in
each coordinate direction. In the above example, the decomposition
is into 4 in the $x$-direction, into 2 in the $y$-direction, while
the $z$-direction is not decomposed. In this example, the local domain
size per MPI
task would then be $16\times32\times64$. The total number of MPI tasks
available must match the total implied by \texttt{grid} (8 in the
example).

The \texttt{grid} specifications must exactly divide the system size;
if no decomposition is possible, the code will terminate with an error
message.
If the requested decomposition is not valid, or \texttt{grid} is
omitted, the code will try to supply a decomposition based on
the number of MPI tasks available and \texttt{MPI\_Dims\_create()};
this may be implementation dependent.


\subsection{Fluid Parameters}
\label{input-fluid-parameters}

Control parameters for a Newtonian fluid include:
\begin{lstlisting}
fluid_rho0                 1.0
viscosity                  0.166666666666666
viscosity_bulk             0.166666666666666
isothermal_fluctuations    off
temperature                0.0
\end{lstlisting}
The mean fluid density is $\rho_0$ (key \texttt{fluid\_rho0}) which
defaults to unity in lattice units; it is not usually necessary to
change this. The shear viscosity is
\texttt{viscosity} and as default value 1/6 to correspond to
unit relaxation time in the lattice Boltzmann picture. Reasonable
values of the shear viscosity are $0.2 > \eta > 0.0001$ in lattice
units. Higher values move further into the over-relaxation region, and can
result in poor behaviour. Lower
values increase the Reynolds number and tend to cause
problems with stability. The bulk
viscosity has a default value which is equal to whatever shear
viscosity has been selected. Higher values of the bulk viscosity
may be set independently and can help to suppress large deviations
from incompressibility and maintain numerical stability
in certain situations.

If fluctuating hydrodynamics is wanted, set the value of
 \texttt{isothermal\_fluctuations} to \texttt{on}. The associated
temperature is in lattice units: reasonable values (at $\rho_0 = 1$)
are $0 < kT < 0.0001$. If the temperature is too high, local
velocities will rapidly exceed the Mach number constraint and
the simulation will be unstable.

\subsection{Lees Edwards Planes}

Constant uniform shear may be introduced via a number of Lees Edwards
planes with given speed. This is appropriate for fluid-only
systems with periodic boundaries.
\begin{lstlisting}
N_LE_plane               2           # Number of planes (default: 0)
LE_plane_vel             0.05        # Constant plane speed
\end{lstlisting}
The placing of the planes in the system is as follows.
The number of planes $N$ must
divide evenly the lattice size in the $x$-direction to give an integer
$\delta x$. Planes are then placed at $\delta x / 2, 3\delta x/2, \ldots$.
All planes have the same, constant, velocity jump associated with them:
this is positive in the positive $x$-direction. (This jump is usually
referred to as the plane speed.) The uniform shear rate
will be $\dot{\gamma} = N U_{LE} / L_x$ where $U_{LE}$ is the plane
speed (which is always in the $y$-direction).

The velocity gradient or shear direction is $x$, the flow
direction is $y$ and the vorticity direction is $z$.

The spacing between planes must not be less than twice the halo size
in lattice units; 8--16 lattice units may be the practical limit in
many cases. In additional, the speed of the planes must not cause a
violation of the
Mach number constraint in the fluid velocity on the lattice, which
will match the plane speed in magnitude directly either side of the
planes. A value of around 0.05 should be regarded as a maximum safe
limit for practical purposes.

Additional keys associated with the Lees Edwards planes are:
\begin{lstlisting}
LE_init_profile          1           # Initialise u(x) (off/on)
LE_time_offset           10000       # Offset time (default 0)
\end{lstlisting}
If \texttt{LE\_init\_profile} is set, the fluid velocity is initialised
to reflect a steady state shear flow appropriate for the number of
planes at the given speed (ie., shear rate). If set to zero (the default),
the fluid is initialised with zero velocity.

The code works out the current displacement of the planes by computing
$U_{LE} t$, where $t$ is the current time step. A shear run should then
start from $t = 0$, i.e. zero plane displacement.
It is often convenient to run an equilibration with no shear, and
then to start an experiment after some number of steps. This
key allows you to offset the start of the Lees-Edwards motion.
It should then take the value of the start time (in time steps)
corresponding to the restart at the end of the equilibration
period.

There are a couple of additional constraints to use the Lees-Edwards
planes in parallel. In particular, the planes cannot fall at a
processor boundary in the $x$-direction. This means you should
arrange an integer number of planes per process in the $x$-direction.
(For example, use one plane per process; this will also ensure the number
of planes
still evenly divides the total system size.)
This will interleave the planes with the processor decomposition.
The $y$-direction and $z$-direction may be decomposed without
further constraint.

Note that this means a simulation with one plane will only work
if there is one process in the $x$ decomposition.



\subsection{Colloids}
If no relevant key words are present, no colloids will be
expected at run time. The simulation will progress in the
usual fashion with fluid only.

If colloids are required, the \texttt{colloid\_init}
key word must be present to allow the code to determine where
colloid information will come from. The options for the
\texttt{colloid\_init} key word are summarised below:
\begin{lstlisting}
colloid_init             none

#  none                  no colloids [DEFAULT]
#  input_one             one colloid from input file
#  input_two             two colloids from input file
#  input_three           three colloids from input file
#  input_random          Small number at random
#  from_file             Read a separate file (including all restarts)
\end{lstlisting}

For idealised simulations which require 1, 2, or 3 colloids, relevant
initial state information 
for each particle can be included in the input file. In principle, most
of the colloid state as defined in the colloid
state structure in \texttt{colloid.h} may be specified. (Some state is
reserved for internal use only and cannot be set from the input file.)
Furthermore, not all the state is relevant in all simulations ---
quantities such as charge and wetting parameters may not be required,
in which case they a simply ignored.

A minimal initialisation is shown in the following example:
\begin{lstlisting}
colloid_init              input_one

colloid_one_a0            1.25
colloid_one_ah            1.25
colloid_one_r             12.0_12.0_12.0
\end{lstlisting}
This initialises a single colloid with an input radius $a_0=1.25$,
and a hydrodynamic radius $a_h=1.25$; in general both are required,
but they do not have to be equal.
A valid position is required within the extent of the system
$0.5 < x,y,z < L + 0.5$ as specified by the \texttt{size} key word.
State which is not explicitly defined is initialised to zero.

\subsubsection{General Initialisation}

A full list of colloid state-related key words is as follows. All
the quantities are floating point numbers unless explicitly stated
to be otherwise:
\begin{lstlisting}
# colloid_one_nbonds      (integer) number of bonds
#   colloid_one_bond1     (integer) index of bond partner 1
#   colloid_one_bond2     (integer) index of bond partner 2
# colloid_one_isfixedr    colloid has fixed position (integer 0 or 1)
# colloid_one_isfixedv    colloid has fixed velocity (integer 0 or 1)
# colloid_one_isfixedw    colloid has fixed angular velocity (0 or 1)
# colloid_one_isfixeds    colloid has fixed spin (0 or 1)
# colloid_one_type        ``default'' COLLOID_TYPE_DEFAULT
#                         ``active''  COLLOID_TYPE_ACTIVE
#                         ``subgrid'' COLLOID_TYPE_SUBGRID
# colloid_one_a0          input radius
# colloid_one_ah          hydrodynamic radius
# colloid_one_r           position vector
# colloid_one_v           velocity (vector)
# colloid_one_w           angular velocity (vector)
# colloid_one_s           spin (unit vector)
# colloid_one_m           direction of motion (unit) vector for swimmers 
\end{lstlisting}

Note that for magnetic particles, the appropriate initialisation involves
the spin key word \texttt{colloid\_one\_s} which relates to the dipole
moment $\mu \mathbf{s}_i$, while \texttt{colloid\_one\_m} relates to the
direction of motion vector. Do not confuse the two.
It is possible in principle to have magnetic active particles,
in which case the dipole direction or spin ($\mathbf{s}_i$) and the
direction of swimming motion $\mathbf{m}_i$ are allowed to be distinct. 

\begin{lstlisting}
# colloid_one_b1          Squirmer parameter B_1
# colloid_one_b2          Squirmer parameter B_2
# colloid_one_rng         (integer) random number generator state
# colloid_one_q0          charge (charge species 0)
# colloid_one_q1          charge (charge species 1)
# colloid_one_epsilon     Permeativity
# colloid_one_c           Wetting parameter C
# colloid_one_h           Wetting parameter H
\end{lstlisting}

{\bf Example}: Single active (squirmer) particle

The following example shows a single active particle with initial
swimming direction along the $x$-axis.
\begin{lstlisting}
colloid_init              input_one

colloid_one_type          active
colloid_one_a0            7.25
colloid_one_ah            7.25
colloid_one_r             32.0_32.0_32.0
colloid_one_v             0.0_0.0_0.0
colloid_one_m             1.0_0.0_0.0
colloid_one_b1            0.05
colloid_one_b2            0.05
\end{lstlisting}



\subsubsection{Random initialisation}

For suspensions with more than few colloids, but still at
relatively low volume fraction (10--20\% by volume), it is
possible to request initialisation at random positions.

The additional key word value pair \texttt{colloid\_random\_no}
determines the total number of particles to be placed in
the system. To prevent particles being initialised very
close together, which can cause problems in the first few
time steps if strong potential interactions are present,
a ``grace'' distance or minimum surface-surface separation
may also be specified (\texttt{colloid\_random\_dh}).

The following example asks for 100 colloids to be initialised
at random positions, with a minimum separation of 0.5 lattice
spacing.

\begin{lstlisting}
colloid_init              input_random

colloid_random_no         100             # Total number of colloids
colloid_random_dh         0.5             # ``Grace'' distance

colloid_random_a0         2.30
colloid_random_ah         2.40
\end{lstlisting}

An input radius and hydrodynamic radius must be provided: these
are the same for all colloids.
If specific initialisations of the colloid state (excepting the
position) other than the radii are wanted, values should be provided
as for the single particle case in the preceding section, but using
\texttt{colloid\_random\_a0} in place of
\texttt{colloid\_one\_a0} and so on.

The code will try to initialise the requested number in the current
system size, but only makes a finite number of attempts to place
particles at random with no overlaps. (The initialisation will also
take into account the presence of any solid walls, using the same
grace distance.) If the the number of particles is too large, the
code will halt with a message to that effect.

In general, colloid information for a arbitrary configuration with many
particles should be read from a pre-prepared file. See the section on
File I/O for further information on reading files.


\subsubsection{Interactions}

Note that two-body pair-potential interactions are defined uniformly for
all colloids in a simulation. The same is true for lubrication corrections.
There are a number of constraints related to the computation of
interactions discussed below.

{\bf Boundary-colloid lubrication correction}.
Lubrication corrections (here the normal force) between a flat wall
(see section XREF) are
required to prevent overlap between colloid  and the wall.
The cutoff distance is set via the key word value pair
\begin{lstlisting}
boundary_lubrication_rcnormal    0.5
\end{lstlisting}
It is recommended that this is used in all cases involving walls.
A reasonable value is in the range $0.1 < r_c < 0.5$ in lattice
units, and should be calibrated for particle hydrodynamic radius
and fluid viscosity if exact results are important.

{\bf Colloid-colloid lubrication corrections}.
The key words to activate the calculation of lubrication corrections
are:
\begin{lstlisting}
lubrication_on                   1
lubrication_normal_cutoff        0.5
lubrication_tangential_cutoff    0.05
\end{lstlisting}



{\bf Soft sphere potential}.
A cut-and-shifted soft-sphere potential of the form
$v \sim \epsilon (\sigma/r)^\nu$ is
available. Some trial-and-error with the parameters may be required in
any given situation to ensure simulation stability in the long run. The
following gives an example of the relevant input key words:
\begin{lstlisting}
soft_sphere_on            1                 # integer 0/1 for off/on 
soft_sphere_epsilon       0.0004            # energy units
soft_sphere_sigma         1.0               # a length
soft_sphere_nu            1.0               # exponent is positive
soft_sphere_cutoff        2.25              # a surface-surface separation
\end{lstlisting}
See Section~\ref{section-colloids-soft-sphere} for a detailed description.

{\bf Lennard-Jones potential}.
The Lennard-Jones potential (Section~\ref{section-colloids-lennard-jones})
is controlled by the following key words:
\begin{lstlisting}
lennard_jones_on          1                 # integer 0/1 off/on
lj_epsilon                0.1               # energy units
lj_sigma                  2.6               # potential length scale
lj_cutoff                 8.0               # a centre-centre separation
\end{lstlisting}


{\bf Yukawa potential}.
A cut-and-shifted Yukawa potential of the form
$v \sim \epsilon \exp(-\kappa r)/r$ is
available using the following key word value pairs:

\begin{lstlisting}
yukawa_on                 1                 # integer 0/1 off/on
yukawa_epsilon            1.330             # energy units
yukawa_kappa              0.725             # an inverse length
yukawa_cutoff             16.0              # a centre-centre cutoff
\end{lstlisting}

{\bf Dipole-dipole interactions (Ewald sum)}.
The Ewald sum is completely specified in the input file
by the uniform dipole strength $\mu$ and the real-space cut off $r_c$.  
\begin{lstlisting}
ewald_sum                 1                 # integer 0/1 off/on
ewald_mu                  0.285             # dipole strength mu
ewald_rc                  16.0              # real space cut off
\end{lstlisting}


If short range interactions are required, particle information is stored
in a cell list, which allows efficient computation of the potentially
$N^2$ interactions present. This gives rise to a constraint that the
width of the cells must be large enough that all relevant interactions
are included. This generally means that the cells must be at least
$2a_h + h_c$ where $h_c$ is the largest relevant cut off distance.

The requirement for at least two cells per local domain in parallel
means that there is a associated minimum local domain size. This is
computed at run time on the basis of the input. If the local domain
is too small, the code will terminate with an error message. The
local domain size should be increased.


\subsubsection{External forces}

{\bf External body force (gravity)}.
The following example requests a uniform body force in the negative
$z$-direction on all particles.
\begin{lstlisting}
colloid_gravity           0.0_0.0_-0.001    # vector
\end{lstlisting}
The counterbalancing body force on the fluid which enforces the
constraint of momentum conservation for the system as a whole is
computed automatically by the code at each time step.


\subsection{Porous media}

Porous media calculations can be undertaken when appropriate
solid/fluid status information is supplied. The are a number
of switches available in the input file:
\begin{lstlisting}
porous_media_file
porous_media_format
porous_media_type
\end{lstlisting}

specifies the file stub name to be read at the start of execution.
(If the stub name is \texttt{file} then the code will expect to
find \texttt{file.001-001} in the current directory.

is either \texttt{ASCII} or \texttt{BINARY} as appropriate. Note that
in parallel, a single data file can be supplied, but it must be binary.
The default is \texttt{BINARY}.

is either \texttt{status\_only} (the default) or \texttt{status\_with\_h}.

\subsubsection{File format}

In all cases the status (fluid/solid) information is represented by a
single \texttt{char} (or integer), which must be supplied via the
porous media file. A single file should contain data matching the
current system size, and have the $z-$direction running fastest,
followed by the $y-$direction. Note that in non-periodic directions,
the structure must be 'closed', i.e., all the points at the edge
should be solid.

Fluid sites are designated by \texttt{0} and boundary or solid sites
by \texttt{1}. These data should be of type \texttt{char} in binary,
and may be integer in ASCII.

Where wetting information is required, the free energy parameter $H$
can be supplied by using the \texttt{status\_with\_h} switch. In this
case, the \texttt{char} status is augmented by a single \texttt{double}
value which is the local value of $H$. The order is then
$s_1,h_1, s_2, h_2, \ldots$.

An example of how to construct a porous media file is provided in
\texttt{util/capillary.c}, which builds an appropriate file for
a square or circular capillary tube. Please see the comments in
the file for further details. Note that the allowed
solid/fluid status values are defined in \texttt{src/site\_map.h}.
A solid boundary is \texttt{BOUNDARY}, while fluid is \texttt{FLUID}.



\subsection{Order parameter initialisations}

Free energy choices requiring one or more fluid order parameters can make
use of the following initial coniditions.

\subsubsection{Composition $\phi$}

The following initialisations are available.
\begin{lstlisting}
phi_initialisation          spinodal    # spinodal
phi0                        0.0         # mean
noise                       0.05        # noise amplitude
random_seed                 102839      # +ve integer
\end{lstlisting}
Suitable for initialising isothermal spinodal decomposition, the order
parameter may be set at random at each position via
$\phi = \phi_0 + A(\hat{\phi} - 1/2)$ with the random variate
$\hat\phi$ selected uniformly on the interval $[0,1)$. For symmetric
quenches (mean order parameter $\phi_0 = 0$ and $\phi^\star = \pm 1$),
a value of $A$ in the range 0.05-0.1 is usually appropriate.

For off-symmetric quenches, larger patches of fluid may be required to
initiate decomposition:
\begin{lstlisting}
phi_initialisation          patches     # patches of phi = +/- 1
phi_init_patch_size         2           # patch size
phi_init_patch_vol          0.1         # volume fraction phi = -1 phase
random_seed                 13          # +ve integer
\end{lstlisting}
The initialises cubics patches of fluid of given size with $\phi= \pm 1$
at random. The requested overall volume fractions may be met approximately.

A spherical drop can be initialised at the centre of the system.
\begin{lstlisting}
phi_initialisation          drop        # spherical droplet
phi_init_drop_radius        16.0        # radius
phi_init_drop_amplitude    -1.0         # phi value inside
\end{lstlisting}
The drop is initialised with a $\tanh(r/\xi)$ profile where the
interfacial width $\xi$ is computed via the current free energy
parameters.

% block
% bath

For restarted simulations, the default position is to read order
parameter information from file
\begin{lstlisting}
phi_initialisation          from_file
\end{lstlisting}
in which case a file or files for the appropriate time step should
be present in the working directory.


\subsubsection{Tensor order $Q_{\alpha\beta}$}

A number of different initialisations are available for the liquid
crystal order parameter $Q_{\alpha\beta}$. Some care may be required
to ensure consistency between the choice and the free energy
parameters, the system size, and so on (particularly for the blue phases).

A summmary of choices is:
\begin{lstlisting}
lc_q_initialisation   nematic          # uniform nematic...
lc_init_nematic       1.0_0.0_0.0      # ...with given director

lc_q_initialisation   cholesteric_x    # cholesteric with helical axis x
lc_q_initialisation   cholesteric_y    # cholesteric with helical axis y
lc_q_initialisation   cholesteric_z    # cholesteric with helical axis z

lc_q_initialisation   o8m              # BPI high chirality limit
lc_q_initialisation   o2               # BPII high chirality limit
lc_q_initialisation   o5
lc_q_initialisation   h2d              # 2d hexagonal
lc_q_initialisation   h3da             # 3d hexagonal BP A
lc_q_initialisation   h3db             # 3d hexagonal BP B
lc_q_initialisation   dtc              # double twist cylinders

lc_q_initialisation   bp3

lc_q_initialisation   cf1_x            # cholesteric ``finger'' axis x
lc_q_initialisation   cf1_y            # cholesteric ``finger'' axis y
lc_q_initialisation   cf1_z            # cholesteric ``finger'' axis z

lc_q_initialisation   cf1_fluc_x       # as cf1_x with random perterbations
lc_q_initialisation   cf1_fluc_y       # as cf1_y with random perturbations
lc_q_initialisation   cf1_flux_z       # as cf1_z with random perturbations

lc_q_initialistion    random           # with randomly chosen unit director
\end{lstlisting}
Note many of the initialiations require an initial amplitude of order,
which should be set via
\begin{lstlisting}
lc_q_init_amplitude   0.01             # initial amplitude of order A
\end{lstlisting}
For example, if an initial uniform nematic is requested with
unit director $n_\alpha$, the corresponding initial tensor will be
$$
Q_{\alpha\beta} = 
{\textstyle \frac{1}{2}} A (3 n_\alpha n_\beta - \delta_{\alpha\beta}).
$$

\vfill
\pagebreak
